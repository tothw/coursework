\documentclass[letterpaper,12pt,oneside,onecolumn]{article}
\usepackage[margin=1in, bottom=1in, top=1in]{geometry} %1 inch margins
\usepackage{amsmath, amssymb, amstext}
\usepackage{fancyhdr}
\usepackage{algorithm}
\usepackage{algpseudocode}
\usepackage{mathtools}

\DeclarePairedDelimiter{\ceil}{\lceil}{\rceil}
\DeclarePairedDelimiter\floor{\lfloor}{\rfloor}

%Macros
\newcommand{\A}{\mathbb{A}} \newcommand{\C}{\mathbb{C}}
\newcommand{\D}{\mathbb{D}} \newcommand{\F}{\mathbb{F}}
\newcommand{\N}{\mathbb{N}} \newcommand{\R}{\mathbb{R}}
\newcommand{\T}{\mathbb{T}} \newcommand{\Z}{\mathbb{Z}}
\newcommand{\Q}{\mathbb{Q}}
 
 
\newcommand{\cA}{\mathcal{A}} \newcommand{\cB}{\mathcal{B}}
\newcommand{\cC}{\mathcal{C}} \newcommand{\cD}{\mathcal{D}}
\newcommand{\cE}{\mathcal{E}} \newcommand{\cF}{\mathcal{F}}
\newcommand{\cG}{\mathcal{G}} \newcommand{\cH}{\mathcal{H}}
\newcommand{\cI}{\mathcal{I}} \newcommand{\cJ}{\mathcal{J}}
\newcommand{\cK}{\mathcal{K}} \newcommand{\cL}{\mathcal{L}}
\newcommand{\cM}{\mathcal{M}} \newcommand{\cN}{\mathcal{N}}
\newcommand{\cO}{\mathcal{O}} \newcommand{\cP}{\mathcal{P}}
\newcommand{\cQ}{\mathcal{Q}} \newcommand{\cR}{\mathcal{R}}
\newcommand{\cS}{\mathcal{S}} \newcommand{\cT}{\mathcal{T}}
\newcommand{\cU}{\mathcal{U}} \newcommand{\cV}{\mathcal{V}}
\newcommand{\cW}{\mathcal{W}} \newcommand{\cX}{\mathcal{X}}
\newcommand{\cY}{\mathcal{Y}} \newcommand{\cZ}{\mathcal{Z}}

%Page style
\pagestyle{fancy}

\listfiles

\raggedbottom

\rhead{MATH 128 Tutorial - Monday May 9} %CHANGE DATE as appropriate
\renewcommand{\headrulewidth}{1pt} %heading underlined
%\renewcommand{\baselinestretch}{1.2} % 1.2 line spacing for legibility (optional)

\begin{document}
\section*{Section 5.5}
\paragraph{Problem 91}
If $a$ and $b$ are positive numbers, show that $$\int_0^1 x^a(1-x)^b dx = \int_0^1 x^b(1-x)^a dx.$$
\paragraph{Solution}
Let $x = 1-u$. Then $dx = -du$. So we have that
\begin{align*}
\int_{x=0}^{x=1} x^a(1-x)^b dx &= \int_{u=1}^{u=0} (1-u)^a u^b(-1)du \\
&= -1 \int_1^0 u^b(1-u)^a du \\
&= \int_0^1 x^b(1-x)^a dx.
\end{align*}
The last step applies the rule for resolving improper integrals:
$$\int_a^b f(x) dx = -1 \int_b^af(x) dx.$$
\section*{Section 7.1}
\paragraph{Problem 52}
Use integration by parts to show that
$$\int x^n e^x dx = x^ne^x - n\int n^{n-1} e^x dx.$$
\paragraph{Solution}
Let $u = x^n$ and let $dv = e^xdx$. Then $$du = n x^{n-1}dx$$ and $$v = \int e^x dx = e^x$$
So by the integration by parts formula:
$$ \int u dv = uv - \int v du$$
we have that
\begin{align*} \int x^n e^x dx &= x^ne^x - \int e^x n x^{n-1}dx\\ &= x^ne^x - n\int x^{n-1} e^x dx.\end{align*}
\section*{Section 7.2}
\paragraph{Problem 55 - Modified}
Find the average value of the function $$f(x) = sin^3xcos^2x$$ on the interval $$[0,\pi/2].$$
\paragraph{Solution}
The average value a function $f$ on interval $[a,b]$ is given by $$f_{ave} = \frac{1}{b-a} \int_a^b f(x) dx.$$
So applying this to our problem gives:
\begin{align*}
f_{ave} &= \frac{1}{\pi/2 - 0}\int_{0}^{\pi/2} sin^3xcos^2x dx \\
&= \frac{2}{\pi} \int_{0}^{\pi/2} (1-cos^2x)cos^2x\ sinx dx.
\end{align*}
Now we perform the substitution $cosx = u$ with $du = -sinxdx$ obtaining
\begin{align*}
f_{ave} &= -1\frac{2}{\pi}\int_{cos^{-1}u = 0}^{cos^{-1}u=\pi/2} (1-u^2)u^2du \\
&=\frac{-2}{\pi}\int_{u=1}^{u=0} u^2 - u^4 du \\
&=\frac{-2}{\pi}[\frac{u^3}{3} -\frac{u^5}{5}\mid^0_1 \\
&= \frac{-2}{\pi}[0 - 0 - (\frac{1}{3} - \frac{1}{5})] \\
&= \frac{-2}{\pi}\cdot\frac{3 - 5}{15} \\
&= \frac{4}{15\pi}. 
\end{align*}
\end{document}
