\documentclass[letterpaper,12pt,oneside,onecolumn]{article}
\usepackage[margin=1in, bottom=1in, top=1in]{geometry} %1 inch margins
\usepackage{amsmath, amssymb, amstext}
\usepackage{fancyhdr}
\usepackage{algorithm}
\usepackage{algpseudocode}
\usepackage{mathtools}

\DeclarePairedDelimiter{\ceil}{\lceil}{\rceil}
\DeclarePairedDelimiter\floor{\lfloor}{\rfloor}

%Macros
\newcommand{\A}{\mathbb{A}} \newcommand{\C}{\mathbb{C}}
\newcommand{\D}{\mathbb{D}} \newcommand{\F}{\mathbb{F}}
\newcommand{\N}{\mathbb{N}} \newcommand{\R}{\mathbb{R}}
\newcommand{\T}{\mathbb{T}} \newcommand{\Z}{\mathbb{Z}}
\newcommand{\Q}{\mathbb{Q}}
 
 
\newcommand{\cA}{\mathcal{A}} \newcommand{\cB}{\mathcal{B}}
\newcommand{\cC}{\mathcal{C}} \newcommand{\cD}{\mathcal{D}}
\newcommand{\cE}{\mathcal{E}} \newcommand{\cF}{\mathcal{F}}
\newcommand{\cG}{\mathcal{G}} \newcommand{\cH}{\mathcal{H}}
\newcommand{\cI}{\mathcal{I}} \newcommand{\cJ}{\mathcal{J}}
\newcommand{\cK}{\mathcal{K}} \newcommand{\cL}{\mathcal{L}}
\newcommand{\cM}{\mathcal{M}} \newcommand{\cN}{\mathcal{N}}
\newcommand{\cO}{\mathcal{O}} \newcommand{\cP}{\mathcal{P}}
\newcommand{\cQ}{\mathcal{Q}} \newcommand{\cR}{\mathcal{R}}
\newcommand{\cS}{\mathcal{S}} \newcommand{\cT}{\mathcal{T}}
\newcommand{\cU}{\mathcal{U}} \newcommand{\cV}{\mathcal{V}}
\newcommand{\cW}{\mathcal{W}} \newcommand{\cX}{\mathcal{X}}
\newcommand{\cY}{\mathcal{Y}} \newcommand{\cZ}{\mathcal{Z}}

%Page style
\pagestyle{fancy}

\listfiles

\raggedbottom

\rhead{MATH 128 Tutorial - Monday May 16} %CHANGE DATE as appropriate
\renewcommand{\headrulewidth}{1pt} %heading underlined
%\renewcommand{\baselinestretch}{1.2} % 1.2 line spacing for legibility (optional)

\begin{document}
\section*{Trigonometric Substitution}
\paragraph{Problem}
Solve the following integral using a trigonometric substitution:
$$\int \frac{\sqrt{16-x^2}}{x^2}dx.$$
\paragraph{Solution}
Let $x = 4\sin\theta$ with $-\frac{\pi}2 \leq \theta \leq \frac{\pi}{2}$ (so that $\sin\theta \geq 0$). Then $dx = 4\cos\theta d\theta$. Applying the substitution we have: 
\begin{align*}
\int \frac{\sqrt{16-x^2}}{x^2}dx &= \int\frac{\sqrt{16-4^2\sin^2\theta}}{16\sin^2\theta}\cdot 4\cos\theta d\theta \\
&=\int\frac{4\cos\theta}{16\sin^2\theta} \cdot 4\cos\theta d\theta \\
&=\int\frac{16\cos^2\theta}{16\sin^2\theta}d\theta\\
&=\int\cot^2\theta d\theta \\
&=\int(\csc^2\theta - 1) d\theta \\
&= -\cot\theta - \theta + c.
\end{align*}
We want the solution in terms of $x$ so we draw the triangle \\\\\\\\\\\\\\\\\\\\\\\\\\\\
So we have $$\cot\theta = \frac{\sqrt{16-x^2}}{x}$$ and $$\theta = \frac{\sin^{-1}\theta}{4}.$$
Therefore our final solution is
$$-\frac{\sqrt{16-x^2}}{x} - \frac{\sin^{-1}\theta}{4} + c.$$
\section*{7.4}
\paragraph{Problem 47}
Make a substitution to express the integrand as a rational function and then evaluate the integral:
$$\int\frac{e^{2x}}{e^{2x} + 3e^x + 2}dx.$$
Let $u = e^x$. Then $du = e^xdx$. Observe that $e^{2x} = e^xe^x = (e^x)^2$. Thus substituting into our integral we have:
$$\int \frac{u}{u^2 + 3u + 2}du
 = \int\frac{u}{(u+2)(u+1)}du.$$
 Now to apply partial fractions we need:
 $$\frac{u}{(u+2)(u+1)} = \frac{A}{u+2} + \frac{B}{u+1}.$$
 That is, $$u = Au +A +Bu + 2B.$$
 So $$A+B =1 \quad \text{and} \quad 2B + A = 0$$
 Simplifying we get
 $$A = -2B$$
 and
 $$-2B + B = 1 \implies B = -1$$
 and hence
 $$A = 2.$$
 Substituting back into our integral we get:
 $$\int(\frac{2}{u+2} + \frac{-1}{u+1})du$$
 Which is equal to
 $$2\ln|u+2| -\ln|u+1| + c.$$
 In terms of $x$ we have:
 $$2\ln|e^x + 2| -\ln|e^x+1| +c.$$
 \section*{7.5}
 \paragraph{Problem 18}
 Solve the definite integral
 $$\int_{1}^{4} \frac{e^{\sqrt{t}}}{\sqrt{t}}dt.$$
 We use the rationalizing substitution $x = t^\frac{1}{2}.$ Then $dx = \frac{1}{2}t^\frac{-1}{2}dt$, and $x^2 = t$. Thus substituting into our integral we have
 $$\int_{x^2=1}^{x^2=4} e^x \cdot 2dx.$$
 Simplifying we obtain:
 $$2\int_{x=1}^{x=2} e^x dx.$$
 Which gives as final solution:
 $$2(e^2 - e).$$
 \end{document}
