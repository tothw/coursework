\documentclass[letterpaper,12pt,oneside,onecolumn]{report}
\usepackage{amsmath, amssymb, amstext}
\usepackage{fancyhdr}
\usepackage{algorithm}
\usepackage{algpseudocode}
\pagestyle{fancy}

\listfiles

\setlength{\hoffset}{0pt}			% 1 inch left margin
\setlength{\oddsidemargin}{0pt}		% 1 inch left margin
\setlength{\voffset}{0pt}			% 1 inch top margin
\setlength{\marginparwidth}{0pt}	% no margin notes
\setlength{\marginparsep}{0pt}		% no margin notes
\setlength{\textwidth}{6.375in}
\raggedbottom

\rhead{William Justin Toth 685 2} %CHANGE n to ASSIGNMENT NUMBER ijk TO COURSE CODE
\renewcommand{\headrulewidth}{0pt}
%\renewcommand{\baselinestretch}{1.2} % 1.2 line spacing for legibility (optional)

\begin{document}
\section*{1.33}
\paragraph{}
Let $p$ be a prime and let $q$ be a prime that divides $p-1$.
\subsection*{a}
\paragraph{}
Let $a \in \mathbb{F}^*_p$ and let $b = a^{(p-1)/q}$. Suppose that $b \neq 1$. Then $b^q = a^{q(p-1)/q} = a^{p-1}$. Thus, by Fermat's Little Theorem, $b^q \equiv 1\ (p)$. So by proposition $1.29$, $ord(b) \mid q$. But $q$ is prime, so $ord(b) = 1$ or $ord(b) = q$. Since $b \neq 1$, $ord(b) = q$. Therefore $b = 1$ or $b$ has order $q$.$\blacksquare$
\subsection*{b}
\paragraph{}
By Theorem $1.30$ there exists a generator $g \in \mathbb{F}_p^*$ such that $\mathbb{F}_p^* = \{1, g, g^2,\dots,g^{p-2} \}$. Therefore $|\mathbb{F}_p^*| = p - 1$, and $|\{a \in \mathbb{F}_p^*: a^{(p-1)/q} \neq 1\}| = |\mathbb{F}^*_p| - |\{a \in \mathbb{F}^*_p : a^{(p-1)/q} = 1\}|$. Now there exists $k \in \{1, \dots, p-1\}$ such that $a = g^k$. Thus $a^{(p-1)/q} = g^{k(p-1)/q}$. Since $g^{p-1} = 1$, $a^{(p-1)/q} = 1$ if and only if $k = q$. Therefore $|\{a \in \mathbb{F}^*_p : a^{(p-1)/q} = 1\}| = 1$, so $|\{a \in \mathbb{F}_p^*: a^{(p-1)/q} \neq 1\}| = p - 1 - 1 = p - 2$. So the probablility of success is $\frac{p-2}{p-1}$. $\blacksquare$
\section*{2.25}
\paragraph{}
Supppose $n = pq$ with $p$ and $q$ distinct odd primes.
\subsection*{a}
\paragraph{}
Suppose that $gcd(a,pq) = 1$. Then $a \in \mathbb{Z}^*_n$. Suppose that $x^2 \equiv a\ (n)$ has any solutions. Since $a \in \mathbb{Z}^*_n$ and $p,q$ are odd primes, there exist $y, z \in \mathbb{Z}$ such that $y^2 \equiv a\ (p)$ and $z^2 \equiv a\ (q)$. Since $p \neq q$, $y$ and $z$ are distinct. Since $p,q > 2$, $-y \neq y$ and $-z \neq z$. Therefore $-y,y$ are square roots of $a$ mod $p$, and $-z, z$ are square roots of $a$ mod $q$. This gives rise to four different systems of linear congruences to find $x$:
\begin{align}
x &\equiv -y\ (p) &x &\equiv -z\ (q)\\
x &\equiv y\ (p) &x &\equiv -z\ (q)\\
x &\equiv -y\ (p) &x &\equiv z\ (q)\\
x &\equiv y\ (p) &x &\equiv z\ (q)
\end{align}
\paragraph{}
By the Chinese Remainder Theorem each system has a unique solution mod $pq = n$. That is, $x^2 \equiv a\ (n)$ has four solutions. $\blacksquare$
\subsection*{b}
\paragraph{}
Let $a \in \mathbb{Z}_n^*$. Suppose a machine has given us four distinct solutions to the congruence $x^2 \equiv a\ (n)$. Call those solutions $x_1,x_2,x_3,x_4$. Exactly one of $x_2, x_3, x_4$ is congruent to $\pm x_1$ mod $n$. Say without loss of generality $x_2$ is such that $x_1 \not\equiv \pm x_2\ (n)$. Then $x_1 \pm x_2 \not\equiv 0\ (n)$. That is $n \nmid x_1 \pm x_2$. Now, since $x_1^2 \equiv a\ (n)$ and $x_2^2 \equiv a\ (n)$ we have:
\begin{align*}
&\ &x_1^2 &\equiv x_2^2 &(n)\\
&\Rightarrow &x_1^2 - x_2^2 &\equiv 0 &(n) \\
&\Rightarrow &(x_1 + x_2)(x_1 - x_2) &\equiv 0 &(n) \\
\end{align*}
\paragraph{}
That is $n \mid (x_1 + x_2)(x_1 - x_2)$. But $n \nmid (x_1 + x_2)$ and $n \nmid (x_1 - x_2)$. Then, since $n = pq$, one of $p,q$ divides $(x_1 + x_2)$ and the other divides $(x_1 - x_2)$. Say without loss of generality that $p \mid (x_1 + x_2)$. Therefore $gcd(x_1 + x_2, n) = p$. Apply the Euclidean algorithm to compute $gcd(x_1 + x_2,n)$ to obtain $p$. Divide $p$ into $n$ to obtain $q$. In this manner the prime factorization of $n$ has been found from the four solutions of $x^2 \equiv a\ (n)$.$\blacksquare$
\section*{3.15}
\paragraph{}
See attached printout from Sage Math, titled Miller-Rabin Test.
\section*{3.39}
Let $p$ be a prime satisfying $p \equiv 3\ (4)$.
\subsection*{a}
Let $a$ be a quadratic residue modulo $p$. Let $b \equiv a^{\frac{p+1}{4}}\ (p)$. Notice $\frac{p+1}{4}$ is an integer since $p \equiv 3\ (4) \Rightarrow p + 1 \equiv 0\ (4)$. Let $g$ be a generator of $\mathbb{Z}_p$. Then there exists a $k \in mathbb{Z}$ such that $g^k \equiv a\ (p)$. Since $a$ is a quadratic residue $k$ is even. Thus there exists $b \in \mathbb{Z}$ such that $k = 2b$. Now we have:
\begin{align*}
b^2 &\equiv a^{\frac{p+1}{2}} &(p)\\
&\equiv a^{1 + \frac{p-1}{2}} &(p)\\
&\equiv a\cdot a^{\frac{p-1}{2}} &(p)\\
&\equiv a\cdot g^{\frac{2b(p-1)}{2}} &(p)\\
&\equiv a\cdot (g^{p-1})^b &(p)\\
&\equiv a\cdot 1^b &(p)\\
&\equiv a &(p)
\end{align*} $\blacksquare$
\subsection*{b}
\paragraph{i}
Let $b^2 \equiv 116\ (587)$. Since $587 \equiv 3\ (4)$, $b \equiv 116^{\frac{587 + 1}{4}}\ (587)$. Thus $b \equiv 65\ (587)$.
\paragraph{ii}
Let $b^2 \equiv 3217 \ (8627)$. Since $8627 \equiv 3\ (4)$, $b \equiv 3217 ^{\frac{8627 + 1}{4}}\ (8627)$. Thus $b \equiv 2980\ (8627)$.
\paragraph{iii}
Let $b^2 \equiv 9109\ (10663)$. Since $9109^{\frac{10663-1}{2}} \equiv -1\ (10663)$, $9109$ is not a quadratic residuue modulo $(10663)$. Therefore no solution for $b$ exists.
\section*{Goldwasser-Micali is IND-CPA}
\paragraph{}
Consider the Goldwasser-Micali Cryptosystem. Let $l$ be the security parameter. Let $p,q$ be prime numbers such that $2^{l-1} < p,q < 2^l$. Let $n = pq$. Let $x \in \mathbb{Z}_n^*$ such that $\left(\frac{x}{n}\right) = 1$ and $x$ is not square modulo $n$. In this case that happens exactly when $\left(\frac{x}{p}\right) = -1$ and $\left(\frac{x}{q}\right) = -1$. The public key will be $(x,n)$. The private key will be $(p,q)$. Let $\mathcal{M} = \{0,1\}$ be the message space. Let $E(m) = x^mr^2$, where $r$ is chosen uniformly at random from elements of $\mathbb{Z}_n^*$, be the encryption function. Let $D(c) = 0$ if $\left(\frac{c}{p}\right) = 1$ and $D(c) = 1$ otherwise.
\paragraph{}
Suppose the Quadratic Residuosity Assumption holds. That is, given $n=pq$ and $x \in \mathbb{Z}_n^*$ such that $\left(\frac{x}{n}\right) = 1$ it is infeasible to determine if $x$ is a square modulo $n$. We will show Goldwasser-Micali is IND-CPA under this assumption. A cryptosystem is IND-CPA if there does not exist a probabilistic polynomial time algorithm which solves the IND-CPA problem with probability $P$ where $max(P,\frac{1}{2}) - \frac{1}{2}$ is non-negligible. The IND-CPA problem is as follows: Given $m_1, m_2 \in \mathcal{M}$ such that $m_1 \neq m_2$ and $c = E(m_i)$ for random $i=1,2$, and given the public key determine  $i$ such that $c=E(m_i)$. Suppose for contradiction that Goldwasser-Micali is not IND-CPA. Let $\mathcal{A}$ be the algorithm that solves the IND-CPA problem for Goldwasser-Micali in probabilistic polynomial time.
\paragraph{}
Let $n = pq$ and $x \in \mathbb{Z}_n^*$, such that $\left(\frac{x}{n}\right) = 1$, be an arbitrary input to the Quadratic Residuosity Problem. Let $m_1 = 0$, and $m_2 = 1$. Let $r = 1 \in \mathbb{Z}_n^*$. Run $\mathcal{A}$ with public key $(x,n)$, messages $m_1, m_2$ and ciphertext $c = E(m_2) = x^mr^2 = x$. Let $i$ be the output of $\mathcal{A}$. If $i = 0$ then $D(x) = 0$ and therefore $\left(\frac{x}{p}\right) = 1$ and $\left(\frac{x}{q}\right) = 1$. So $x$ is a square modulo $n=pq$. Conversely if $i=1$ them $D(x) = 1$ and therefore $\left(\frac{x}{p}\right) = -1$ and $\left(\frac{x}{q}\right) = -1$. So $x$ is not a square modulo $n=pq$. Thus if $\mathcal{A}$ exists it can be used to solve the Quadratic Residuosity Problem in probabilistic polynomial time, a contradiction. Therefore Goldwasser-Micali is IND-CPA. $\blacksquare$

\end{document}
