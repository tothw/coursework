\documentclass[letterpaper,12pt,oneside,onecolumn]{report}
\usepackage{amsmath, amssymb, amstext}
\usepackage{fancyhdr}
\usepackage{algorithm}
\usepackage{algpseudocode}
\pagestyle{fancy}

\listfiles

\setlength{\hoffset}{0pt}			% 1 inch left margin
\setlength{\oddsidemargin}{0pt}		% 1 inch left margin
\setlength{\voffset}{0pt}			% 1 inch top margin
\setlength{\marginparwidth}{0pt}	% no margin notes
\setlength{\marginparsep}{0pt}		% no margin notes
\setlength{\textwidth}{6.375in}
\raggedbottom

\rhead{William Justin Toth 685 2} %CHANGE n to ASSIGNMENT NUMBER ijk TO COURSE CODE
\renewcommand{\headrulewidth}{0pt}
%\renewcommand{\baselinestretch}{1.2} % 1.2 line spacing for legibility (optional)

\begin{document}
\section*{1.33}
\paragraph{}
Let $p$ be a prime and let $q$ be a prime that divides $p-1$.
\subsection*{a}
\paragraph{}
Let $a \in \mathbb{F}^*_p$ and let $b = a^{(p-1)/q}$. Suppose that $b \neq 1$. Then $b^q = a^{q(p-1)/q} = a^{p-1}$. Thus, by Fermat's Little Theorem, $b^q \equiv 1\ (p)$. So by proposition $1.29$, $ord(b) \mid q$. But $q$ is prime, so $ord(b) = 1$ or $ord(b) = q$. Since $b \neq 1$, $ord(b) = q$. Therefore $b = 1$ or $b$ has order $q$.$\blacksquare$
\subsection*{b}
\paragraph{}
By Theorem $1.30$ there exists a generator $g \in \mathbb{F}_p^*$ such that $\mathbb{F}_p^* = \{1, g, g^2,\dots,g^{p-2} \}$. Therefore $|\mathbb{F}_p^*| = p - 1$, and $|\{a \in \mathbb{F}_p^*: a^{(p-1)/q} \neq 1\}| = |\mathbb{F}^*_p| - |\{a \in \mathbb{F}^*_p : a^{(p-1)/q} = 1\}|$. Now there exists $k \in \{1, \dots, p-1\}$ such that $a = g^k$. Thus $a^{(p-1)/q} = g^{k(p-1)/q}$. Since $g^{p-1} = 1$, $a^{(p-1)/q} = 1$ if and only if $k = q$. Therefore $|\{a \in \mathbb{F}^*_p : a^{(p-1)/q} = 1\}| = 1$, so $|\{a \in \mathbb{F}_p^*: a^{(p-1)/q} \neq 1\}| = p - 1 - 1 = p - 2$. So the probablility of success is $\frac{p-2}{p-1}$. $\blacksquare$
\section*{2.25}
\paragraph{}
Supppose $n = pq$ with $p$ and $q$ distinct odd primes.
\subsection*{a}
\paragraph{}
Suppose that $gcd(a,pq) = 1$. Then $a \in \mathbb{Z}^*_n$. Suppose that $x^2 \equiv a\ (n)$ has any solutions. Since $a \in \mathbb{Z}^*_n$ and $p,q$ are odd primes, there exist $y, z \in \mathbb{Z}$ such that $y^2 \equiv a\ (p)$ and $z^2 \equiv a\ (q)$. Since $p \neq q$, $y$ and $z$ are distinct. Since $p,q > 2$, $-y \neq y$ and $-z \neq z$. Therefore $-y,y$ are square roots of $a$ mod $p$, and $-z, z$ are square roots of $a$ mod $q$. This gives rise to four different systems of linear congruences to find $x$:
\begin{align}
x &\equiv -y\ (p) &x &\equiv -z\ (q)\\
x &\equiv y\ (p) &x &\equiv -z\ (q)\\
x &\equiv -y\ (p) &x &\equiv z\ (q)\\
x &\equiv y\ (p) &x &\equiv z\ (q)
\end{align}
\paragraph{}
By the Chinese Remainder Theorem each system has a unique solution mod $pq = n$. That is, $x^2 \equiv a\ (n)$ has four solutions. $\blacksquare$
\end{document}
