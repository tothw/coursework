\documentclass[letterpaper,12pt,oneside,onecolumn]{report}
\usepackage{amsmath, amssymb, amstext}
\usepackage{fancyhdr}
\usepackage{algorithm}
\usepackage{algpseudocode}
\usepackage{mathtools}

\DeclarePairedDelimiter{\ceil}{\lceil}{\rceil}

\pagestyle{fancy}

\listfiles

\setlength{\hoffset}{0pt}			% 1 inch left margin
\setlength{\oddsidemargin}{0pt}		% 1 inch left margin
\setlength{\voffset}{0pt}			% 1 inch top margin
\setlength{\marginparwidth}{0pt}	% no margin notes
\setlength{\marginparsep}{0pt}		% no margin notes
\setlength{\textwidth}{6.375in}
\raggedbottom

\rhead{William Justin Toth 685 4} %CHANGE n to ASSIGNMENT NUMBER ijk TO COURSE CODE
\renewcommand{\headrulewidth}{0pt}
%\renewcommand{\baselinestretch}{1.2} % 1.2 line spacing for legibility (optional)

\begin{document}
\section*{6.9}
\paragraph{}
Let $E$ be an elliptic curve over $\mathbb{F}_p$ and let $P$ and $Q$ be points in $E(\mathbb{F}_p)$. Suppose that $Q$ is a multiple of $P$ and let $n_0 > 0$ be the smallest solution to $Q = nP$. Let $s > 0$ be the smallest solution to $sP = \infty$. Let $n$ be such that $Q = nP$. By the division algorithm there exists $i, r \in \mathbb{Z}$ such that $n= is + r$, and $0 \leq r < s$. Then $Q = (is + r)P = isP + rP = \infty + rP = rP$.
\paragraph{}
Consider the case $Q = \infty$. Then $\infty = rP$ since $Q = rP$. Now, since $0 \leq r < s$ and $s$ is minimal such that $s>0$ and $sP = \infty$, $r = 0$. Since $Q = n_oP$ and $Q = \infty$, $\infty = n_oP$. Since $n_0, s > 0$ are minimal such that $Q = n_0P$ and $\infty = sP$ respectively, and $Q = \infty$, we have that $n_0 = s$. Now $n = is + r = is + 0 = (i-1)s + s = (i-1)s + n_0$ as desired.
\paragraph{}
Now consider the case $Q \neq \infty$. Then $rP \neq \infty$. Thus $r > 0$, and so $r \geq n_0$ by minimality of $n_0$. Therefore we have $0 < n_0 \leq r < s$. So there exists $0 \leq l < s$, such that $n_0 + l = r$. Then
\begin{align*}
Q &= rP \\
&= (n_0 + l)P \\
&= n_0P + lP \\
&= Q + lP.
\end{align*}
Thus we have $Q = Q + lP$ and so $\infty = lP$. Therefore by the minimality of $s$, $l = 0$. Thus $n_0 = r$, and so $n = is + n_0$ as desired. $\blacksquare$
\end{document}
