\documentclass[letterpaper,12pt,oneside,onecolumn]{report}
\usepackage{amsmath, amssymb, amstext}
\usepackage{fancyhdr}
\pagestyle{fancy}

\listfiles

\setlength{\hoffset}{0pt}			% 1 inch left margin
\setlength{\oddsidemargin}{0pt}		% 1 inch left margin
\setlength{\voffset}{0pt}			% 1 inch top margin
\setlength{\marginparwidth}{0pt}	% no margin notes
\setlength{\marginparsep}{0pt}		% no margin notes
\setlength{\textwidth}{6.375in}
\raggedbottom

\rhead{William Justin Toth 685  1} %CHANGE n to ASSIGNMENT NUMBER
\renewcommand{\headrulewidth}{0pt}
%\renewcommand{\baselinestretch}{1.2} % 1.2 line spacing for legibility (optional)

\begin{document}

\section*{1.11}
\subsection*{a}
\paragraph{}
Suppose that there exist $u, v \in \mathbb{Z}$ satisfying 
\begin{equation}
au + bv = 1.
\end{equation}
Since $gcd(a,b) \mid a$, there exists $j \in \mathbb{Z}$ such that $a = gcd(a,b)j.$\\
Since $gcd(a,b) \mid b$, there exists $k \in \mathbb{Z}$ such that $b = gcd(a,b)k.$\\
Substituting into $(1)$ we obtain $gcd(a,b)ju + gcd(a,b)kv = 1$.\\
Thus $gcd(a,b)(ju + kv) = 1$, so $gcd(a,b) \mid 1$.\\
Then $gcd(a,b) \leq 1$, but $gcd(a,b) \geq 1$ by definition, and therefore $gcd(a,b) = 1$.$\blacksquare$
\subsection*{b}
Suppose there exist $u,v \in \mathbb{Z}$ such that $au+bv=6$. It is not necessarily true that $gcd(a,b)=6$. Consider the counterexample $a=b=3$ then if $u=v=1 \in \mathbb{Z}$ we have $au + bv = 3+3 = 6$ but $gcd(a,b) = gcd(3,3) = 3 \neq 6$.\\
Consider the equation $gcd(a,b)(ju + kv) = 6$ obtained as described in part a. This indicates that $gcd(a,b) \mid 6$.\\
That is, if $au + bv = 6$ then $gcd(a,b) \mid 6$.\\
Now suppose that $gcd(a,b) \mid 6$. Then there exists $k \in \mathbb{Z}$ such that $gcd(a,b)k = 6$.\\
By Bezout's identity there exists $u_0, v_0 \in \mathbb{Z}$ such that $gcd(a,b) = au_0 + bv_0$.\\
Then $6 = gcd(a,b)k = au_0k + bv_0k = au+bv$ where $u = u_0k$ and $v = v_0k$.\\
Thus there exist integers $u,v \in \mathbb{Z}$ satisfying $au + bv = 6$ if and only if $gcd(a,b) \mid 6$. $\blacksquare$
\subsection*{c}
Suppose that $(u_1,v_1)$ and $(u_2,v_2)$ are two solutions in integers to the equation $au + bv = 1$.\\
By $(a)$ this implies that $gcd(a,b) = 1$, that is $a$ and $b$ are relatively prime.\\
Further since $au_1 + bv_1 = 1$ and $au_2 + bv_2 = 1$, by subtracting we obtain:\\
$a(u_1-u_2) + b(v_1-v_2) = 0$\\
$\Rightarrow a(u_1-u_2) = b(v_2-v_1)$\\
$\Rightarrow a \mid b(v_2-v_1)$\\
$\Rightarrow a \mid v_2 - v_1$ since $gcd(a,b) = 1$.\\
Similarly $a(u_1-u_2) + b(v_1-v_2) = 0$\\
$\Rightarrow b(v_1-v_2) = a(u_2-u_1)$\\
$\Rightarrow b \mid a(u_2-u_1)$\\
$\Rightarrow b \mid u_2 - u_1$ since $gcd(a,b) = 1$.\\ $\blacksquare$
\subsection*{d}
Let $g = gcd(a,b)$ and let $(u_0, v_0)$ be a solution in integers to $au + bv = g$.\\
Let $(u_1, v_1)$ be a solution in integers to $au + bv = g$.\\
Then $\frac{a}{g} \mid v_1-v_0$ and $\frac{b}{g} \mid u_1-u_0$ by part $(c)$.\\
Thus we can choose $k$ in the integers and write $u_1 = u_0 + k\frac{b}{g}$ and $v_1 = v_0 - k \frac{a}{g}$.\\
It remains to validate the solution.\\
Indeed we have $au_1 + bv_1 = a(u_0 + k \frac{b}{g}) + b(v_0 - k\frac{a}{g}) = au_0 + k \frac{ab}{g} + bv_0 - k\frac{ab}{g} = au_0 + bv_0 = g$.$\blacksquare$  
\section*{1.24}
\subsection*{a}
Suppose $x \equiv 3\ (7)$ and $x \equiv 4\ (9)$.\\
Then there exists $y \in \mathbb{Z}$ such that $x = 3 + 7y$.
\begin{align*}
3 + 7y &\equiv 4 &(9)\\
7y &\equiv 1 &(9)\\
4 \cdot 7y &\equiv 4 \cdot 1 &(9)\\
y &\equiv 4 &(9)
\end{align*}
So there exists $z \in \mathbb{Z}$ such that $y = 4 + 9z$.\\
Then $x = 3 + 7(4 + 9z) = 3 + 28 + 63z = 31 + 63z$.\\
Therefore $x \equiv 31\ (63)$.
\subsection*{b}
Suppose $x \equiv 13\ (71)$ and $x \equiv 41\ (97)$.\\
Then there exists $y \in \mathbb{Z}$ such that $x = 13 + 71y$.
\begin{align*}
13 + 71y &\equiv 41 &(97)\\
71y &\equiv 28 &(97)\\
41 \cdot 71y &\equiv 41 \cdot 28 &(97)\\
y &\equiv 81 &(97)\\
\end{align*}
So there exists $z \in \mathbb{Z}$ such that $y = 81 + 97z$.\\
Then $x = 13 + 71(81 + 97z) = 13 + 5751 + 71\cdot97z = 5764 + 6887z$.\\
Therefore $ x \equiv 5764\ (6887)$.
\subsection*{c}
Suppose $x \equiv 4\ (7)$, $x \equiv 5\ (8)$ and $x \equiv 11\ (15)$.\\
Then there exists $a \in \mathbb{Z}$ such that $x = 4 + 7a$.\\
\begin{align*}
4 + 7a &\equiv 5 &(8)\\
7a &\equiv 1 &(8)\\
7\cdot7a &\equiv 7\cdot1 &(8)\\
a &\equiv 7 &(8)\\
\end{align*}
So there exists $b \in \mathbb{Z}$ such that $a = 7 + 8b$.\\
Then $x = 4 + 7(7 + 8b) = 53 + 56b$.\\
\begin{align*}
53 + 56b &\equiv 11 &(15)\\
56b &\equiv -42 &(15)\\
11b &\equiv 3 &(15)\\
11\cdot11b &\equiv 11\cdot3 &(15)\\
b &\equiv 3 &(15)\\
\end{align*}
Then there exists $c \in \mathbb{Z}$ such that $b = 3 + 15c$.\\
So $x = 53 + 56(3 + 15c) = 221 + 840c$.\\
Therefore $x \equiv 221\ (840)$.\\
\subsection*{d}
Suppose that $gcd(m,n) = 1$ for $m,n \in \mathbb{Z}$.\\
Let $a,b \in \mathbb{Z}$.\\
Consider the system of congruences $x \equiv a\ (m)$ and $x \equiv b\ (n)$.\\
Then there exists $y \in \mathbb{Z}$ such that $x = a + my$.\\
\begin{align*}
a + my &\equiv  b &(n)\\
my &\equiv b-a &(n)\\
\end{align*}
Since $gcd(m,n) = 1$ m has an inverse mod $m$, call it $m^{-1}$.
Then we have:
\begin{align*}
m^{-1}my &\equiv m^{-1}(b-a) &(n)\\
y &\equiv m^{-1}(b-a) &(n)\\
\end{align*}
Thus there exists $z \in \mathbb{Z}$ such that $y = m^{-1}(b-a) + nz$.\\
Therefore $x = a + m(m^{-1}(b-a) + nz) = (1-mm^{-1})a + mm^{-1}b + mnz$.\\
Thus $x \equiv (1-mm^{-1})a + mm^{-1}b\ (mn)$.\\
That is, a solution exists.$\blacksquare$\\
To show that $gcd(m,n)=1$ is necessary, consider the following situation:\\
Let $m=2$, $n=4$. Then $gcd(m,n) = 2 \neq 1$.\\
Consider $x\equiv1\ (2)$ and $x\equiv2\ (4)$.\\
Then there exists $y \in \mathbb{Z}$ such that $x = 1 + 2y$.
So $2y \equiv 1\ (4)$.\\
But,
\begin{align*}
2\cdot0 &\equiv 0 &(4)\\
2\cdot1 &\equiv 2 &(4)\\
2\cdot2 &\equiv 0 &(4)\\
2\cdot3 &\equiv 2 &(4)\\ 
\end{align*}
so no solution exists. 
\section*{1.27}
Consider the congruence $ax \equiv c\ (m)$.
\subsection*{a}
\paragraph{}
Suppose there exists a solution $x \in \mathbb{Z}$ to the congruence.\\
Then there exists $k \in \mathbb{Z}$ such that $ax = c + km$.\\
Then $ax - km = c$.\\
Since $gcd(a,m) \mid a$ and $gcd(a,m) \mid m$, we have $gcd(a.m) \mid ax - km = c$.
\paragraph{}
Now suppose $gcd(a,m) \mid c$.\\
Then there exists $k \in \mathbb{Z}$ such that $k \cdot gcd(a,m) = c$.\\
By Bezout's Identity, there exists $x_0, y_0 \in \mathbb{Z}$ such that $ax_0 + my_0 = gcd(a,m)$.\\
Then $a(kx_0) + m(ky_0) = k\cdot gcd(a.m) = c$.\\
Letting $x = kx_0$ in above we obtain $ax \equiv c\ (m)$.\\
That is, a solution to the congruence exists.$\blacksquare$
\subsection*{b}
Suppose there is a solution to the congruence.\\
Then by part a $gcd(a,m) \mid c$. Choose $z \in \mathbb{Z}$ such that $z\cdot gcd(a,m)=c$.\\
By the Extended Euclidean Algorithm $au + mv = gcd(a,m)$ always has a solution, say $(u_0, v_0)$.\\
Further, all solutions are of the form $(u,v)$ where:
\begin{align*}
u &= u_0 + \frac{mk}{gcd(a,m)}\\
v &= v_0 - \frac{ak}{gcd(a,m)}\\
\end{align*}
for some $k \in \mathbb{Z}$.\\
Multiplying through by $z$ we obtain $az(u_0 + \frac{mk}{gcd(a,m)}) + mz(v_0 - \frac{ak}{gcd(a,m)}) = c$.\\
So all solutions of the congruence are of the form $x = z(u_0 + \frac{mk}{gcd(a,m)})$ for some $k \in \mathbb{Z}$.\\
We claim the solutions are periodic mod $m$  with period $gcd(a,m)$.\\
Let $k \in \mathbb{Z}$. Let $x = z(u_0 + \frac{m(k + gcd(a,m))}{gcd(a,m)})$.\\
Then we have:
\begin{align*}
x &\equiv z(u_0 + \frac{m(k + gcd(a,m))}{gcd(a,m)}) &(m)\\
 &\equiv z(u_0 + \frac{mk}{gcd(a,m)}) + zm &(m)\\
 &\equiv z(u_0 + \frac{mk}{gcd(a,m)}) &(m)
\end{align*}
Since the solutions are periodic mod $m$ with period $gcd(a,m)$ there are at most $gcd(a,m)$ solutions.\\
To see that there are at least $gcd(a,m)$ solutions notice that:\\
$gcd(a,m) \nmid 0,1, \dots, gcd(a,m)-1$.\\
Thus each $k \in \{0,1, \dots, gcd(a,m)-1\}$ will emit a distinct solution to the congruence.\\
Therefore there are exactly $gcd(a,m)$ distinct solutions to the congruence. $\blacksquare$
\section*{5.46}
Suppose we have a shift cipher such that for all $k \in K$, $f_K(k) = \frac{1}{|K|}$. That is to say the keys are chosen uniformly.
\subsection*{a}
\paragraph{}
Let $c \in C$.\\
Then $d_k(c) : K \rightarrow M$ is a bijection since every $(c,m)$ pair corresponds uniquely to a key $k \in K$.\\
Since $K = M = C = \{0,1,\dots,25\}^l$ where $l$ is the message length in fact we have $d_k(c) : M \rightarrow M$ is a permutation of $M$.\\
Thus $\sum_{k \ in K}f_M(d_k(c)) = \sum_{m \in M} f_M(m) = 1$.$\blacksquare$
\subsection*{b}
$f_C(c) = \sum_{k \in K} f_K(k)f_M(d_k(c)) = \frac{1}{|K|} \sum_{k \in K}f_M(d_k(c)) = \frac{1}{|K|}$.
\subsection*{c}
\paragraph{}
Let $m \in M$.\\
Then, since we are using a shift cipher, there is a unique key, $k$, such that $d_k(c) = m$.\\
Thus given a message, the probability of a certain ciphertext is uniquely determined by the probably of choosing $k$.\\
That is to say, $f_{C \mid M}(c \mid m) = f_K(k) = \frac{1}{K} = f_C(c)$.\\
Therefore the cryptosystem has perfect secrecy. $\blacksquare$
\end{document}
