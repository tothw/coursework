\documentclass[letterpaper,12pt,oneside,onecolumn]{article}
\usepackage[margin=1in, bottom=1in, top=1in]{geometry} %1 inch margins
\usepackage{amsmath, amssymb, amstext}
\usepackage{fancyhdr}
\usepackage{algorithm}
\usepackage{algpseudocode}
\usepackage{mathtools}

\DeclarePairedDelimiter{\ceil}{\lceil}{\rceil}

%Macros
\newcommand{\A}{\mathbb{A}} \newcommand{\C}{\mathbb{C}}
\newcommand{\D}{\mathbb{D}} \newcommand{\F}{\mathbb{F}}
\newcommand{\N}{\mathbb{N}} \newcommand{\R}{\mathbb{R}}
\newcommand{\T}{\mathbb{T}} \newcommand{\Z}{\mathbb{Z}}
\newcommand{\Q}{\mathbb{Q}}
 
 
\newcommand{\cA}{\mathcal{A}} \newcommand{\cB}{\mathcal{B}}
\newcommand{\cC}{\mathcal{C}} \newcommand{\cD}{\mathcal{D}}
\newcommand{\cE}{\mathcal{E}} \newcommand{\cF}{\mathcal{F}}
\newcommand{\cG}{\mathcal{G}} \newcommand{\cH}{\mathcal{H}}
\newcommand{\cI}{\mathcal{I}} \newcommand{\cJ}{\mathcal{J}}
\newcommand{\cK}{\mathcal{K}} \newcommand{\cL}{\mathcal{L}}
\newcommand{\cM}{\mathcal{M}} \newcommand{\cN}{\mathcal{N}}
\newcommand{\cO}{\mathcal{O}} \newcommand{\cP}{\mathcal{P}}
\newcommand{\cQ}{\mathcal{Q}} \newcommand{\cR}{\mathcal{R}}
\newcommand{\cS}{\mathcal{S}} \newcommand{\cT}{\mathcal{T}}
\newcommand{\cU}{\mathcal{U}} \newcommand{\cV}{\mathcal{V}}
\newcommand{\cW}{\mathcal{W}} \newcommand{\cX}{\mathcal{X}}
\newcommand{\cY}{\mathcal{Y}} \newcommand{\cZ}{\mathcal{Z}}

%Page style
\pagestyle{fancy} 

\listfiles

\raggedbottom

\rhead{William Justin Toth 650 A5} %CHANGE n to ASSIGNMENT NUMBER ijk TO COURSE CODE
\renewcommand{\headrulewidth}{1pt} %heading underlined
%\renewcommand{\baselinestretch}{1.2} % 1.2 line spacing for legibility (optional)

\begin{document}
\section*{1}
\paragraph{}
Let $G=(V,E)$ be an undirected graph. Let $k \in \Z_+^V$. Let $E'$ be a set of vertices each corresponding to an edge, $e\in E$. For ease of notation, for any $A \subseteq V$ let $E'(U) := \{e \in E' : e \text{ corresponds to } e \in E(U) \}$. Construct a capacitated digraph $D=(N,A)$, with capacities $\mu$, where $N = E' \cup V \cup \{r,s\}$ and $r$ and $s$ denote the source and sink nodes respectively. Construct the arc set $A$ as follows. For each $e \in E'$, $a = (r,e) \in A$ and $\mu_a = 1$. For each $e \in E'$ such that $e=uv \in E$, $a_1 = (e,u) \in A$ and $a_2 = (e,v) \in A$ such that $\mu_{a_1} = 1$ and $\mu_{a_2} = 1$. For each $v \in V$, $a = (v,s) \in A$ and $\mu_a = k_v$.
\paragraph{}
Suppose that there exists an orientation of $G$ such that $|\delta^-(v)|\leq k_v$ for all $v \in V$. From this orientation we will construct a feasible flow, $x$, on $D$. Let $a\in A$. If $a = (r,e)$ for some $e \in E'$ then set $x_a = 1$. If $a = (e,v)$ for some $e=uv \in E'$ and $v \in V$ then set $x_a = 1$ if $(u,v)$ is in the orientation of $G$, and set $x_a = 0$ otherwise. Now each $e \in E'$ has one unit flow incoming, and one unit of flow exiting as either $(u,v)$ is in the orientation of $G$ or $(v,u)$ is. The remaining case is if $a = (v,s)$ for some $v \in V$. In this case set $x_a = |\delta^-(v)|$. This is feasible since $|\delta^-(v)|\leq k_v$, and conservation is preserved because exactly $|\delta^-(v)|$ edges are oriented towards $v$, which happens to be the number of $(e,v)$ arcs given one unit of flow. Thus we have a feasible flow. The value of this flow is $f_x(s) = f_x(r) = \sum_{e\in E'} 1 = |E|$. Now by Max-Flow Min-Cut, $|E|$ is a lower bound on the capacity of any $rs$-cut.
\paragraph{}
Let $U \subseteq V$. Consider the $rs$-cut $R = \{r\} \cup E' \cup U$. Let $R_1 = \{a \in A : a = (v,s) $ for some $v \in U\}$ and let $R_2 = \{a \in A : a = (e,v)$ for some $e \in E'$ and $v \in V\backslash U$ such that $v \in e \}$. Then $\delta(R) = R_1 \cup R_2 $. Therefore
\begin{align*}
\mu(\delta(R)) &= \sum_{a\in R_1} \mu_a + \sum_{a\in R_2} \mu_a \\
&= \sum_{v \in U} k_v + \sum_{a \in R_2} \mu_a  &\text{as $k_v = \mu_{(v,s)}$}\\
&= \sum_{v \in U} k_v + \sum_{e \in E : e \not\in E(G[U])} 1 &\text{as $\mu_{(e,v)} = 1$ and $U\not\in \bar{R}$}\\ 
&= \sum_{v \in U} k_v + |E \backslash E(G[U])| \\
&= \sum_{v \in U} k_v + |E| - |E(U)|.
\end{align*}
Since $|E|$ is a lower bound on the capacity of any $rs$-cut, $|E| \leq \mu(\delta(R)) = \sum_{v \in U} k_v + |E| - |E(U)|$. Cancelling $|E|$ and rewriting we see that $|E(U)| \leq \sum_{v \in U} k_v$, as desired.
\paragraph{}
Now suppose that for all $U \subseteq V$, $|E(U)| \leq \sum_{u \in U} k_u$. Consider the digraph $D$ with capacities $\mu$ constructed as discussed previously. Let $x$ be an integral max flow on $D$. Let $R$ be a cut on $D$. Let $R_1 = \{ (r,e): e \in E' \cap \bar{R}\}$. Let $R_2 = \{(e,v): e \in E' \cap R \text{ and } v \in V \cap \bar{R}\}$. Let $R_3 = \{(v,s) : v \in V \cap R\}$. Then $\delta(R) = R_1 \cup R_2 \cup R_3$.
\paragraph{}
Then we have that:
\begin{align}
\mu(\delta(R)) &= \sum_{(r,e) \in R_1} 1 + \sum_{(e,v) \in R_2} 1 + \sum_{(v,s) \in R_3} k_v \nonumber\\
&= |E' \cap \bar{R}| + |R_2| + \sum_{(v,s) \in R_3} k_v \nonumber\\
&\geq |E' \cap \bar{R}| + |R_2| + |E'(V \cap R)| &\text{since $|E(U)| \leq \sum_{u \in U} k_u$, for all $U \subseteq V$} \nonumber\\
&= |E'| - |E'(V \cap \bar{R}) \cap R| - |\delta(V\cap R)| + |R_2|. 
\end{align}
With the last equality following by counting missed edges in $(E' \cap \bar{R}) \cup (E'(V \cap R))$. Let $\delta'(v) := \{e' \in E: e \text{ corresponds to } uv \in \delta(v)\}$. Now $$R_2 = \{(e,v): e \in E'(V \cap \bar{R}) \cap R \} \dot\cup \{(e,v): v \in V \cap \bar{R}, e\in \delta'(v) \cap R \}.$$ Thus $$|R_2| = |E'(V \cap \bar{R}) \cap R| + |\{(e,v): v \in V \cap \bar{R}, e\in \delta'(v) \cap R \}|.$$
So by $(1)$ we have
\begin{align*}
\mu(\delta(R)) &\geq |E'| - |E'(V \cap \bar{R}) \cap R| - |\delta(V\cap R)| + |R_2| \\
&= |E'| - |E'(V \cap \bar{R}) \cap R| - |\delta(V\cap R)| + |E'(V \cap \bar{R}) \cap R| + |\{(e,v): v \in V \cap \bar{R}, e\in \delta'(v) \cap R \}| \\
&\geq |E|.
\end{align*}
Therefore any cut in $D$ is of capacity at least $|E|$.
\paragraph{}
Now consider the cut $R = \{r\}$. Then $\mu(R) = |E|$, so $R$ is a min cut in $D$ since $|E|$ is a lower bound on the capacity of a cut. Thus by Max-Flow Min-Cut, $f_x(r) = f_x(s) = |E|$ since $x$ is a max flow and $R$ is a min cut. Then there is one unit of flow entering each $e \in E'$. Let $e \in E'$ such that $e$ corresponds to $uv \in E$. By conservation of flow there is one unit of flow leaving each $e \in E'$. Since $x$ is an integral max flow either $x_{(e,u)} = 1$ or $x_{(e,v)} = 1$. Construct an orientation of $G$, $D_G$, by orienting each $uv \in E$ such that $(u,v) \in A(D_G)$ if $x_{(e,v)} = 1$ and $(v,u) \in A(D_G)$ otherwise (that is, if $x_{(e,u)} = 1$).
\paragraph{}
Let $v \in V(G)$. By construction of $D_G$, $|\delta^-(v)| = |\{(e,v) : e \in E', x_{(e,v)} = 1\}|$. Since $|\{(e,v) : e \in E', x_{(e,v)} = 1\}|$ is the incoming flow to $v$, by conservation of flow $|\{(e,v) : e \in E', x_{(e,v)} = 1\}|$ is the outgoing flow from $v$. But there is only one outgoing arc from $v$. That arc is of capacity $k_v$ and so $|\{(e,v) : e \in E', x_{(e,v)} = 1\}| \leq k_v$ by feasibility of $x$. Therefore $|\delta^-(v)| \leq k_v$ for all $v \in V$ as desired. Thus there exists an orientation of $G$ such that $|\delta^-(v)| \leq k_v$ for all $v \in V$ if and only if $|E(U)| \leq \sum_{u \in U} k_u$, for all $U \subseteq V$.$\blacksquare$

\section*{2}
\paragraph{}
Let $D=(V,A)$ be a digraph, $u \in \R_+^A$ be arc capacities and $r,s \in V$ be the source and sink nods.
\subsection*{a}
\paragraph{}
Let $x \in \R_+^A$ be a feasible $r-s$ flow. Suppose that any $r-s$ dipath in the residual graph, $D_x$, has residual capacity at most $K > 0$. Let $u'$ denote the residual capacities. Let $x^* \in \R_+^A$ be a max flow on $D$. Let $\cP$ be the set of $r-s$ dipaths in $D_x$. For each $P \in \cP$ let $a(P)$ denote the unique edge of minimum residual capacity closest to $r$ along $P$, and let $u_P = u$ where $a(P) = (u,v)$. Let $R = \{v \in V: \exists P \in \cP$, $v \in V(rPu_p)\}$ where $xPy$ denotes the subpath of $P$ from $x$ to $y$ along $P$. Now $r \in R$ but $s \not\in R$ as $s$ is not the tail of any arc on an $r-s$ dipath in $D_x$. Thus $R$ is an $r-s$ cut in $D_x$.
\paragraph{}
Now $\delta(R) = \{a_P : P \in \cP \}$ by construction. Then $u'(\delta(R)) \leq K|\{a_P : P \in \cP \}| \leq Km$, since $u'_{a_P} \leq K$ for all $P \in \cP$. But also we have:
\begin{align*}
u'(\delta(R)) &= \sum_{(u,v) \in \delta(R)} (u_{uv} - x_{uv} + x_{vu}) &\text{by definition of residual capacity}\\
&= u(\delta(R)) - x(\delta(R)) + x(\delta(\bar{R})) \\
&= u(\delta(R)) - f_x(s) &\text{by Proposition $3.3$} \\
&\geq f_{x^*}(s) - f_x(s). &\text{by Max-Flow Min-Cut} \\
\end{align*}
Thus we have $u'(\delta(R)) \leq Km$ and $u'(\delta(R)) \geq f_{x^*}(s) - f_x(s)$ so by transitivity $Km \geq f_{x^*}(s) - f_x(s)$. Therefore $f_x(s) \geq f_{x^*}(s) - Km$ as desired. $\blacksquare$
\subsection*{b}
\paragraph{}
Let $u \in \Z_+^A$ and let $u_{max}=\max_{a\in A}u_a$. Consider the algorithm which results from modifying the in-class augmenting path algorithm as follows: In stage $i\geq 1$, augment flows using only augmenting paths with residual capacity at least $u_{max}/2^i$. We will show this leads to a polynomial time algorithm for solving the max flow problem.
\paragraph{Lemma 2.1}
For each stage stage of the algorithm there are at most $2m$ augmentations.
\paragraph{Proof of Lemma 2.1}
Suppose the algorithm is beginning stage $i$. Let $x$ denote the current flow found by the algorithm, and $x^*$ denote the optimal flow. Let $K$ be as in $(2a)$. Then by $(2a)$, $$f_{x*}(s) \leq f_x(s) + Km.$$ Since the algorithm just finished stage $i-1$, $$K \geq \frac{u_{max}}{2^{i-1}}.$$
Thus,
\begin{align}
f_{x^*}(s) &\leq f_x(s) + m\frac{u_{max}}{2^{i-1}} \nonumber\\
&= f_x(s) + 2m\frac{u_{max}}{2^i}. 
\end{align}
Now each augmentation in stage $i$ increases the value of $f_x(s)$ by at least $\frac{u_{max}}{2^i}$, so by $(2)$ there are at most $2m$ such augmentations (as otherwise $f_x(s)$ would exceed $f_{x^*}(s)$ which cannot happen by maximality). $\blacksquare$
\paragraph{}
After stage $i = log_2(u_{max})$, there will be no residual graph $r-s$ dipaths of residual capacity greater than or equal to:
$$
\frac{u_{max}}{2^i} = \frac{u_{max}}{u_{max}} = 1.
$$
Thus the residual capacity of every $r-s$ dipath in the residual graph is $0$, and so there are no flow augmenting paths in the graph. Therefore the algorithm has found a max flow after this stage by Theorem $3.6$. This algorithm terminates after $O(log_2(u_{max}))$ stages each consisting of $O(m)$ augmentations. Therefore the algorithm terminates in polynomial time as $O(mlog_2(u_{max}))$ is polynomial in the size of the input since $log_2(u_{max})$ is the size of the binary representation of $u_{max}$. $\blacksquare$

\section*{3}
\paragraph{}
Let $G=(V,E)$ be a network with source $r$ and sink $s$. Let $u \in (\R_+ \cup \{\infty\})^E$. Suppose there exists an arc $rv \in E$ such that $u_{rv} = \infty$. We may assume that $v \neq s$ as otherwise the maximum flow is unbounded. Let $x^*$ be a max flow on $G$. We claim that $f_{x^*}(s) = f_{x'}(s)$ where $x'$ is a finite max flow on $G'=G/rv$ with capacities $u' \in (\R_+ \cup \{\infty\})^{E\backslash\{rs\}}$.
\paragraph{}
To show this claim proceed by induction on the number of $\infty$ capacity arcs in $E$. Suppose there is only one $\infty$ capacity arc $rv$. Let $r'$ be the vertex of $G'$ resulting from contracting $rv$. This vertex is to be considered the source of $G'$. Since $v\neq s$, the sink $s$ remains the sink in $G'$. Since $G$ is connected and acyclic, $G'$ is connected and acyclic as the sets of graphs with these properties are closed under taking contractions. Since $G$ had only one $\infty$ capacity arc, $G'$ has no $\infty$ capacity edges. Therefore the min cut in $G'$ is finite valued, and thus by max-flow min-cut the max flow is finite valued. Let $x'$ be the max flow on $G'$. Let $R'$ be the min cut of $G'$. Then $u'(\delta(R')) = f_{x'}(s)$. Now $r' \in R'$ and $s \in \bar{R'}$. Let $R = R' \backslash\{r'\} \cup \{r,v\}$. Then $r \in R$ and $s \in \bar{R} = \bar{R'}$. So $R$ is an $rs$-cut in $G$ with $\delta(R) = \delta(R')$ and $u(\delta(R)) = u'(\delta(R')$. Since $R$ is an $rs$-cut, $f_{x^*}(s) \leq u(\delta(R))$ by max-flow min-cut. Suppose for a contradiction that $R$ is not a min cut of $G$. Let $R^*$ be a min cut of $G$. Then $u(\delta(R^*)) < u(\delta(R))$. Since $u(\delta(R))$ is finite, $u(\delta(R^*))$ is finite, and thus $rv \not\in \delta(R^*)$. But $r \in R^*$ (as $R^*$ is an $rs$-cut), so $v \in R^*$. Let $R'' = R^* \backslash \{r,v\} \cup \{r'\}$. Then $R''$ is an $r's$-cut in $G'$ and $\delta(R'') = \delta(R)$. But then:
\begin{align*}
u'(\delta(R'')) &= u(\delta(R^*)) \\
&< u(\delta(R)) \\
= u'(\delta(R')).
\end{align*}
This contradicts that $R'$ is a min cut in $G'$. Therefore $R$ is a min cut of $G$, and thus $f_{x*}(s) = u(\delta(R)) = u'(\delta(R')) = f_{x'}(s)$ as desired.
\paragraph{}
Now suppose for induction that for all graphs with at least one but less than $n$ arcs of $\infty$ capacity the claim holds. Let $n$ be the number of $\infty$ capacity $rv$ arcs of $G$. Let $rv$ be one such arc. Let $G'$, $r'$, $u'$ be as discussed previously. Let $x'$ be a max flow on $G'$. Now $G'$ has less than $n$ $\infty$ capacity arcs since $rv$ was contracted. Let $r'v' \in E(G')$ such that $u_{r'v'} = \infty$. Let $x''$ be a max flow on $G'/r'v'$. By the induction hypothesis $f_{x'}(s) = f_{x''}(s)$ which is finite. Therefore the min cut in $G'$ is finite valued, and thus by max-flow min-cut the max flow is finite valued. Let $R'$ be the min cut of $G'$. Then $u'(\delta(R')) = f_{x'}(s)$. Now $r' \in R'$ and $s \in \bar{R'}$. Let $R = R' \backslash\{r'\} \cup \{r,v\}$. Then $r \in R$ and $s \in \bar{R} = \bar{R'}$. So $R$ is an $rs$-cut in $G$ with $\delta(R) = \delta(R')$ and $u(\delta(R)) = u'(\delta(R')$. Since $R$ is an $rs$-cut, $f_{x^*}(s) \leq u(\delta(R))$ by max-flow min-cut. Suppose for a contradiction that $R$ is not a min cut of $G$. Let $R^*$ be a min cut of $G$. Then $u(\delta(R^*)) < u(\delta(R))$. Since $u(\delta(R))$ is finite, $u(\delta(R^*))$ is finite, and thus $rv \not\in \delta(R^*)$. But $r \in R^*$ (as $R^*$ is an $rs$-cut), so $v \in R^*$. Let $R'' = R^* \backslash \{r,v\} \cup \{r'\}$. Then $R''$ is an $r's$-cut in $G'$ and $\delta(R'') = \delta(R)$. But then:
\begin{align*}
u'(\delta(R'')) &= u(\delta(R^*)) \\
&< u(\delta(R)) \\
= u'(\delta(R')).
\end{align*}
This contradicts that $R'$ is a min cut in $G'$. Therefore $R$ is a min cut of $G$, and thus $f_{x*}(s) = u(\delta(R)) = u'(\delta(R')) = f_{x'}(s)$ as desired. So by the principle of mathematical induction the claim holds.
\paragraph{}
This claim gives rise to an algorithm for constructing a maximum flow when $\infty$ capacity arcs are present. Given $G$, $u$, $r$, $v$ the algorithm does the following:
\begin{enumerate}
\item If $G$ has $\infty$ capacity arc from $r$ to $s$ stop, problem is unbounded.\\
\item If $G$ has no $\infty$ capacity arcs return max flow using a standard algorithm\\
\item Else let $rv$ be an $\infty$ capacity arc.\\
\item Call the algorithm recursively on $G' = G/rv$.\\
\item Let $x'$ be the flow returned. Construct the max flow on $G$, $x^*$ as follows:\\
\subitem For all outgoing arcs from $v$, $vu$, let $x^*{vu} = x'_{r'u}$.\\
\subitem Set $x^*_{rv} = \sum_{a \in \delta^+(v)} x^*_{a}$.\\
\subitem For all arcs $r'u$ such that $u$ is not head of an outgoing arc from $v$, let $x^*_{ru} = x'_{r'u}$,\\
\subitem For all remainig arcs $uw$ set $x^*_{uw} = x'_{uw}$.
\end{enumerate}
\paragraph{}
This algorithm constructs a flow on $G$ of the same value as a flow on the contracted graph $G'$, so by our claim, provided that the constructed flow is feasible, the constructed flow is maximal. Indeed the only vertex in the graph $G$ for which it is not clear that the flow passing through in $x^*$ and $x'$ is the same is $v$, but in the second step of constructing the flow it is set to be such, and hence since $x'$ is feasible so is $x^*$. Therefore this algorithm constructs a maximum flow for $G$. This algorithm runs in polynomial time because calling a standard max flow returns in polynomial time, contracting an edge is polynomial time, and building the new flow can be done in polynomial time by scanning the edge sets of $G$ and $G'$. $\blacksquare$

\section*{4}
\paragraph{}
Modify the push-relabel algorithm as follows: modify the definition of "active" nodes to include the condition that a node is only active if $h(v) < n$. Throughout this writing I will refer to this property as modified-active, and the active will refer to active in the original sense. For any feasible pre-flow $x$, define $S_x = \{ v \in V: \text{ there does not exist } (v,s) \text{ dipath in } D'_x \}$. That is, $\bar{S}_x$ is the set of nodes that can reach $s$ by a dipath in $D'_x$.The modified push-relabel algorithm terminates when there are no remaining modified-active nodes and returns a valid preflow $x$, labelling $h$, and minimum which is $S_x$. Otherwise the operations are the same as the original algorithm.
\paragraph{Complexity}
 In the original algorithm the count on total pushes was at most $4n^2m + 2nm$. This bound was based on the fact that there are at most $2n-1$ relabels that the algorithm will perform per vertex. Notice that the modified algorithm is now artificially bounded in the number of relabels it will do to at most $n-1$. Therefore the count on total pushes will decrease by a factor of two, that is to say there will be at most $2n^m + nm$ pushes total in modified push-relabel.
\paragraph{Lemma 4.1}
If $h$ is a valid labelling and $x$ is a valid pre-flow then $r \in S_x$.
\paragraph{Proof of Lemma 4.1}
Suppose $h$ is a valid labelling, $x$ is a valid pre-flow, and $r \in \bar{S}_x$. Then $d_x(r,s)$ is finite. So $d_x(r,s) \leq n-1$. But by Lemma $3.24$ in the textbook, $d_x(r,s) \leq h(r) - h(s) = n - 0 = n$. Thus we have that $n \leq n-1$, a contradiction. $\blacksquare$
 \paragraph{Lemma 4.2}
 If $h$ is a valid labelling and $x$ is a valid pre-flow then for all $v \in \bar{S}_x$, $h(v) \leq n-2$.
 \paragraph{Proof of Lemma 4.2}
 Suppose $h$ is a valid labelling and $x$ is a valid pre-flow. Let $v \in \bar{S}_x$. Then there exists a $(v,s)$ dipath in $D'_x$. So $d_x(v,s)$ is finite. Since $r \not\in \bar{S}_x$ (by Lemma $4.1$) this shortest path does not use $r$ (as otherwise $r \in \bar{S}_x$). Therefore $d_x(v,s) \leq n-2$. By Lemma $3.24$, $d_x(v,s) \leq h(v) + h(s) = h(v)$. Therefore we have that $h(v) \leq n-2$. $\blacksquare$
 \paragraph{Lemma 4.3}
 If $h$ is a valid labelling and $x$ is a valid pre-flow both of which were returned by the modified push-relabel algorithm then for all $v \in \bar{S_x}$, $v$ is inactive.
 \paragraph{Proof of Lemma 4.3}
 Suppose that the modified push-relabel algorithm returns labelling $h$ and pre-flow $x$. Let $v \in \bar{S_x}$. Since $v$ is modified-inactive, either $h(v) \geq n$ or $v$ is inactive. But by Lemma $4.2$, $h(v) \leq n-2$, that is $h(v) \not\geq n$. Therefore $v$ is inactive. $\blacksquare$  
 \paragraph{Lemma 4.4}
 If $h$ is a valid labelling and $x$ is a valid pre-flow then for all $v \in S_x$ and $w \in \bar{S}_x$, $(v,w) \not\in D'_x$.
 \paragraph{Proof of Lemma 4.4}
Suppose $h$ is a valid labelling and $x$ is a valid pre-flow. Let $v \in S_x$ and let $w \in \bar{S}_x$. Suppose for a contradiction that $(v,w) \in D'_x$. Since $w \in \bar{S}_x$ there exists a $(w,s)$ dipath in $D'_x$. Joining the arc $(v,w)$ to this dipath forms a $(v,s)$ dipath in $D'_x$. This contradicts that $v \in S_x$. $\blacksquare$
\paragraph{Observation}
Since the $h$ and $x$ returned by the modified push-relabel algorithm are such that $h$ is a valid labelling, and $x$ is a feasible pre-flow, we may consider what happens when we begin running the original push-relabel algorithm on the original input graph from the state where the labelling is $h$ and the pre-flow is $x$.
\paragraph{Theorem 4}
If $h$ and $x$ are the labelling and pre-flow (respectively) returned by the modified push-relabel algorithm for some input graph $G$, then after every iteration, $i$, of the original push-relabel algorithm considered to begin running on input $G$ with labelling $h$ and pre-flow $x$ we have that the flow after iteration $i$, $x_i$, is such that $S_{x_i} = S_x$.
\paragraph{Proof of Theorem 4}
Proceed by induction on $i$. If $i = 0$ then $x_i = x_0 = x$ since the algorithm has yet to begin. Clearly $S_{x_i} = S_x$. For induction suppose that Theorem $4$ holds for all iterations of the algorithm prior to iteration $i$. That is, at the beginning of iteration $i$ we have flow $x_{i-1}$ such that $S_{x_{i-1}} = S_x$. Let $v \in \bar{S}_{x_{i-1}}$. Then $v \in \bar{S}_x$. Let $w \in \delta^+(v)$ (in $D'_{x_{i-1}}$). Since $v$ is inactive (by Lemma 4.3), $(v,w)$ is never pushed on during remaining operations as $v$ is inactive. Let $w \in \delta^-(v)$ (in $D'_{x_{i-1}}$). If $w \in \bar{S}_{x_{i-1}} = \bar{S}_x$ then $w$ is inactive and $(w,v)$ is never pushed on. If $w \in S_{x_{i-1}} = S_x$ then $(w,v) \not\in D'_{x_{i-1}}$ (by Lemma $4.4$) and hence is never pushed on. Since this hold for all $v \in \bar{S}_{x_{i-1}}$ no arcs on any $(v,s)$ dipath are pushed on for $v \in \bar{S}_{x_{i-1}}$. Therefore no $v \in \bar{S}_{x_{i-1}}$ leaves the set after iteration $i$. That is $\bar{S}_{x_{i-1}} \subseteq \bar{S}_{x_i}$, and hence $S_x \subseteq S_{x_i}$.
\paragraph{}
Now let $v \in \bar{S}_{x_i}$. Suppose that $v \not\in \bar{S}_{x_{i-1}}$. Then $v \not \in \bar{S}_x$ by the induction hypothesis. Consider the $(v,s)$ dipath in $D'_{x_i}$, $P$. Since $P$ is not in $D'_{x_{i-1}}$, there exists an arc $(v',w') \in E(P)$ that joins a $(v,v')$ dipath in $D'_{x_{i}}$ to a $(w',s)$ dipath in $D'_{x_{i-1}}$. That is arc $(v',w')$ is an arc which enters the residual graph during iteration $i$. Then $w' \in \bar{S}_{x_{i-1}}$ since there is a $(w',s)$ dipath in $D'_{x_{i-1}}$. Further $v' \in S_{x_{i-1}}$, as otherwise the $(v,v')$ dipath in $D'_{x_{i-1}}$ could be joined to the $(w',s)$ dipath in $D'_{x_{i-1}}$ to form a $(v,s)$ dipath in $D'_{x_{i-1}}$ implying that $v \in \bar{S}_{x_{i-1}}$ (contradicting $v \in S_{x_{i-1}}$). Notice that arcs only enter $D'_{x_i}$ from pushes on their endpoints during iteration $i$. But since $w' \in S_{x_{i-1}}$, by induction $w' \in S_{x}$ and so $w'$ is inactive during iteration $i$ by Lemma $4.3$. So the push happened from $v'$ to $w'$, implying that $(v',w') \in D'_{x_{i-1}}$. But this is a contradiction. Therefore $\bar{S}_{x_i} \subseteq \bar{S}_{x_{i-1}}$, and hence $S_{x_i} \subseteq S_x$. That is, $S_{x_i} = S_x$ as desired. Therefore by the principle of mathematical induction Theorem $4$ holds. $\blacksquare$
\paragraph{Correctness}
Let $k$ be the final iteration of the original algorithm after running from the state returned by the modified algorithm. Let $x^* = x_k$. By Theorem $4$, $S_{x^*} = S_x$. Notice that $x^*$ is a maximum flow, and thus is feasible. Now $S_x$ is an $r-s$ cut since $r \in S_x$ by Lemma $4.1$ and $s \in \bar{S}_x$ by definition. Let $v \in S_x$ and let $w \in \bar{S}_x$. Then $v \in S_{x^*}$ and $w \in \bar{S}_{x^*}$. Now $(v,w)$ is not an arc in $D'_{x^*}$, as otherwise the arc $(v,w)$ joined to the $(w,s)$ dipath in $D'_{x^*}$ is a $(v,s)$ dipath in $D'_{x^*}$ contradicting that $v \in S_{x^*}$. Since $(v,w)$ is not an arc in $D'_{x^*}$, $x^*_{vw} = u_{vw}$ and $x^*_{wv} = 0$. That is to say, since $x^*$ is feasible and $S_{x^*} = S_x$ is an $r-s$ cut, by Corollary $3.8$, $S_{x^*}$ is a minimum $r-s$ cut. Since $S_x = S_{x^*}$, $S_x$ is a min cut. Therefore the modified push-relabel algorithm returns a minimum cut as desired. $\blacksquare$

\section*{5}
\paragraph{}
Let $G$ be an undirected graph with capacities $u \in R_+^E$. Let $p,q,r \in V$. We may assume without loss of generality, up to relabelling, that $\lambda(G,p,q) \leq \lambda(G,p,r) \leq \lambda(G,q,r)$. So we have that 
\begin{align*}
\lambda(G,p,r) &\geq min(\lambda(G,p,q), \lambda(G,q,r)) &\text{by Lemma $3.37$ of the textbook}\\
&= \lambda(G,p,q). &\text{ since $\lambda(G,p,q) \leq \lambda(G,q,r)$}
\end{align*} 
Thus $\lambda(G,p,r) \geq \lambda(G,p,q)$. Similarly we have that
\begin{align*}
\lambda(G,p,q) &\geq min(\lambda(G,p,r), \lambda(G,q,r)) &\text{by Lemma $3.37$}\\
&= \lambda(G,p,r). &\text{since $\lambda(G,p,r) \leq \lambda(G,q,r)$}
\end{align*}
Thus $\lambda(G,p,q) \geq \lambda(G,p,r)$. Combining inequalities we obtain that $\lambda(G,p,q) = \lambda(G,p,r)$ (the smallest two of $\lambda(G,p,q),\lambda(G,p,r),\lambda(G,q,r)$ are equal) as desired. $\blacksquare$

\section*{6}
\paragraph{Definition}
Let $G$ be an undirected graph. Let $T_G$ be a spanning tree for $G$.  For any edge $e \in E(T_G)$ let $C(e)$ (the cut in $G$ induced by $e$) denote the edges in $G$ between the two connected components of $T-e$. Notice in particular that Gomory-Hu trees span $G$ and hence $C(e)$ is defined for them.
\paragraph{Lemma 6.1}
Let $G=(V,E)$ be a graph. Let $T \subseteq V$, $|T|$ is even. Let $\delta(S)$ be a $T$-cut. Let $J$ be a $T$-join. Then $|J \cap \delta(S)|$ is odd. (Note: This is mentioned in the textbook without a complete proof)
\paragraph{Proof of Lemma 6.1}
Let $G'$ be a graph such that $G' = (S, \gamma(S) \cap J)$. Recall that $\gamma(S) = \{ uv \in E(G) : u, v \in S\}$. Let $d'$ be the degree function on vertices of $G'$. Let $D' = \{ v \in S : d'(v) \equiv 1 (\text{mod }2)\}$. Since $G'$ is a graph, $|D'|$ is even (this is a common graph theory fact, that the number of odd degree vertices in a graph is even). Now let $G''$ be a graph such that $G'' = (V, J)$. Let $d''$ be the degree function on vertices of $G''$. Since $\delta(S)$ is a $T$-cut, $|S \cap T| \equiv 1$ (mod $2$). Since $J$ is a $T$-join, $d''(v) \equiv 1$ (mod $2$) if and only if $v \in T$. Let $D'' = \{v \in S : d''(v) \equiv 1 (\text{mod }2)\}$. Then we have that $|D''|$ is odd. By a counting argument one can see that $|J \cap \delta(S)| = \sum_{v \in S} d''(v) - \sum_{v \in S} d'(v)$. One can see this by realizing that the first sum counts edges with at least one endpoint in $S$, while the second sum counts edges with two endpoints in $S$. The edges remaining uncounted are exactly the edges on the $S$-boundary, $\delta(S)$. Thus we have that
\begin{align*}
|J \cap \delta(S) &\equiv \sum_{v \in S} d''(v) - \sum_{v \in S} d'(v) &(\text{mod }2)\\
&\equiv |D''| - |D'| &(\text{mod }2)\\
&\equiv 1 - 0 &(\text{mod }2)\\
&\equiv 1. &(\text{mod }2)
\end{align*}
Therefore we have that $|J \cap \delta(S)|$ is odd. $\blacksquare$
\paragraph{Lemma 6.2}
Let $G=(V,E)$ be a graph. Let $T \subseteq V$, $|T|$ is even. Let $\delta(S)$ be a $T$-cut. Let $J$ be a $T$-join. Then $J \cap \delta(S) \neq \emptyset$.
\paragraph{Proof of Lemma 6.2}
Since $|J \cap \delta(S)| \geq 0$, and by Lemma $6.1$ we have that $|J \cap \delta(S)|$ is odd so $|J \cap \delta(S)| \geq 1$ since $0$ is even. Therefore $J \cap \delta(S) \neq \emptyset$. $\blacksquare$
\paragraph{Lemma 6.3}
Let $G$ be an undirected graph. Let $T_G$ be a spanning tree for $G$. Let $T \subseteq V(G)$ such that $|T|$ is even. Let $J = \{e \in E(T_G) : C(e) \text{ is a } T-\text{cut in } G\}$. Then $J$ is a $T$-join in $T_G$.
\paragraph{Proof of Lemma 6.3}
We proceed by induction on $|V(T_G)|$. If $|V(T_G)| = 0$ or $|V(T_G)| = 1$ then $|T| = 0$ and the Lemma holds trivially. If $|V(T_G)| = 2$ then if $|T| = 0$ the Lemma holds trivially. If $|T|=2$ then let $u,v \in V(T_G)$ with $u\neq v$. Then $u,v\in T$ since $|T| = 2$. Since $T_G$ is connected, $e=uv \in E(T_G)$. Since $C(e)$ separates $u$ and $v$, the components of $T-e$ each contain one element of $T$ (in particular an odd number). Thus $e \in J$. Therefore $|J \cap \{u\}| = 1$ and $|J \cap \{v\}| = 1$ so $J$ is a $T$-join.
\paragraph{}
Let $n = |V(T_G)|$ and suppose for induction that for every undirected graph with a Gomory-Hu tree of size less than $n$ that Lemma $6.3$ holds. Since every tree contains two distinct leaves let $u,v$ be two distinct  leaves of $T_G$. If $u \not\in T$ or $v \not\in T$ then say without loss of generality that $v \not\in T$ (up to replacing $v$ with $u$ in following). In this case $T$ is an even subset of $V(T_G-v)$ and $J = \{e \in E(T_G-v) : C(e) \text{ is a } T-\text{cut in } G\}$ since the edge containing $v$ is not in $J$ (as $|T \cap \{v\}| = 0$ and hence the cut separating $v$ from the rest of $G$ is not a $T$-cut). So by the induction hypothesis, $J$ is a $T$-join of $T_G - v$, and thus $J$ is a $T$-join of $T_G$ as $v \not\in T$.
\paragraph{}
It remains to consider the case where $u, v \in T$. Let $T' = T \backslash\{u,v\}$. Since $|T|$ is even and $|T'| = |T| - 2$, $|T'|$ is even. Let $J' = \{e \in E(T_G) : C(e) \text{ is a } T'-\text{cut in } G\}$. By the induction hypothesis, $J'$ is a $T'$-join. Let $T'' = \{u,v\}$. Then $|T''| = 2$ is even. Let $J''=\{e \in E(T_G) : C(e) \text{ is a } T''-\text{cut in } G\}$. Let $P$ be the unique path from $u$ to $v$ in $T_G$. Then $E(P) = J''$, as the edges of $P$ are exactly the edges whose corresponding $C(e)$ cuts separate $u$ from $v$. Thus $|J'' \cap \{u\}| = |J''\cap \{v\}| = 1$ as $u$ and $v$ are leaves. Therefore $J''$ is $T''$-join.
\paragraph{}
By Proposition $???$ from the textbook, $J' \Delta J''$ is $T' \Delta T''$-join. But $T' \cap T'' = \emptyset$ so $T' \Delta T'' = T' \cup T'' = T$. That is, $J' \Delta J''$ is a $T$-join. But $J' \cup J'' = J$ so $J' \Delta J'' = J \backslash (J' \cap J'')$. Let $e \in J' \cap J''$. Then $e \in E(P)$ and $e \in J''$. Let $C$ be the connected component of $T_G - e$ containing $u$. Then $|V(C) \cap T''|$ is odd, but $u \in V(C)$ and $u \not\in T''$ so $|V(C) \cap (T'' \cup\{u\})|$ is even. Since $v \not\in V(C)$ and $T'' \cup\{u\} \cup \{v\} = T$ we have $|V(C) \cap T|$ is even. That is $e \not \in J$. Therefore $J \backslash (J' \cap J'') = J$, and hence $J$ is a $T$-join as desired. So by the principle of mathematical induction, Lemma $6.3$ holds. $\blacksquare$
\paragraph{Theorem 6}
If $G$ is an undirected graph with capacities $u \in \R_+^E$, $T\subseteq V$ such that $|T|$ is even, and $T_G$ is a Gomory-Hu tree for $G$ then there exists $e\in E(T_G)$ such that $C(e)$ is a $T$-cut in $G$ and $u(C(e))$ is minimal over all $T$-cuts in $G$.
\paragraph{Proof of Theorem 6}
Suppose that $G$ be an undirected graph with capacities $u \in \R_+^E$, $T\subseteq V$ such that $|T|$ is even, and $T_G$ is a Gomory-Hu tree for $G$. We may assume that $G$ is a complete graph, as otherwise we may add $0$-capacity edges to $G$ between non-adjacent vertices in order to complete $G$ without affecting the capacity of any cut in $G$. Notice that since $G$ is a complete graph, $T_G$ is a subgraph of $G$. Let $\delta(S)$ be a minimum $T$-cut in $G$. Let $J  = \{e \in E(T_G) : C(e) \text{ is a } T-\text{cut in } G\}$. By Lemma $6.3$, $J$ is a $T$-join of $T_G$ (Gomory-Hu trees span $G$). Since $T_G$ is a subgraph of $G$ we may invoke Lemma $6.2$ obtaining that $J \cap \delta(S) \neq \emptyset$. Let $e=uv \in J \cap \delta(S)$. Then, since $e \in J$, $C(e)$ is a $T$-cut in $G$. Since $e \in \delta(S)$, $\delta(S)$ separates $u$ and $v$. Therefore, by minimality of $\lambda(G, u,v) = f_e$ we have that
\begin{align*}
u(\delta(S)) &\geq \lambda(G,u,v) \\
&= f_e \\
&= u(C(e)).
\end{align*}
Thus, since $\delta(S)$ is a minimum $T$-cut in $G$, we have that $C(e)$ is a minimum $T$-cut in $G$. $\blacksquare$
\paragraph{}
Theorem $6$ gives rise to an algorithm for finding the minimum capacity $T$-cut. Let $n = |V(G)|$. To find the minimum capacity $T$-cut, simply compute the Gomory-Hu tree for $G$ (this can be done in $O(n^4)$ time), and return the minimum $T$-cut among cuts induced by edges in the Gomory-Hu tree. By Theorem $6$ this algorithm will in fact return a correct minimum capacity $T$-cut. For any edge in the Gomory-Hu tree, the cut induced by that edge can be verified to be a $T$-cut in $O(n^2)$ time by scanning the vertex set of the partitions of the cut and counting the number of vertices in $T$ (verifying that the count is odd). There are $O(n)$ edges to scan in the tree, so the entire process of finding the minimum $T$-cut once the tree is computed can be done in $O(n^3)$ time. Therefore the entire algorithm returns a minimum capacity $T$-cut in $G$ in $O(n^4)$ time. $\blacksquare$
\end{document}
