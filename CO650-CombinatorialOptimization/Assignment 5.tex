\documentclass[letterpaper,12pt,oneside,onecolumn]{article}
\usepackage[margin=1in, bottom=1in, top=1in]{geometry} %1 inch margins
\usepackage{amsmath, amssymb, amstext}
\usepackage{fancyhdr}
\usepackage{algorithm}
\usepackage{algpseudocode}
\usepackage{mathtools}

\DeclarePairedDelimiter{\ceil}{\lceil}{\rceil}

%Macros
\newcommand{\A}{\mathbb{A}} \newcommand{\C}{\mathbb{C}}
\newcommand{\D}{\mathbb{D}} \newcommand{\F}{\mathbb{F}}
\newcommand{\N}{\mathbb{N}} \newcommand{\R}{\mathbb{R}}
\newcommand{\T}{\mathbb{T}} \newcommand{\Z}{\mathbb{Z}}
\newcommand{\Q}{\mathbb{Q}}
 
 
\newcommand{\cA}{\mathcal{A}} \newcommand{\cB}{\mathcal{B}}
\newcommand{\cC}{\mathcal{C}} \newcommand{\cD}{\mathcal{D}}
\newcommand{\cE}{\mathcal{E}} \newcommand{\cF}{\mathcal{F}}
\newcommand{\cG}{\mathcal{G}} \newcommand{\cH}{\mathcal{H}}
\newcommand{\cI}{\mathcal{I}} \newcommand{\cJ}{\mathcal{J}}
\newcommand{\cK}{\mathcal{K}} \newcommand{\cL}{\mathcal{L}}
\newcommand{\cM}{\mathcal{M}} \newcommand{\cN}{\mathcal{N}}
\newcommand{\cO}{\mathcal{O}} \newcommand{\cP}{\mathcal{P}}
\newcommand{\cQ}{\mathcal{Q}} \newcommand{\cR}{\mathcal{R}}
\newcommand{\cS}{\mathcal{S}} \newcommand{\cT}{\mathcal{T}}
\newcommand{\cU}{\mathcal{U}} \newcommand{\cV}{\mathcal{V}}
\newcommand{\cW}{\mathcal{W}} \newcommand{\cX}{\mathcal{X}}
\newcommand{\cY}{\mathcal{Y}} \newcommand{\cZ}{\mathcal{Z}}

%Page style
\pagestyle{fancy} 

\listfiles

\raggedbottom

\rhead{William Justin Toth 650 A5} %CHANGE n to ASSIGNMENT NUMBER ijk TO COURSE CODE
\renewcommand{\headrulewidth}{1pt} %heading underlined
%\renewcommand{\baselinestretch}{1.2} % 1.2 line spacing for legibility (optional)

\begin{document}
\section*{1}
\paragraph{}
Let $G=(V,E)$ be an undirected graph. Let $k \in \Z_+^V$. Let $E'$ be a set of vertices each corresponding to an edge, $e\in E$. For ease of notation, for any $A \subseteq V$ let $E'(U) := \{e \in E' : e \text{ corresponds to } e \in E(U) \}$. Construct a capacitated digraph $D=(N,A)$, with capacities $\mu$, where $N = E' \cup V \cup \{r,s\}$ and $r$ and $s$ denote the source and sink nodes respectively. Construct the arc set $A$ as follows. For each $e \in E'$, $a = (r,e) \in A$ and $\mu_a = 1$. For each $e \in E'$ such that $e=uv \in E$, $a_1 = (e,u) \in A$ and $a_2 = (e,v) \in A$ such that $\mu_{a_1} = 1$ and $\mu_{a_2} = 1$. For each $v \in V$, $a = (v,s) \in A$ and $\mu_a = k_v$.
\paragraph{}
Suppose that there exists an orientation of $G$ such that $|\delta^-(v)|\leq k_v$ for all $v \in V$. From this orientation we will construct a feasible flow, $x$, on $D$. Let $a\in A$. If $a = (r,e)$ for some $e \in E'$ then set $x_a = 1$. If $a = (e,v)$ for some $e=uv \in E'$ and $v \in V$ then set $x_a = 1$ if $(u,v)$ is in the orientation of $G$, and set $x_a = 0$ otherwise. Now each $e \in E'$ has one unit flow incoming, and one unit of flow exiting as either $(u,v)$ is in the orientation of $G$ or $(v,u)$ is. The remaining case is if $a = (v,s)$ for some $v \in V$. In this case set $x_a = |\delta^-(v)|$. This is feasible since $|\delta^-(v)|\leq k_v$, and conservation is preserved because exactly $|\delta^-(v)|$ edges are oriented towards $v$, which happens to be the number of $(e,v)$ arcs given one unit of flow. Thus we have a feasible flow. The value of this flow is $f_x(s) = f_x(r) = \sum_{e\in E'} 1 = |E|$. Now by Max-Flow Min-Cut, $|E|$ is a lower bound on the capacity of any $rs$-cut.
\paragraph{}
Let $U \subseteq V$. Consider the $rs$-cut $R = \{r\} \cup E' \cup U$. Let $R_1 = \{a \in A : a = (v,s) $ for some $v \in U\}$ and let $R_2 = \{a \in A : a = (e,v)$ for some $e \in E'$ and $v \in V\backslash U$ such that $v \in e \}$. Then $\delta(R) = R_1 \cup R_2 $. Therefore
\begin{align*}
\mu(\delta(R)) &= \sum_{a\in R_1} \mu_a + \sum_{a\in R_2} \mu_a \\
&= \sum_{v \in U} k_v + \sum_{a \in R_2} \mu_a  &\text{as $k_v = \mu_{(v,s)}$}\\
&= \sum_{v \in U} k_v + \sum_{e \in E : e \not\in E(G[U])} 1 &\text{as $\mu_{(e,v)} = 1$ and $U\not\in \bar{R}$}\\ 
&= \sum_{v \in U} k_v + |E \backslash E(G[U])| \\
&= \sum_{v \in U} k_v + |E| - |E(U)|.
\end{align*}
Since $|E|$ is a lower bound on the capacity of any $rs$-cut, $|E| \leq \mu(\delta(R)) = \sum_{v \in U} k_v + |E| - |E(U)|$. Cancelling $|E|$ and rewriting we see that $|E(U)| \leq \sum_{v \in U} k_v$, as desired.
\paragraph{}
Now suppose that for all $U \subseteq V$, $|E(U)| \leq \sum_{u \in U} k_u$. Consider the digraph $D$ with capacities $\mu$ constructed as discussed previously. Let $x$ be an integral max flow on $D$. Let $R$ be a cut on $D$. Let $R_1 = \{ (r,e): e \in E' \cap \bar{R}\}$. Let $R_2 = \{(e,v): e \in E' \cap R \text{ and } v \in V \cap \bar{R}\}$. Let $R_3 = \{(v,s) : v \in V \cap R\}$. Then $\delta(R) = R_1 \cup R_2 \cup R_3$.
\paragraph{}
Then we have that:
\begin{align}
\mu(\delta(R)) &= \sum_{(r,e) \in R_1} 1 + \sum_{(e,v) \in R_2} 1 + \sum_{(v,s) \in R_3} k_v \nonumber\\
&= |E' \cap \bar{R}| + |R_2| + \sum_{(v,s) \in R_3} k_v \nonumber\\
&\geq |E' \cap \bar{R}| + |R_2| + |E'(V \cap R)| &\text{since $|E(U)| \leq \sum_{u \in U} k_u$, for all $U \subseteq V$} \nonumber\\
&= |E'| - |E'(V \cap \bar{R}) \cap R| - |\delta(V\cap R)| + |R_2|. 
\end{align}
With the last equality following by counting missed edges in $(E' \cap \bar{R}) \cup (E'(V \cap R))$. Let $\delta'(v) := \{e' \in E: e \text{ corresponds to } uv \in \delta(v)\}$. Now $$R_2 = \{(e,v): e \in E'(V \cap \bar{R}) \cap R \} \dot\cup \{(e,v): v \in V \cap \bar{R}, e\in \delta'(v) \cap R \}.$$ Thus $$|R_2| = |E'(V \cap \bar{R}) \cap R| + |\{(e,v): v \in V \cap \bar{R}, e\in \delta'(v) \cap R \}|.$$
So by $(1)$ we have
\begin{align*}
\mu(\delta(R)) &\geq |E'| - |E'(V \cap \bar{R}) \cap R| - |\delta(V\cap R)| + |R_2| \\
&= |E'| - |E'(V \cap \bar{R}) \cap R| - |\delta(V\cap R)| + |E'(V \cap \bar{R}) \cap R| + |\{(e,v): v \in V \cap \bar{R}, e\in \delta'(v) \cap R \}| \\
&\geq |E|.
\end{align*}
Therefore any cut in $D$ is of capacity at least $|E|$.
\paragraph{}
Now consider the cut $R = \{r\}$. Then $\mu(R) = |E|$, so $R$ is a min cut in $D$ since $|E|$ is a lower bound on the capacity of a cut. Thus by Max-Flow Min-Cut, $f_x(r) = f_x(s) = |E|$ since $x$ is a max flow and $R$ is a min cut. Then there is one unit of flow entering each $e \in E'$. Let $e \in E'$ such that $e$ corresponds to $uv \in E$. By conservation of flow there is one unit of flow leaving each $e \in E'$. Since $x$ is an integral max flow either $x_{(e,u)} = 1$ or $x_{(e,v)} = 1$. Construct an orientation of $G$, $D_G$, by orienting each $uv \in E$ such that $(u,v) \in A(D_G)$ if $x_{(e,v)} = 1$ and $(v,u) \in A(D_G)$ otherwise (that is, if $x_{(e,u)} = 1$).
\paragraph{}
Let $v \in V(G)$. By construction of $D_G$, $|\delta^-(v)| = |\{(e,v) : e \in E', x_{(e,v)} = 1\}|$. Since $|\{(e,v) : e \in E', x_{(e,v)} = 1\}|$ is the incoming flow to $v$, by conservation of flow $|\{(e,v) : e \in E', x_{(e,v)} = 1\}|$ is the outgoing flow from $v$. But there is only one outgoing arc from $v$. That arc is of capacity $k_v$ and so $|\{(e,v) : e \in E', x_{(e,v)} = 1\}| \leq k_v$ by feasibility of $x$. Therefore $|\delta^-(v)| \leq k_v$ for all $v \in V$ as desired. Thus there exists an orientation of $G$ such that $|\delta^-(v)| \leq k_v$ for all $v \in V$ if and only if $|E(U)| \leq \sum_{u \in U} k_u$, for all $U \subseteq V$.$\blacksquare$

\section*{2}
\paragraph{}
Let $D=(V,A)$ be a digraph, $u \in \R_+^A$ be arc capacities and $r,s \in V$ be the source and sink nods.
\subsection*{a}
\paragraph{}
Let $x \in \R_+^A$ be a feasible $r-s$ flow. Suppose that any $r-s$ dipath in the residual graph, $D_x$, has residual capacity at most $K > 0$. Let $u'$ denote the residual capacities. Let $x^* \in \R_+^A$ be a max flow on $D$. Let $\cP$ be the set of $r-s$ dipaths in $D_x$. For each $P \in \cP$ let $a(P)$ denote the unique edge of minimum residual capacity closest to $r$ along $P$, and let $u_P = u$ where $a(P) = (u,v)$. Let $R = \{v \in V: \exists P \in \cP$, $v \in V(rPu_p)\}$ where $xPy$ denotes the subpath of $P$ from $x$ to $y$ along $P$. Now $r \in R$ but $s \not\in R$ as $s$ is not the tail of any arc on an $r-s$ dipath in $D_x$. Thus $R$ is an $r-s$ cut in $D_x$.
\paragraph{}
Now $\delta(R) = \{a_P : P \in \cP \}$ by construction. Then $u'(\delta(R)) \leq K|\{a_P : P \in \cP \}| \leq Km$, since $u'_{a_P} \leq K$ for all $P \in \cP$. But also we have:
\begin{align*}
u'(\delta(R)) &= \sum_{(u,v) \in \delta(R)} (u_{uv} - x_{uv} + x_{vu}) &\text{by definition of residual capacity}\\
&= u(\delta(R)) - x(\delta(R)) + x(\delta(\bar{R})) \\
&= u(\delta(R)) - f_x(s) &\text{by Proposition $3.3$} \\
&\geq f_{x^*}(s) - f_x(s). &\text{by Max-Flow Min-Cut} \\
\end{align*}
Thus we have $u'(\delta(R)) \leq Km$ and $u'(\delta(R)) \geq f_{x^*}(s) - f_x(s)$ so by transitivity $Km \geq f_{x^*}(s) - f_x(s)$. Therefore $f_x(s) \geq f_{x^*}(s) - Km$ as desired. $\blacksquare$
\subsection*{b}
\paragraph{}
Let $u \in \Z_+^A$ and let $u_{max}=\max_{a\in A}u_a$. Consider the algorithm which results from modifying the in-class augmenting path algorithm as follows: In stage $i\geq 1$, augment flows using only augmenting paths with residual capacity at least $u_{max}/2^i$. We will show this leads to a polynomial time algorithm for solving the max flow problem.
\paragraph{Lemma 2.1}
For each stage stage of the algorithm there are at most $2m$ augmentations.
\paragraph{Proof of Lemma 2.1}
Suppose the algorithm is beginning stage $i$. Let $x$ denote the current flow found by the algorithm, and $x^*$ denote the optimal flow. Let $K$ be as in $(2a)$. Then by $(2a)$, $$f_{x*}(s) \leq f_x(s) + Km.$$ Since the algorithm just finished stage $i-1$, $$K \geq \frac{u_{max}}{2^{i-1}}.$$
Thus,
\begin{align}
f_{x^*}(s) &\leq f_x(s) + m\frac{u_{max}}{2^{i-1}} \nonumber\\
&= f_x(s) + 2m\frac{u_{max}}{2^i}. 
\end{align}
Now each augmentation in stage $i$ increases the value of $f_x(s)$ by at least $\frac{u_{max}}{2^i}$, so by $(2)$ there are at most $2m$ such augmentations (as otherwise $f_x(s)$ would exceed $f_{x^*}(s)$ which cannot happen by maximality). $\blacksquare$
\paragraph{}
After stage $i = log_2(u_{max})$, there will be no residual graph $r-s$ dipaths of residual capacity greater than or equal to:
$$
\frac{u_{max}}{2^i} = \frac{u_{max}}{u_{max}} = 1.
$$
Thus the residual capacity of every $r-s$ dipath in the residual graph is $0$, and so there are no flow augmenting paths in the graph. Therefore the algorithm has found a max flow after this stage by Theorem $3.6$. This algorithm terminates after $O(log_2(u_{max}))$ stages each consisting of $O(m)$ augmentations. Therefore the algorithm terminates in polynomial time as $O(mlog_2(u_{max}))$ is polynomial in the size of the input since $log_2(u_{max})$ is the size of the binary representation of $u_{max}$. $\blacksquare$

\section*{3}
\paragraph{}
Let $G=(V,E)$ be a network with source $r$ and sink $s$. Let $u \in (\R_+ \cup \{\infty\})^E$. Suppose there exists an arc $rv \in E$ such that $u_{rv} = \infty$. We may assume that $v \neq s$ as otherwise the maximum flow is unbounded. Let $x^*$ be a max flow on $G$. We claim that $f_{x^*}(s) = f_{x'}(s)$ where $x'$ is a finite max flow on $G'=G/rv$ with capacities $u' \in (\R_+ \cup \{\infty\})^{E\backslash\{rs\}}$.
\paragraph{}
To show this claim proceed by induction on the number of $\infty$ capacity arcs in $E$. Suppose there is only one $\infty$ capacity arc $rv$. Let $r'$ be the vertex of $G'$ resulting from contracting $rv$. This vertex is to be considered the source of $G'$. Since $v\neq s$, the sink $s$ remains the sink in $G'$. Since $G$ is connected and acyclic, $G'$ is connected and acyclic as the sets of graphs with these properties are closed under taking contractions. Since $G$ had only one $\infty$ capacity arc, $G'$ has no $\infty$ capacity edges. Therefore the min cut in $G'$ is finite valued, and thus by max-flow min-cut the max flow is finite valued. Let $x'$ be the max flow on $G'$. Let $R'$ be the min cut of $G'$. Then $u'(\delta(R')) = f_{x'}(s)$. Now $r' \in R'$ and $s \in \bar{R'}$. Let $R = R' \backslash\{r'\} \cup \{r,v\}$. Then $r \in R$ and $s \in \bar{R} = \bar{R'}$. So $R$ is an $rs$-cut in $G$ with $\delta(R) = \delta(R')$ and $u(\delta(R)) = u'(\delta(R')$. Since $R$ is an $rs$-cut, $f_{x^*}(s) \leq u(\delta(R))$ by max-flow min-cut. Suppose for a contradiction that $R$ is not a min cut of $G$. Let $R^*$ be a min cut of $G$. Then $u(\delta(R^*)) < u(\delta(R))$. Since $u(\delta(R))$ is finite, $u(\delta(R^*))$ is finite, and thus $rv \not\in \delta(R^*)$. But $r \in R^*$ (as $R^*$ is an $rs$-cut), so $v \in R^*$. Let $R'' = R^* \backslash \{r,v\} \cup \{r'\}$. Then $R''$ is an $r's$-cut in $G'$ and $\delta(R'') = \delta(R)$. But then:
\begin{align*}
u'(\delta(R'')) &= u(\delta(R^*)) \\
&< u(\delta(R)) \\
= u'(\delta(R')).
\end{align*}
This contradicts that $R'$ is a min cut in $G'$. Therefore $R$ is a min cut of $G$, and thus $f_{x*}(s) = u(\delta(R)) = u'(\delta(R')) = f_{x'}(s)$ as desired.
\paragraph{}
Now suppose for induction that for all graphs with at least one but less than $n$ arcs of $\infty$ capacity the claim holds. Let $n$ be the number of $\infty$ capacity $rv$ arcs of $G$. Let $rv$ be one such arc. Let $G'$, $r'$, $u'$ be as discussed previously. Let $x'$ be a max flow on $G'$. Now $G'$ has less than $n$ $\infty$ capacity arcs since $rv$ was contracted. Let $r'v' \in E(G')$ such that $u_{r'v'} = \infty$. Let $x''$ be a max flow on $G'/r'v'$. By the induction hypothesis $f_{x'}(s) = f_{x''}(s)$ which is finite. Therefore the min cut in $G'$ is finite valued, and thus by max-flow min-cut the max flow is finite valued. Let $R'$ be the min cut of $G'$. Then $u'(\delta(R')) = f_{x'}(s)$. Now $r' \in R'$ and $s \in \bar{R'}$. Let $R = R' \backslash\{r'\} \cup \{r,v\}$. Then $r \in R$ and $s \in \bar{R} = \bar{R'}$. So $R$ is an $rs$-cut in $G$ with $\delta(R) = \delta(R')$ and $u(\delta(R)) = u'(\delta(R')$. Since $R$ is an $rs$-cut, $f_{x^*}(s) \leq u(\delta(R))$ by max-flow min-cut. Suppose for a contradiction that $R$ is not a min cut of $G$. Let $R^*$ be a min cut of $G$. Then $u(\delta(R^*)) < u(\delta(R))$. Since $u(\delta(R))$ is finite, $u(\delta(R^*))$ is finite, and thus $rv \not\in \delta(R^*)$. But $r \in R^*$ (as $R^*$ is an $rs$-cut), so $v \in R^*$. Let $R'' = R^* \backslash \{r,v\} \cup \{r'\}$. Then $R''$ is an $r's$-cut in $G'$ and $\delta(R'') = \delta(R)$. But then:
\begin{align*}
u'(\delta(R'')) &= u(\delta(R^*)) \\
&< u(\delta(R)) \\
= u'(\delta(R')).
\end{align*}
This contradicts that $R'$ is a min cut in $G'$. Therefore $R$ is a min cut of $G$, and thus $f_{x*}(s) = u(\delta(R)) = u'(\delta(R')) = f_{x'}(s)$ as desired. So by the principle of mathematical induction the claim holds.
\paragraph{}
This claim gives rise to an algorithm for constructing a maximum flow when $\infty$ capacity arcs are present. Given $G$, $u$, $r$, $v$ the algorithm does the following:
\begin{enumerate}
\item If $G$ has $\infty$ capacity arc from $r$ to $s$ stop, problem is unbounded.\\
\item If $G$ has no $\infty$ capacity arcs return max flow using a standard algorithm\\
\item Else let $rv$ be an $\infty$ capacity arc.\\
\item Call the algorithm recursively on $G' = G/rv$.\\
\item Let $x'$ be the flow returned. Construct the max flow on $G$, $x^*$ as follows:\\
\subitem For all outgoing arcs from $v$, $vu$, let $x^*{vu} = x'_{r'u}$.\\
\subitem Set $x^*_{rv} = \sum_{a \in \delta^+(v)} x^*_{a}$.\\
\subitem For all arcs $r'u$ such that $u$ is not head of an outgoing arc from $v$, let $x^*_{ru} = x'_{r'u}$,\\
\subitem For all remaing arcs $uw$ set $x^*_{uw} = x'_{uw}$.
\end{enumerate}
\paragraph{}
This algorithm constructs a flow on $G$ of the same value as a flow on the contracted graph $G'$, so by our claim, provided that the constructed flow is feasible, the constructed flow is maximal. Indeed the only vertex in the graph $G$ for which it is not clear that the flow passing through in $x^*$ and $x'$ is the same is $v$, but in the second step of constructing the flow it is set to be such, and hence since $x'$ is feasible so is $x^*$. Therefore this algorithm constructs a maximum flow for $G$. This algorithm runs in polynomial time because calling a standard max flow returns in polynomial time, contracting an edge is polynomial time, and building the new flow can be done in polynomial time by scanning the edge sets of $G$ and $G'$. $\blacksquare$

\section*{4}
\end{document}
