\documentclass[letterpaper,12pt,oneside,onecolumn]{report}
\usepackage{amsmath, amssymb, amstext}
\usepackage{fancyhdr}
\usepackage{algorithm}
\usepackage{algpseudocode}
\pagestyle{fancy}

\listfiles

\setlength{\hoffset}{0pt}			% 1 inch left margin
\setlength{\oddsidemargin}{0pt}		% 1 inch left margin
\setlength{\voffset}{0pt}			% 1 inch top margin
\setlength{\marginparwidth}{0pt}	% no margin notes
\setlength{\marginparsep}{0pt}		% no margin notes
\setlength{\textwidth}{6.375in}
\raggedbottom

\rhead{William Justin Toth 650 2} %CHANGE n to ASSIGNMENT NUMBER ijk TO COURSE CODE
\renewcommand{\headrulewidth}{0pt}
%\renewcommand{\baselinestretch}{1.2} % 1.2 line spacing for legibility (optional)

\begin{document}
\section*{1}
\subsection*{a}
\paragraph{}
Let $M = (S,I)$ be the uniform matroid of rank $2$ over four elements. Then we can write $S = \{e_1, e_2, e_3, e_4\}$ and $I = \{\{e_1, e_2\}, \{e_1, e_3\}, \{e_1,e_4\},\{e_2,e_3\}, \{e_2,e_4\}, \{e_3,e_4\}, \{e_1\}, \{e_2\}, \{e_3\}, \{e_4\}, \emptyset\}$. Suppose for contradiction that $M$ is a graphic matroid corresponding to the graph $G = (V, S)$. Now $\{e_1,e_2,e_3\} \not\in I$ implies $e_1, e_2,e_3$ for a $3$-cycle in $G$. Since $\{e_1, e_2\} \in I$, $e_1,e_2$ forms a forest so they intersect at at most one vertex. Since $e_1,e_2,e_3$ forms a cycle, $e_1,e_2$ is a connected path. Therefore $|e_1 \cap e_2| = 1$. Let $v \in V$ such that $v \in e_1$ and $v \in e_2$. Let $e_1 = \{w,v\}$ and $e_2 = \{u,v\}$ with $u \neq w$. Since $e_1,e_2,e_3$ is a cycle on three vertices and $e_3 \neq e_1$ and $e_3 \neq e_2$, $e_3 = \{w,u\}$. Similarly $\{e_1,e_2,e_4\} \not\in I$ so $e_1,e_2,e_4$ forms a $3$-cycle in $G$, but as before $e_4 = \{w,u\}$. Thus $e_3 = e_4$ contradicting $|S| = 4$. $\blacksquare$
\subsection*{b}
\paragraph{}
Let $D = (V,A)$ be a directed graph. Let $N$ be the incidence matrix of $D$ of dimensions $n \times m.$ Let $M = (S,I)$ be the linear matroid defined by $N$ (over the rationals) where $S = \{1, \dots, m\}$ indexes columns of $N$. Let $n_i$ be the $i$th column of $N$. Let $G = (V,E)$ be the undirected graph corresponding to $D$. Write $e_i$ for the edge corresponding to column $i$ in $N$. In this way $S$ also indexes the edges of $G$. We will show the independence relation of the graphic matroid on $G$ corresponds to the independence relation of the linear matroid $M$, in this way demonstrating $M$'s equivalence to the graphic matroid on $G$ since it is already clear their ground sets correspond.
\paragraph{}
Let $B \subseteq I$. Let $M_B = \{n_i : i \in B\}$. Suppose $M_B$ is linearly independent. Let $E_B = \{e_i : i \in B\}$. Suppose $E_B$ contains a cycle in $G$, call it $C$. The idea of what follows is to "orient" the cycle $C$ in $D$ so that all the arcs face the same direction. Consider an edge $\{v_i, v_j \} \in E(C)$. Let $P$ be the path from $v_i$ to $v_j$ on $C - \{v_i,v_j\}$. That is, $P$ is the path along $C$ avoiding the $\{v_i, v_j\}$ edge. Let $a_i$ be the arc in $D$ corresponding to $e_i$ in $G$. Choose $\beta_k = 1$ if $a_k$ is forward on $P$ and $\beta_k = -1$ if $a_k$ is backward on $P$. Let $\{v_i, v_{i+1}\}$ be an edge of $P$. The corresponding arc is forward on $P$ in $D$, if it is of the form $(v_i, v_{i+})$ and it is backward otherwise.
\paragraph{}
For all $v \in V(C)$, $deg_Cv = 2$. Thus there are exactly two edges $e_i \neq e_j \in E_B$ incident to $v$. Let $(m_i)_v$ denote the $v$th entry of column $m_i$. Then $\beta_i m_i$ and $\beta_j m_j$ correspond to an incoming and outgoing arc with respect to $v$. Suppose without loss of generality $\beta_i m_i$ is incoming and $\beta_j m_j$ is outgoing. Then $(\beta_i m_i)_v = 1$ and $(\beta_j m_j)_v = -1$. Therefore $(\beta_i m_i)_v + (beta_j m_j)v = 0$. Further for all $e_k \in E(C)$ such that $k \neq i$ and $k \neq j$, $(m_k)_v = 0$. Thus $\sum_{e_i \in E(C)} \beta_i m_i = 0$. So choose $\alpha_k = \beta_k$ if $e_k \in E(C)$ and $\alpha_k = 0$ otherwise. Then $\sum_{i\in B} \alpha_i m_i = \sum_{e_i \in E(C)} \beta_i m_i = 0$. This implies $M_B$ is linearly dependent, a contradiction. Thus $E_B$ is acyclic, that is to say a forest.
\paragraph{}
Let $F \subseteq E$. Let $B$ be the index set of the edges of $F$. Suppose that $(V,F)$ is a forest. We want to show $M_B = \{n_i : i \in B\}$ is linearly independent. Proceed by induction on $|F|$. If $|F| = 0$ the result holds trivially, similarly if $|F| = 1$ the result holds trivially since a single column vector is always linearly independent. Suppose for all $t \in \mathbb{Z}$ such that $1 \leq t$ that the set of columns of $M$ corresponding to a sub-forest of $G$ of size $t$ is linearly independent. Let $|F| = t+1$. 
\paragraph{}
Consider a leaf vertex of $F$, $l$, and the corresponding edge $e_i \in E(F)$ such that $e_i = \{l,w\}$ for some $w$ in $V(F)$. Then $(m_i)_l = \pm 1$ and $(m_i)_w = \pm 1$. Now the $deg_Fl = 1$ so for all $j \in B$ such that $j \neq i$, $(m_j)\l = 0$. Thus $\sum_{k \in B} \alpha_k m_k = 0$ if and only if $\alpha_i = 0$ since $\sum_{k \in B} \alpha_k (m_k)l = \alpha_i (m_i )l = \pm \alpha_i$. By the induction hypothesis, $M_{B-\{i\}}$ is linearly independent. So $\sum_{k \in B,\ k\neq i} \alpha_k m_k = 0$ if and only if $\alpha_k = 0$ for all $k \neq i$. Thus, $\sum_{k \in B} \alpha_k m_k = \sum_{k \in B,\ k\neq i} (\alpha_k m_k) + \alpha_i m_i = 0$ if and only if $\alpha_k = 0$ for all $k \in B$. Therefore $M_B$ is linearly independent. 
\paragraph{}
So we have that for all $B \subseteq I$, $M_B$ is linearly independent if and only if $E_B$ is a forest. Thus the $M$ is the graphic matroid of $G$. $\blacksquare$
\subsection*{c}
\paragraph{}
Let $G = (V,E)$ be an undirected graph and let $\mathcal{I} := \{J \subseteq E : \kappa(G - J) = \kappa(G)\}$. Let $M = (E,\mathcal{I})$. We will show that $M$ is a matroid by demonstrating that the matroid axioms hold for $M$.
\paragraph{M1}
Let $J = \emptyset$. Then $J \subseteq E$ and $\kappa(G-J)  = \kappa(G - \emptyset) = \kappa(G)$, so $J \in \mathcal{I}$.
\paragraph{M2}
Let $J \in \mathcal{I}$ and let $J' \subseteq J$. Then $\kappa(G-J') \geq \kappa(G-J) = \kappa(G)$, but $\kappa(G-J')$ cannot be greater than $\kappa(G)$ so $\kappa(G-J') = \kappa(G)$. Therefore $J' \in \mathcal{I}$.
\paragraph{M3}
Let $J \in \mathcal{I}$. Suppose that $|J|$ is maximal. Let $G_1$ be a connected component of $G$. Since $\kappa(G-J) = \kappa(G)$, $G_1 - J$ is connected. If there exists $e \in E(G_1)$ such that $e \not\in J$ and $G_1 - e$ is connected then $\kappa(G-(J \cup \{e\})) = \kappa(G)$ so $J \cup \{e\} \in \mathcal{I}$, but $J$ is maximal, a contradiction. Therefore for every $e \in E(G_1-J)$, $(G_1 - J) - e$ is disconnected. That is $G_1 - J$ is minimally connected. So $G_1 - J$is a tree. Then $G - J$ is a forest since its connected components are tree. Therefore $|E-J| = |V| - \kappa(G-J)$, and thus $|J| = |E| - |V| + \kappa(G-J) = |E| - |V| + \kappa(G)$. That is, every maximal independent subset of $\mathcal{I}$ has the same size, $|E| - |V| + \kappa(G)$. $\blacksquare$
\subsection*{d}
\paragraph{}
Let $G=(V,E)$ be an undirected graph. Let $\mathcal{I} := \{J \subseteq E : $ each component of (V,J) has at most one circuit$\}$. Let $M = (E, \mathcal{I})$. Let $\mathcal{C} = \{C \subseteq E : $ there exists a connected component of $(V,C)$ that contains at least two circuits and for all $e \in E(C), C - e \in \mathcal{I} \}$. We will demonstrate that $M$ is a matroid by showing that $\mathcal{C}$ satisfies the properties for a family of circuits of $M$.
\paragraph{C1}
Since $(V, \emptyset)$ has no circuits, $\emptyset \not\in \mathcal{C}$.
\paragraph{C2}
Let $C_1, C_2 \in \mathcal{C}$, and $C_1 \subseteq C_2$. Suppose for contradiction that $C_1 \neq C_2$. Then $|C_1| < |C_2|$. So there exists $e \in C_2$ such that $e \not\in C_1$. So $C_1 \subseteq C_2 - e$. But $C_2 - e \in \mathcal{I}$. Thus every connected component of $(V,C_2 - e)$ has at most one circuit. Edge deletion cannot create circuits so $C_1 \subset C_2 - e$ implies that every connected component of $(V,C_1)$ has at most one circuit. A contradiction. Therefore $C_1 = C_2$.
\paragraph{C3}
Let $C_1, C_2 \in \mathcal{C}$. Suppose $C_1 \neq C_2$ and $e \in C_1 \cap C_2$.
\subparagraph{Lemma 1.1}
Let $C \in \mathcal{C}$. Then $(V,C)$ contains exactly one connected component with a non-empty edge set.
\subparagraph{Proof}
The graph $(V,C)$ contains at least one connected component with a non-empty edge set by definition. Suppose for contradiction that there are at least two connected components of $(V,C)$ with non-empty edge sets.  Since $C \in \mathcal{C}$ there exists a connected component of $(V,C)$, $G_1$, such that $G_1$ has two circuits. Since $(V,C)$ has two connected components with non-empty edge sets, there is a connected component distinct from $G_1$, $G_2$, $E(G_2) \neq \emptyset$. Let $e \in E(G_2)$. Then $(V, C-e)$ contains $G_1$ as a connected component. So $(V, C-e) \not\in \mathcal{I}$ since it contains a connected component with two circuits. A contradiction. $\blacksquare$
\subparagraph{Lemma 1.2}
Let $C \in \mathcal{C}$. Let $G_1$ be the connected component of $(V,C)$ with a non-empty edge set. Let $e \in E(G_1)$. Then both connected components of $G_1 - e$ have a circuit.
\subparagraph{Proof}
By the definition of $C$, $G_1$ contains two distinct cycle $C'$ and $C''$. Suppose $G_1 - e$ is connected. If $e \not\in E(C')$ then $C'$ is contained in $G_1 - e$ as desired. If $e \not\in E(C'')$ then $C''$ is contained in $G_1 - e$ as desired. So we may assume $e \in E(C')$, and $e \in E(C'')$. Let $e = \{x,y\}$. Let $P_1$ be  the $xy$-path along $C'-e$. Let $P_2$ be the $yx$-path along $C''-e$. Then there is a circuit on $P_1P_2$ since $C' \neq C''$. Thus $G_1 - e$ contains a circuit. Now suppose $G_1 - e$ is disconnected. Then $G_1 - e$ has two connected components. Call them $G'$ and $G''$. In this case $e$ does not lie any circuit in $G_1$, as otherwise $G_1 - e$ would be connected. Now if no circuit lies in $G'$ or no circuit lies in $G''$ then one $G'$ or $G''$ contains at least two circuits, implying that $C - e \not \in \mathcal{I}$, a contradiction. So a circuit lies in each of $G'$, $G''$.$\blacksquare$
\paragraph{Back to C3}
Let $D = (C_1 - \{e\}) \cup (C_2 - \{e\}) = (C_1 \cup C_2) - \{e\}$. Let $G_1$ be the connected component of $(V,C_1)$ such that $E(G_1) \neq \emptyset$ (invoking Lemma $1.1$). Similarly let $G_2$ be the connected component of $(V,C_2)$ such that $E(G_2) \neq \emptyset$. Let $e = \{x,y\}$. Let $G'$ be the connected component of $G_1 - e$ containing $x$. Let $G''$ be the connected component of $G_2 - e$ containing $x$. By lemma $1.2$, $G'1$ has a circuit $C'$ and $G''$ has a circuit $C''$. If they have distinct $C'$ and $C''$ then $G' \cup G''$ is a connected component of $(V,D)$ containing two or more circuits. If they do not have circuits $C'$, $C''$ such that $C' \neq C''$ then $G' = G''$. Since $C_1 \neq C_2$, $G_1 - e$ and $G_2 - e$ must differ. In this case take $G'$ to be the connected component of $G_1 - e$ containing $y$, and $G''$ to be the connected component of $G_2 -e$ containing $y$. In this case they must have distinct cycles or else $C_1 = C_2$. Thus $G' \cup G''$ is a connected component of $(V,D)$ containing two or more cycles. Now iteratively delete edges from $D$ until only the edges of the two circuits in $G' \cup G''$, and possibly an edge connecting them if they are vertex disjoint, remains. This subset of $D$, call it $D^*$ is such that deleting any element of $D^*$ results in an independent set. That is $D^* \subseteq D$ such that $D^* \in \mathcal{C}$.
\paragraph{}
Thus $\mathcal{C}$ satisfies the circuit family properties for $M$, so we conclude that $M$ is a matroid. $\blacksquare$
\section*{2}
\section*{3}
\section*{4}
\paragraph{}
Let $g : 2^S \rightarrow \mathbb{R}_+$ be a submodular function and let $P(g)$ be the corresponding nonempty polymatroid. Choose $f : 2^S \rightarrow \mathbb{R}_+$ to be defined by $f(A) = max_{x \in P(g)} x(A)$. Then $f(\emptyset) = max_{x \in P(g)} x(\emptyset) = 0$. Let $A, B \subseteq S$. Then we have:
\begin{align*}
f(A \cup B) + f(A \cap B) &= max_{x \in P(g)} x(A \cup B) + max_{x \in P(g)} x(A \cap B) \\
&\leq max_{x \in P(g)}(x(A) + x(B) - x(A\cap B)) + max_{x \in P(g)} x(A \cap B) \\
&\leq max_{x \in P(g)}x(A) + max_{x \in P(g)}x(B) - max_{x \in P(g)} x(A \cap B) + max_{x \in P(g)} x(A \cap B) \\
& = f(A) + f(B) 
\end{align*}
so $f$ is submodular. Now let $A \subseteq B \subseteq S$. Then we have:
\begin{align*}
f (A) &= max_{x \in P(g)}x(A) \\
&\leq max_{x \in P(g)} x(B) &\text{ Since $A\subseteq B$} \\
&= f(B)
\end{align*}
so $f$ is monotone.
\paragraph{}
Let $x \in P(g)$. Let $A \subseteq S$. Then $x(A) \leq max_{x \in P(g)} x(A) \leq f(A)$. So $x \in P(f)$. Now let $x \in P(f)$. Let $A \subseteq S$. Then $x(A) \leq f(A) = max_{x \in P(g)} x(A) \leq g(A)$. So $x \in P(g)$. Therefore $P(f) = P(g)$. $\blacksquare$
\end{document}
