\documentclass[letterpaper,12pt,oneside,onecolumn]{report}
\usepackage{amsmath, amssymb, amstext}
\usepackage{fancyhdr}
\usepackage{algorithm}
\usepackage{algpseudocode}
\usepackage{mathtools}

\DeclarePairedDelimiter{\ceil}{\lceil}{\rceil}

\pagestyle{fancy}

\listfiles

\setlength{\hoffset}{0pt}			% 1 inch left margin
\setlength{\oddsidemargin}{0pt}		% 1 inch left margin
\setlength{\voffset}{0pt}			% 1 inch top margin
\setlength{\marginparwidth}{0pt}	% no margin notes
\setlength{\marginparsep}{0pt}		% no margin notes
\setlength{\textwidth}{6.375in}
\raggedbottom

\rhead{William Justin Toth 650 4} %CHANGE n to ASSIGNMENT NUMBER ijk TO COURSE CODE
\renewcommand{\headrulewidth}{0pt}
%\renewcommand{\baselinestretch}{1.2} % 1.2 line spacing for legibility (optional)

\begin{document}
\section*{1}

\section*{2}
\paragraph{}
Let $G = (V,E)$ be a graph. Suppose that $G$ is connected and that every node, $v$, is inessential. Let $v \in V$. We will show that $G-v$ has a perfect matching. Let $M$ be a maximal matching in $G$ such that $v$ is $M$-exposed. Such a matching exists as $v$ is inessential. 
\paragraph{}
Let $w\in V\backslash \{v\}$. Suppose for contradiction that $w$ is $M$-exposed. Since $G$ is connected there is a $vw$-path in $G$, call it $P$. Since $w$ is $M$-exposed there exists an $M$-exposed vertex in $V(P)$. Let $w'$ be the $M$-exposed vertex nearest to $v$ along the path $P$.
\paragraph{}
Suppose for contradiction that $vPw'$ (the subpath of $P$ from $v$ to $w'$) is not $M$-alternating. Then there exists an $M$-exposed vertex $u$ not equal to $v$ or $w'$, as otherwise every edge alternates being in $M$ and not being in $M$. But $vPu$ is a shorter path than $vPw'$ contradicting the minimality of $w'$. Thus $vPw'$ is $M$-alternating.
\paragraph{}
Since $vPw'$ is $M$-alternating and both $v$ and $w'$ are $M$-exposed, $vPw'$ is an $M$-augmenting path. But by the Augmenting Path Theorem this contradicts the maximality of $M$. Thus for all $w \in V\backslash \{v\}$, $w$ is not $M$-exposed. Therefore $M$ covers every vertex in $V\backslash \{v\}$. Since $v$ is $M$-exposed, $\delta(v) \cap M = \emptyset$. Therefore $M$ is a perfect matching in $G-v$.
\paragraph{}
Now for all $v \in V$, $G-v$ has a perfect matching. Therefore $G$ is factor critical. To conclude that $\nu(G) = (|V| - 1)/2$ observe two things. First any perfect matching of $G-v$ is a matching in $G$ so $\nu(G) \geq (|V|-1)/2$. Also $|V\backslash \{v\}|$ is even as otherwise a perfect matching does not exist. Thus $|V|$ is odd. Now, secondly, observe that by the Tutte-Berge formula (as an inequality) we have $\nu(G) \leq \frac{1}{2}(|V| - oc(G) + |\emptyset|) = (|V| - 1 + 0)/2 = (|V| - 1)/2$. Therefore $\nu(G) = (|V|-1)/2$. $\blacksquare$

\section*{3}
Let $B$ be the set of inessential nodes in a given undirected graph $G$. Let $C$ be the set of essential neighbours of $B$, and let $D$ be all remaining nodes.
\subsection*{a}
\paragraph{Lemma 3.1}
Let $v \in C$. Let $M'$ be a maximal matching in $G' = G-v$. Let $M$ be a maximal matching in $G$. Then $|M'| + 1 = |M|$.
\paragraph{Proof of Lemma 3.1}
The matching $M'$ is a matching in $G$ where $v$ is $M'$-exposed. Since $v$ is essential $M'$ is not a maximal matching in $G$. Therefore $|M'| < |M|$. That is, $|M'| + 1 \leq |M|$ since the size of sets is integral. Suppose for contradiction that $|M| \geq |M'| + 2$. Then let $e=vu \in M$. Such an edge exists since $M$ is maximal and $v$ is essential. Now $|M\backslash\{e\}| = |M| - 1 \geq |M'| + 1$. But then $M\backslash\{e\}$ is a matching in $G'$ of larger size than $M'$, contradicting the maximality of $M'$. Therefore $|M| < |M'| +2$. That is $|M| \leq |M'| + 1$. So $|M'| + 1 \leq |M| \leq |M'|+1$, and thus $|M| = |M'| + 1$. $\blacksquare$
\paragraph{}
Let $G' = G-v$ be the graph obtained from $G$ by removing node $v \in C$. Let $B'$ be the set of inessential nodes of $G'$, let $C'$ be the set of essential neighbours of $B'$, and let $D'$ be the remaining nodes of $G'$.
\paragraph{}
Let $b \in B$. Since $b$ is inessential for $G$ there exists a maximal matching $M$ of $G$ such that $b$ is $M$-exposed. Now $v$ is essential for $G$ so $v$ is $M$-covered. Therefore there exists $e=vu \in M$. Now $M\backslash\{e\}$ is a $G'$ matching where $b$ is $M\backslash\{e\}$-exposed. Suppose for contradiction that $M\backslash\{e\}$ is not maximal for $G'$. That is that there exists a maximal matching $M'$ for $G'$ such that $|M\backslash\{e\}| < |M'|$. Then $|M| - 1 < |M'|$, that is $|M| < |M'| + 1$. Since size of matchings are integral, $|M| \leq |M'|$. Now $E(M') \subseteq E(G') \subseteq E(G)$. Thus $M'$ is a matching of $G$. Since $M$ is a maximal matching for $G$, $|M'| \leq |M|$. Thus $|M'| = |M|$. Thus $M'$ is a maximal matching for $G$. But $M'$ is a $G'$ matching, so $v$ is $M'$-exposed. Therefore there is a maximal matching of $G$ where $v$ is exposed, contradicting that $v$ is essential for $G$. So $M\backslash\{e\}$ is maximal for $G'$. Since $b$ is $M\backslash\{e\}$-exposed $b$ is inessential for $G'$, that is $b \in B'$. Therefore $B \subseteq B'$. 
\paragraph{}
Let $b \in B'$. Suppose for contradiction that $b \not\in B$. Let $u \in \delta(v) \cap B$. Such a $u$ exists by the definition of $C$. Let $M$ be a maximal matching in $G$ such that $u$ in $M$-exposed. Let $M'$ be a maximal matching in $G'$ such that $b$ is $M'$-exposed. Then $M'$ is a matching in $G$ where $v$ is $M'$-exposed and $b$ is $M'$-exposed. By Lemma $3.1$, $|M'| + 1 = |M|$.
\paragraph{}
Now consider the graph with vertex set $V(G)$ and edge set $M \triangle M'$. If $u$ is $M'$-exposed then $M' \cup \{uv\}$ is a matching in $G$ as $v$ is $M'$-exposed. Such a matching is maximal as $|M' \cup \{uv\}| = |M'| + 1 = |M|$. Thus $u$ is $M'$-covered. Let $P$ be the path in this graph starting at $u$. The edges of $P$ alternate between $M'$ and $M$. If $P$ terminates at an $M$-exposed vertex then $P$ is an $M$-augmenting path in $G$, contradicting the maximality of $M$. So $P$ ends at an $M'$-exposed vertex. Since $b$ and $v$ are $M'$-exposed, they are not internal vertices of $P$. hence either $b$ is not an endpoint of $P$ or $v$ is not an endpoint of $P$.
\paragraph{}
If $b$ is not an endpoint of $P$ then $M' \triangle P$ is a maximal matching of $G$ (as it is a matching and $|M' \triangle P| = |M'| + 1 = |M|$) such that $b$ is $M' \triangle P$-exposed. This contradicts that $b$ is essential in $G$. So $v$ is not an endpoint of $P$. Then similarly, $M' \triangle P$ is a maximal matching of $G$ such that $v$ is $M' \triangle P$-exposed. This contradicts that $v$ is essential in $G$. Thus $b \in B$, that is $B' \subseteq B$.
\paragraph{}
Since $B \subseteq B'$ and $B' \subseteq B$, $B = B'$. Then $C' = \{ \text{essential neighbours of $B'$ in $G'$}\} = \{\text{essential neighbours of $B$ in $G-v$} \} = C \backslash \{ v \}$. Further $D' = V\backslash \{ v\} \backslash (\{ B' \cup C' \} ) = V\backslash \{ v\} \backslash (B \cup (C\backslash \{ v\} )) = D$. Therefore the Gallai-Edmonds partition of $G'$ is $B,C\backslash \{v\}, D$. $\blacksquare$ 
\subsection*{b}
\paragraph{Lemma 3.2}
For all $S \subseteq C$, the Gallai-Edmonds partition of $G\backslash S$ is $B, C\backslash S, D$.
\paragraph{Proof of Lemma 3.2}
Proceed by induction on $|S|$. If $|S|  = 0$ then $S = \emptyset$ and the lemma is immediate. Suppose the lemma holds for all subsets of size less that $|S|$. Let $v \in S$. Then $G\backslash S = G\backslash (S \backslash \{v\}) - v$. By induction the Gallai-Edmonds partition of $G\backslash (S\backslash \{v\})$ is $\{B, C\backslash (S\backslash \{v\}), D \}$. So by $(3a)$ the Gallai-Edmonds partition of $G\backslash (S \backslash \{v\}) - v = G\backslash S$ is $\{B, C\backslash (S\backslash \{v\})\backslash \{v\}, D\} = \{V, C\backslash S, D\}$. Thus by the principle of mathematical induction the lemma holds. $\blacksquare$
\paragraph{Corollary 3.2}
Every vertex of $G[B]$ is inessential and every vertex of $G[D]$ is essential.
\paragraph{Proof of Corollary 3.2}
Let $S = C$. By Lemma $3.2$, $G\backslash C$ has Gallai-Edmonds partition $\{B, \emptyset, D\}$. But $G \backslash C = G[B] \dot \cup G[D]$. Thus $G[B]$ has Gallai-Edmonds partition $\{B, \emptyset, ]\emptyset\}$ and $G[D]$ has Gallai-Edmonds partition $\{\emptyset, \emptyset, D\}$. That is, every vertex of $G[B]$ is inessential and every vertex of $G[D]$ is essential. $\blacksquare$
\paragraph{}
Let $G'$ be a connected component of $G[B]$. By Corollary $3.2$ every vertex of $G[B]$ is inessential, so every vertex of $G'$ is inessential. Further $G'$ is connected by $(2)$, so $G'$ is factor critical. By Corollary $3.2$ every vertex of $G[D]$ is essential so in any maximal matching every vertex of $G[D]$ is covered, that is $G[D]$ has a perfect matching. $\blacksquare$
\subsection*{c}
\paragraph{Lemma 3.3}
Let $\{B,C,D\}$ be the Gallai-Edmonds partition of a graph $G$. Then in any maximal matching $M$ there does not exist $c\in C$ and $d \in D$ such that $cd \in M$. 
\paragraph{Proof of Lemma 3.3}
Let $M$ be a maximal matching of $G$. Suppose for contradiction that there exists $c \in C$ and $d \in D$ such that $cd \in M$. Let $G' = G - c$. By $(3a)$ the Gallai-Edmonds partition of $G'$ is $\{B,C\backslash \{c\}, D\}$. Thus $d$ is essetial in $G'$. Let $M' = M \backslash \{cd\}$. Then $M'$ is matching of $G'$. Notice that $|M'| = |M| - |\{cd\}| = |M| - 1$. Therefore by Lemma $3.1$, $M'$ is a maximal matching of $G'$. But $d$ is $M'$-exposed, contradicting that $d$ is essential in $G'$. $\blacksquare$
\paragraph{}
Let $M$ be a maximum matching of $G$. By Lemma $3.3$ there does not exist an edge $cd \in M$ such that $c \in C$ and $d \in D$.  Thus $M = M_1 \dot \cup M_2$ where $M_1$ is a maximal matching of $G[B \cup C]$ and $M_2$ is a maximal matching of $G[D]$. By $(3b)$, $M_2$ is a perfect matching of $G[D]$. Therefore $M$ contains a perfect matching of $G[D]$.
\paragraph{}
Now let $G'$ be a connected component of $G[B]$. Let $M' = M_1 \cap E(G')$. Suppose that $M'$ is not maximal, that is that $|M'| < (|V(G')| - 1)/2$ since $G'$ is factor-critical by $(3b)$. Let $N = \delta(V(G') \cap M_1$ be the set of edges i $M_1$ between $G'$ and $C$. First consider if $|N| = 0$. Then let $M_{PM}$ be a perfect matching on $G' - v$ for some $v \in V(G')$. Then $|M_{PM}| > |M'|$, and $(M_1 \backslash M') \cup M_{PM}$ is matching of $G[B \cup C]$. Such a matching is of size greater than $M_2$, contradicting the maximality of $M_1$.
\paragraph{}
So $|N| \geq 1$. Suppose for contradiction that $|N| \geq 2$. Let $b_1c_1 \in N$ such that $b_1 \in V(G')$ and $c_1 \in C$. Then $N \backslash {b_1c_1}$ is not empty. Let $PM$ be a perfect matching of $G' - b_1$. Let $M_{PM}$ be the matching $((M_1 \backslash M')\backslash N) \cup PM \cup \{b_1c_1\}$. Consider the graph on vertices $B \cup C$ with edge set $M_{PM} \triangle M_1$. This graph is the disjoint union of paths and cycles. Let $b_2c_2 \in N\backslash \{b_1c_1\}$ such that $b_2 \in V(G')$ and $c_2 \in C$. Then $c_2$ is $M_{PM}$-exposed, but $M_1$-covered by the edge $b_1c_1$ in the graph. Let $P$ be the $M_1M_{PM}$-alternating path starting at $c_2$. Since $c_2$ is $M_{PM}$-exposed, $P$ is not a cycle.
\paragraph{}
If $P$ terminates at an $M_1$-exposed vertex then $P$ is an $M_1$-alternating path of even length. So $M_1 \triangle P$ is a matching in $G[B \cup C]$ of the same size as $M_1$, thus it is a maximal matching. But $c_2$ is $M_1 \triangle P$-exposed, contradicting that $c_2$ is essential.
So $P$ does not terminate at an $M_1$-exposed vertex.
\paragraph{}
Then $P$ terminates at an $M_{PM}$-exposed vertex, $c_3$. Such a $v \not\in V(G')$ as $V(G')$ in $M_{PM}$-covered. But $P$ enters $G'$ through its first edge, $b_2c_2$, and terminates outside $G'$, so it exits $G'$ along some edge. Let $b_3c_3 \in N\backslash \{b_1c_1\}$, such that $b_3 \in V(G')$ and $c_3 \in C$, be the edge $P$ exits $G'$ along. Then $c_3$ is $M_{PM}$ exposed so $P$ terminates at $c_3$.
\paragraph{}
Now the paths in the graph on edge set $M_1 \triangle M_{PM}$ are disjoint, and we have shown that for every $b_2c_2 \in N \backslash \{b_1c_1\}$ the path starting at $c_2$ terminates at a distinct $c_3 \in C$ such that $b_3c_3 \in N \backslash \{b_1c_1\}$. Let $\mathcal{P}$ be the collection of all such paths. Then $|\mathcal{P}| = |N \backslash \{b_1c_1\}|/2$. That is, these $M_{PM}$-augmenting paths pair $C$-neightbours of $G'$. Let $M^*$ be the graph resulting from taking the symmetric difference of $M_{PM}$ with each $P \in \mathcal{P}$. Then $|M^*| = |M_{PM}| + |\mathcal{P}|$.
\paragraph{}
Now we consider $|M'|$. Since $|M'|$ is not maximal $|M'| < (|V(G')| - 1)/2$. More tightly $|M'| < (|V(G')| - 1)/2 - |\mathcal{P}|$ as otherwise there are not enough $M_{PM}$ exit edges from $G'$ for the each $P \in \mathcal{P}$ to be edge-disjoint. Now we have:
\begin{align*}
|M_{PM}| &= |M_1| -|M'| - |N| + (|V(G')| - 1)/2  + 1 &\text{by construction}\\
&= |M_1| - |M'| - |N \backslash \{b_1c_1\}| - 1  + (|V(G')| - 1)/2  + 1 \\
&= |M_1| - |M'| - 2|\mathcal{P}| + (|V(G')| - 1)/2 &\text{by the size of $\mathcal{P}$}\\
&> |M_1| - (|V(G')| - 1)/2 + |\mathcal{P}| - 2|\mathcal{P}| + (|V(G')| - 1)/2  &\text{by the size of $|M'|$}\\
&= |M_1| - |\mathcal{P}|.
\end{align*}
Thus $|M_{PM}| > |M_1| - |\mathcal{P}|$. Therefore $|M*| = |M_{PM}| + |\mathcal{P}| > |M_1| - |\mathcal{P}| + |\mathcal{P}| = |M_1|$. So $M^*$ is a larger $G[B \cup C]$ matching than $M_1$, contradicting the maximality of $M_1$. Therefore $|N| = 1$.
\paragraph{}
Now since $|N| = 1$, if $M'$ is not a perfect matching (that is, not maximal), then as in the case of no $N$ edges replacing $M'$ with a perfect matching yields a larger matching than $M_1$ for $G[B \cup C]$. Therefore $M'$ is a perfect matching, that is maximal. So for all connected components of $G[B]$, $G'$, $M_1$ contains a maximal matching of of $G'$, and thus $M$ contains a maximal matching of $G'$. $\blacksquare$
\subsection*{d}
\paragraph{}
Let $k$ be the number of connected components of $G[B]$. Let $M$ be a maximal matching of $G$. As in $(3c)$, $M = M_1 \dot\cup M_2$ where $M_1$ is a maximal matching of $G[B \cup C]$ and $M_2$ is a maximal matching of $G[D]$. Now since $D$ is perfectly matched by $(3c)$, $|M_2| = |D|/2$. Since $M_1$ contains a maximal matching of each connected component of $G[B]$ and a cover of every vertex of $C$, $|M_1| = \sum_{i=1}^k \frac{|V(G_i)| -1}{2} + |C|$ where $G_i$ denotes the $i$-th connected component of $G[B]$. Therefore $|M_1| = \frac{|B| + |C| -k}{2} + \frac{|C|}{2}$. Finally we have that $\nu(G) = |M| = |M_1| + |M_2| = \frac{|B| +|C| -k}{2} + \frac{|C|}{2} + \frac{|D|}{2} = \frac{|B| +|C| +|D| -k}{2} + \frac{|C|}{2} = \frac{n - k + |C|}{2}$. $\blacksquare$
\subsection*{e}
\paragraph{}
The Tutte-Berge formula states that $\nu(G) = \frac{1}{2}min_{S \subseteq V} \{n - oc(G\backslash S) + |S|\}$. We have already noted that $\nu(G) \leq \frac{n - oc(G\backslash S) + |S|}{2}$ for any $S \subseteq V$. So if we can show there is a subset of $V$ for which this upper bound is attained then we are done. Indeed let $S = C$. Since $D$ has a perfect matching $|D| \equiv 0\ (mod\ 2)$, and since every connected component of $G[B]$, $G'$, is factor critical, $|V(G')| \equiv 1\ (mod\ 2)$. Therefore $oc(G\backslash C) = k$ where $k$ is the number of connected components of $G[B]$. Then $\frac{1}{2}(n - oc(G\backslash C) + |C|) = \frac{n - k + |C|}{2}$. By $(3d)$, $\nu(G) = \frac{n-k + |C|}{2}$, so for $S=C$, $\nu(G) = \frac{1}{2}(n - oc(G\backslash S) + |S|)$. That is, the Tutte-Berge formula holds. $\blacksquare$

\section*{4}
\subsection*{a}
\paragraph{}
Let $M$ be a maximal matching of $G$. Let $X = \{v \in V : \forall e \in M, v \not\in e\}$. Then $2|M| + |X| = |V|$, as each edge in $M$ connects two vertices not in $X$. Notice that $X$ is an independent set, as otherwise let $e$ be an edge between two vertices in $X$ and then see that $M \cup e $ is a valid matching of size larger than $M$ contradicting the maximality of $M$. Construct an edge cover $F$ by starting with $M$ and greedily adding edges to cover the remaining vertices in $X$. Since there are no isolated vertices such an edge can always be found. Since $X$ is an independent set one edge will need to be added to $F$ for each vertex in $X$. That is $|F| = |M| + |X|$. Therefore, since $2|M| + |X| = |V|$, we have that $|M| + |F| = |V|$.
\paragraph{}
Let $F^*$ be a minimal edge cover. Since $|M| + |F| = |V|$, $|M| + |F^*| \leq |V|$. Let $G_{F*}$ be the graph with vertex set $V$ and edge set $F^*$. Let $G'$ be a connected component of $G_{F^*}$. Suppose there exists $v_1, v_2 \in V(G')$ such that $v_1 \neq v_2$, and $deg(v_1) \geq 2$ and $deg(v_2)\geq 2$. Then since $G'$ is connected there is a path, $P$, between $v_1$ and $v_2$. Every vertex along $P$ has degree two. Thus every vertex along $P$ is covered by at least two edges. Let $e \in E(P)$. Since every vertex along $P$ is covered by at least two edges, $F^* \backslash \{e\}$ is still an edge cover. But $|F^* \backslash \{e\}| < |F^*|$, contradicting the minimality of $F^*$. Therefore every connected component of $F^*$ has at most one vertex of degree greater than one, that is $F^*$ is the disjoint union of "star" graphs.
\paragraph{}
Let $k$ be the number of connected components of $F^*$. Then $|F^*| + k = |V|$, as the number of edges in $F^*$ count the degree one vertices of each star in $G_{F^*}$, and it remains to count the $k$ higher degree vertices to reach $|V|$. Let $M'$ be the matching obtained by choosing one edge from each connected component of $F^*$. Since the edges were chosen from disjoint connected components, $M'$ is indeed a valid matching. Now $|M'| = k$, so $|F^*| + |M'| = |V|$. But $|M'| \leq |M|$, so in fact $|M| + |F^*| \geq |F^*| + |M'| = |V|$. Therefore we have $|M| + |F^*| \leq |V|$ and $|M| + |F^*| \geq |V|$, hence $|M| + |F^*| = |V|$. That is to say, $\nu(G) + \rho(G) = |V|$. $\blacksquare$
\subsection*{b}
\paragraph{}
Let $c \in \mathbb{R}^E$. Make a copy $G^*$ of $G$. Form a graph $\hat{G}$ by taking $G$ and $G^*$ and adding edges from nodes in $G$ to the corresponding nodes in $G^*$. Let $Q$ denote the set of such edges added joining $G$ and $G^*$. As argued in the textbook, $\hat{G}$ has a perfect matching.
\paragraph{}
Let $\hat{c}$ be the weight function for $\hat{G}$, defined as follows: 
$$
\hat{c}(e) = 
\begin{cases}
c_e, \text{ if $e \in E(G)$ or $e \in E(G^*)$}\\
2min\{c_e : e \in \delta(v)\}, \text{ if $e=vv^* \in Q$}
\end{cases}
$$.
\paragraph{}
Use the minimum weight perfect matching algorithm from the text on $\hat{G}$ with cost function $\hat{c}$ to obtain matching $M$. Obtain an edge cover of $G$, $F$, by starting with $M \cap E(G)$, and for all $q=vv^* \in M\cap Q$ add edge $e \in \delta(v)$, $c_e = \hat{c}(q)$. Then $c(F) = c(M\cap E(G)) + \frac{1}{2}\hat{c}(M \cap Q)$. Now we have:
\begin{align*}
\hat{c}(M) &= 2c(M \cap E(G)) + \hat{c}(Q \cap M) \\
&= 2(c(M \cap E(G)) + \frac{1}{2}\hat{c}(Q\cap M)) \\
&= 2c(F).
\end{align*}
Then $c(F) = \frac{1}{2}\hat{c}(M)$.
\paragraph{}
Let $F^*$ be a minimum weight edge cover of $G$. We will construct a perfect matching of $\hat{G}$ from $F^*$. Denote this matching $M^*$. Let $G_{F^*}$ be the graph with vertex set $V$ and edge set $F^*$. For each $v \in V$ do the following: Let $N(v)$ be the neighbourhood of $v$ in $G_{F^*}$. Let $w \in N(v)$, $c_{wv} = max\{c_{uv} : u \in N(v)\}$. For each $u \in N(v) \backslash \{w\}$, add edge $uu^* \in Q$ to $M^*$ and remove $uv$ from $F^*$.
\paragraph{}
What edges remain in $F^*$ form a matching on $G$. Add these edges to $M^*$, and add the corresponding edges of $G^*$ to $M^*$. Now $M^*$ is a perfect matching of $\hat{G}$ as it contains a matching of $G$, a matching of $G^*$, and the remaining vertices are matched through $Q$ edges. By our construction, $c(F^*) \geq \frac{1}{2}\hat{c}(M^* \cap Q) + c(M^* \cap E(G))$. But $\hat{c}(M^*) = 2c(M* \cap E(G)) + \hat{c}(M^* \cap Q) \leq 2c(F^*)$. Then $c(F^*) \geq \frac{1}{2} \hat{c}(M^*)$. Since $M$ is a minimum cost perfect matching $\hat{c}(M) \leq \hat{c}(M^*)$, and thus $c(F^*) \geq \frac{1}{2} \hat{c}(M^*) \geq \frac{1}{2}\hat{c}(M) = c(F)$. That is, $c(F^*) \geq c(F)$.
\paragraph{}
But $F^*$ is a minimum cost edge cover so $c(F^*) \leq c(F)$, and therefore $c(F^*) = c(F)$. Thus the edge cover $F$ constructed by our process discussed above is a minimum cost edge cover of $G$. To confirm that $F$ is indeed an edge cover, see that $F$ contains a matching of $G$ and an edge (derived from a $Q$ edge) for each vertex unmatched in $G$ and thus covers every vertex. Now it remains to confirm this process runs in polynomial time. The copy of $G$, $G^*$, can be formed in $O(nm)$ time. The edges of $Q$ can be added in $O(n)$ time. So the graph $\hat{G}$ can be formed in polynomial time. The new cost function can be written in constructed in polynomial time in the size of $\hat{G}$ which is itself polynomial in the size of $G$. The minimum cost perfect matching algorithm runs in polynomial time on the size of the input which is itself a polynomial in the size of $G$, so the minimum cost perfect matching algorithm will terminate in polynomial time in the size of $G$. Constructing $F$ from the result can be done in $O(m)$ time. So the total running time of this process is polynomial. $\blacksquare$

\section*{5}
\paragraph{}
Let $G=(V,E)$ be a graph. Suppose that $G$ has no perfect matching. By the Tutte-Berge formula there exists $A \subseteq V$ such that $oc(G\backslash A) > |A|$. Let $k = |A|$. Let $S_1, S_2, \dots, S_k, S_{k+1}$ be distinct odd components of $G\backslash A$.
\paragraph{}
Consider the dual to the linear program:
\begin{align*}
max &\sum_{v\in V} y_v + \sum_{S\subseteq V: |S| \text{ is odd}} y_S \\
s.t. &y_u + y_v + \sum_{uv \in \delta(S), |S| \text{ is odd}} y_S \leq c_{uv} &\text{,}\forall uv \in E \\
&y_S \geq 0 &\text{,}\forall S\subseteq V, |S| \text{ is odd}
\end{align*}
\paragraph{}
Let $c_{uv} = 0$ for all $uv \in E$. Consider the dual solution vector $y$. Set $y_v=0$ for all $v \in V\backslash A$. Let $\epsilon \in \mathbb{R}_+$ and set $y_{S_i} = \epsilon$ for all $i \in \{1, \dots, k+1\}$. Set $y_{S} = 0$ for all odd $S \subseteq V$ such that $S \not \in \{S_1, \dots, S_{k+1}\}$. Set $y_v = -\epsilon$ for all $v \in A$. Notice that $y_v \leq 0$ for all $v \in V$, and $y_S \geq 0$ for all $S \subseteq V$ such that $|S|$ is odd.
\paragraph{}
We claim this $y$ is dual feasible. Let $ab \in E$. If $ab \not\in \delta(S_i)$ for any $i \in \{1, \dots, k+1\}$, then $\sum_{ab \in \delta(S), |S| \text{ is odd}} y_S \leq 0 $ and thus $y_u + y_v + \sum_{uv \in \delta(S), |S| \text{ is odd}} y_S \leq 0 + 0 + 0 = c_{ab}$. If $ab \in \delta(S_i)$  for some $i \in \{1, \dots, k+1\}$ then $\sum_{ab \in \delta(S), |S| \text{ is odd}} y_S = \sum_{i = 1}^{k+1} y_{S_i} = y_{S_i} = \epsilon$. This follows since $ab$ is on the boundary of exactly one $S_i$ since the $S_i$ are all disjoint. Say without loss of generality that $a \in A$ and $b \in S_i$. Then $y_a + y_b + \sum_{ab \in \delta(S), |S| \text{ is odd}} y_S = -\epsilon + 0 + \epsilon = 0 = c_{ab}$. Therefore our choice of $y$ is a feasible dual solution.
\paragraph{}
Now consider the value of the objective function:
\begin{align*}
\sum_{v\in V} y_v + \sum_{S\subseteq V: |S| \text{ is odd}} y_S &= \sum_{v \in A} y_v + \sum_{i=1}^k y_{S_i} \\
&= -k\epsilon + (k+1)\epsilon \\
&= \epsilon.
\end{align*}
Therefore for any choice of epsilon, $y$ is feasible and the object function evaluated at $y$ has value $\epsilon$. Thus the dual linear program is unbounded as we can make the object value arbitrarily large. By the Weak Duality Theorem this implies that the primal linear program is infeasible. $\blacksquare$
\end{document}
