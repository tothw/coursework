\documentclass[letterpaper,12pt,oneside,onecolumn]{article}
\usepackage[margin=1in, bottom=1in, top=1in]{geometry} %1 inch margins
\usepackage{amsmath, amssymb, amstext}
\usepackage{fancyhdr}
\usepackage{algorithm}
\usepackage{algpseudocode}
\usepackage{mathtools}

\DeclarePairedDelimiter{\ceil}{\lceil}{\rceil}
\DeclarePairedDelimiter\floor{\lfloor}{\rfloor}

%Macros
\newcommand{\A}{\mathbb{A}} \newcommand{\C}{\mathbb{C}}
\newcommand{\D}{\mathbb{D}} \newcommand{\F}{\mathbb{F}}
\newcommand{\N}{\mathbb{N}} \newcommand{\R}{\mathbb{R}}
\newcommand{\T}{\mathbb{T}} \newcommand{\Z}{\mathbb{Z}}
\newcommand{\Q}{\mathbb{Q}}
 
 
\newcommand{\cA}{\mathcal{A}} \newcommand{\cB}{\mathcal{B}}
\newcommand{\cC}{\mathcal{C}} \newcommand{\cD}{\mathcal{D}}
\newcommand{\cE}{\mathcal{E}} \newcommand{\cF}{\mathcal{F}}
\newcommand{\cG}{\mathcal{G}} \newcommand{\cH}{\mathcal{H}}
\newcommand{\cI}{\mathcal{I}} \newcommand{\cJ}{\mathcal{J}}
\newcommand{\cK}{\mathcal{K}} \newcommand{\cL}{\mathcal{L}}
\newcommand{\cM}{\mathcal{M}} \newcommand{\cN}{\mathcal{N}}
\newcommand{\cO}{\mathcal{O}} \newcommand{\cP}{\mathcal{P}}
\newcommand{\cQ}{\mathcal{Q}} \newcommand{\cR}{\mathcal{R}}
\newcommand{\cS}{\mathcal{S}} \newcommand{\cT}{\mathcal{T}}
\newcommand{\cU}{\mathcal{U}} \newcommand{\cV}{\mathcal{V}}
\newcommand{\cW}{\mathcal{W}} \newcommand{\cX}{\mathcal{X}}
\newcommand{\cY}{\mathcal{Y}} \newcommand{\cZ}{\mathcal{Z}}

\newcommand\numberthis{\addtocounter{equation}{1}\tag{\theequation}}
%Page style
\pagestyle{fancy}

\listfiles

\raggedbottom

\rhead{William Justin Toth 671 Assignment 4} %CHANGE n to ASSIGNMENT NUMBER ijk TO COURSE CODE
\renewcommand{\headrulewidth}{1pt} %heading underlined
%\renewcommand{\baselinestretch}{1.2} % 1.2 line spacing for legibility (optional)

\begin{document}
\section*{21}
\paragraph{}
Let $||\cdot||$ be a norm on $\R^d$. Let $\cL = \{(x,t) \in \R^d \times \R: ||x|| \leq t\}$. Let $(x,s), (y,t) \in \cL$ and let $\alpha, \beta \in \R$ such that $\alpha, \beta \geq 0$. Then 
\begin{align*}
||\alpha x +\beta y|| &\leq ||\alpha x|| + ||\beta y|| &\text{by triangle inequality} \\
&= \alpha ||x|| + \beta ||y|| &\text{since $\alpha, \beta \geq 0$} \\
&\leq \alpha s + \beta t &\text{since $(x,s), (y,t) \in \cL$}
\end{align*}
and hence $(\alpha x, \alpha s) + (\beta y, \beta t)  = \alpha(x,s) + \beta(y,t)\in \cL$. Therefore for any norm $||\cdot ||$ on $\R^d$, $\cL$ is a convex cone. $\blacksquare$
\section*{22}
\paragraph{}
Let $\cC$ be a closed, bounded, centrally symmetric (that is, for all $x \in \cC, -x\in \cC$) convex set in $\R^d$. Let $\pi$ denote the metric projection map. For any $y \in \R^d$ let $\beta(y)$ denote the unique boundary point on the half-line through the origin and $y$. Let $\nu(y)$ be the scalar satisfying $\nu(y)\beta(y) = y$. We will show that $\nu(y)$ satisfies all three properties in the definition of a norm.
\paragraph{}
First we will show that for all $y \in \R^d$, $\nu(y) \geq 0$. Let $y \in \R^d$. Then $\beta(y)$ lies on the half-line from $0$ through $y$. So there exists a scalar $\alpha \geq 0$ such that $\beta(y) = \alpha (y-0) = \alpha y$. Then $\frac{1}{\alpha} = \nu(y)$ if $\alpha \neq 0$ and $\nu(y) = 0$ if $\alpha = 0$. In either case $\nu(y) \geq 0$ as desired.
\paragraph{}
Now we will show that for any scalar $\alpha$, for any $y \in \R^d$, $\nu(\alpha y) = |\alpha| \nu(y)$. Let $y \in \R^d$ and let $\alpha$ be a scalar. We have
\begin{equation}
\nu(\alpha y) \beta(\alpha y) = \alpha y = \alpha \nu(y) \beta(y) \label{eq:nudef}
\end{equation}
. We claim that \begin{equation} \beta(\alpha y) = \begin{cases} \beta(y) &\text{if $\alpha \geq 0$} \\
-\beta(y) &\text{if $\alpha <0$}\end{cases}\label{eq:betacases}\end{equation}
. If $\alpha \geq 0$ then $\alpha y$ lies on the half-line from $0$ through $y$ and hence $\beta(y)$ lies on the half-line from $0$ through $\alpha y$. By uniqueness, $\beta(y) = \beta(\alpha y)$.
\paragraph{}
If $\alpha < 0$ then $\alpha y$ lies on the reflection of the half-line from $0$ through $y$. That is to say, $\alpha y$ lies on the half-line from $0$ through $-y$. Hence by the previous argument, $\beta(\alpha y) = \beta(-y)$. To finish this case we must show that $\beta(-y) = -\beta(y)$. By symmetry of reflection, $-\beta(y)$ lies on the half-line through $0$ and $-y$. By the central symmetry of $\cC$, $-\beta(y) \in \cC$. Thus if $-\beta(y)$ is a boundary point of $\cC$ then by uniqueness we are done and $-\beta(y) = \beta(-y)$. But if $\beta(y)$ were an interior point of $\cC$ then the open ball about $-\beta(y)$ could be reflected to give an open ball about $\beta(y)$ which would be contained in $\cC$ by central symmetry.
\paragraph{}
Thus by (\ref{eq:nudef}) and (\ref{eq:betacases}) we have that
$$\nu(\alpha y) \beta(\alpha y) = \begin{cases}\alpha \nu(y) \beta(\alpha y) &\text{if $\alpha \geq 0$}\\
-\alpha \nu(y) \beta(\alpha y) &\text{if $\alpha < 0$} \end{cases}$$
and hence $\nu(\alpha y) = |\alpha| \nu(y)$ as desired.
\paragraph{}
Finally we will show that $\nu$ satisfies the triangle equality. Let $x,y\in \R^d$. If $\nu(x+y) = 0$ then $$\nu(x+y) = 0 \leq \nu(x) + \nu(y).$$ Hence we may assume that $\nu(x+y) \neq 0$. Let $$z = \frac{\nu(x)}{\nu(x)+ \nu(y)} \beta(x) + \frac{\nu(y)}{\nu(x) + \nu(y)} \beta(y)$$. Now by definition we have that
$$\nu(x+y) \beta(x+y) = x+y = \nu(x) \beta(x) + \nu(y) \beta(y).$$
So then $$\frac{1}{\nu(x) + \nu(y)}(x+y) = z.$$
Thus $z$ lies on the half-line from $0$ through $x+y$. Further $z$ is a convex combination of $\beta(x)$ and $\beta(y)$ and hence lies in $C$. Since $\beta(x+y)$ lies on the half-line from $0$ through $x+y$ and is also a boundary point of $C$ we conclude that $z$ lies on the line-segment joining $0$ and $\beta(x+y)$. Therefore there exists $\alpha$ such that $0\leq \alpha\leq 1$ and $$z = \alpha \beta(x+y)$$.
Now we observe that
\begin{align*}
\nu(x+y) \beta(x+y) &= \nu(x) \beta(x) + \nu(y)\beta(y)\\
 &= (\nu(x) + \nu(y))z \\
&= (\nu(x) + \nu(y)\alpha \beta(x+y).
\end{align*}
Therefore
$$\nu(x+y) = \alpha(\nu(x) + \nu(y)).$$
Since $\alpha \leq 1$ we conclude that $\nu(x+y) \leq \nu(x) + \nu(y)$ as desired. Hence $\nu$ satisfies the triangle equality. So $\nu$ satisfies the three properties defining a norm and is thus a norm. $\blacksquare$
\end{document}
