\documentclass[letterpaper,12pt,oneside,onecolumn]{article}
\usepackage[margin=1in, bottom=1in, top=1in]{geometry} %1 inch margins
\usepackage{amsmath, amssymb, amstext}
\usepackage{fancyhdr}
\usepackage{algorithm}
\usepackage{algpseudocode}
\usepackage{mathtools}

\DeclarePairedDelimiter{\ceil}{\lceil}{\rceil}
\DeclarePairedDelimiter\floor{\lfloor}{\rfloor}

%Macros
\newcommand{\A}{\mathbb{A}} \newcommand{\C}{\mathbb{C}}
\newcommand{\D}{\mathbb{D}} \newcommand{\F}{\mathbb{F}}
\newcommand{\N}{\mathbb{N}} \newcommand{\R}{\mathbb{R}}
\newcommand{\T}{\mathbb{T}} \newcommand{\Z}{\mathbb{Z}}
\newcommand{\Q}{\mathbb{Q}}
 
 
\newcommand{\cA}{\mathcal{A}} \newcommand{\cB}{\mathcal{B}}
\newcommand{\cC}{\mathcal{C}} \newcommand{\cD}{\mathcal{D}}
\newcommand{\cE}{\mathcal{E}} \newcommand{\cF}{\mathcal{F}}
\newcommand{\cG}{\mathcal{G}} \newcommand{\cH}{\mathcal{H}}
\newcommand{\cI}{\mathcal{I}} \newcommand{\cJ}{\mathcal{J}}
\newcommand{\cK}{\mathcal{K}} \newcommand{\cL}{\mathcal{L}}
\newcommand{\cM}{\mathcal{M}} \newcommand{\cN}{\mathcal{N}}
\newcommand{\cO}{\mathcal{O}} \newcommand{\cP}{\mathcal{P}}
\newcommand{\cQ}{\mathcal{Q}} \newcommand{\cR}{\mathcal{R}}
\newcommand{\cS}{\mathcal{S}} \newcommand{\cT}{\mathcal{T}}
\newcommand{\cU}{\mathcal{U}} \newcommand{\cV}{\mathcal{V}}
\newcommand{\cW}{\mathcal{W}} \newcommand{\cX}{\mathcal{X}}
\newcommand{\cY}{\mathcal{Y}} \newcommand{\cZ}{\mathcal{Z}}

\newcommand\numberthis{\addtocounter{equation}{1}\tag{\theequation}}

\DeclareMathOperator{\conv}{conv}
%Page style
\pagestyle{fancy}

\listfiles

\raggedbottom

\rhead{William Justin Toth 671 Review of Linear Programming Duality} %CHANGE n to ASSIGNMENT NUMBER ijk TO COURSE CODE
\renewcommand{\headrulewidth}{1pt} %heading underlined
%\renewcommand{\baselinestretch}{1.2} % 1.2 line spacing for legibility (optional)

\begin{document}
\paragraph{Definition}
We say that a linear program written in standard form
\begin{align*}
\max c^Tx \\
\text{s.t.} Ax &\leq b \\
x &\geq 0
\end{align*}
is the Primal problem. To each Primal problem their corresponds a Dual problem
\begin{align*}
\min b^T y\\
\text{s.t.} A^Ty &\geq c \\
y &\geq 0.
\end{align*}
The idea behind the Dual is to provide the tightest possible upper bound on the Primal objective function. The weak duality theorem states that Dual solutions do indeed bound Primal solutions from above.
\paragraph{Weak Duality Theorem}
Let $x$ be a solution to the Primal problem. Let $y$ be a solution to the Dual problem. Then $c^Tx \leq b^Ty$.
\paragraph{}
Infact the amazing thing about Linear Programming is that we can say something stronger: that optimal solutions to the primal and dual coincide. This is known as strong duality. Along the way to proving strong duality is the famous Farkas Lemma.
\paragraph{Farkas Lemma}
Let $A$ be an $m \times n$ matrix and $b \in \R^m$. Then exactly one of the following holds:
\begin{align}
\text{There exists $x\in \R^n$ such that $Ax=b$ and $x\geq 0$}\\
\text{There exists $y \in \R^m$ such that $y^TA \geq 0$ and $y^Tb < 0$.}
\end{align}
\paragraph{Strong Duality Theorem}
Let $x^*$ be an optimal solution to the Primal problem and let $y^*$ be an optimal solution to the Dual problem. Then $c^Tx^* = b^Ty^*$. Further if the Primal is infeasible then the Dual is unbounded and vice versa.
\paragraph{}
Lastly we mention the amazingly useful complementary slackness conditions which characterize optimal solutions.
\paragraph{Complementary Slackness Conditions}
Let $x \in \R^n$ be a feasible solution to the Primal problem and let $y\in \R^m$ be a feasible solution to the Dual problem. Let $A_i$ denote the $i^{th}$ row of matrix $A$ (and similarly for $A^T$). Then $x$ and $y$ are optimal if and only if
\begin{align*}
x_i(c_i - (A^T)_i y) &= 0 &\text{for all $i$ and }\\
y_j(b_j - A_j x) &= 0 &\text{for all $j$}.
\end{align*}
\end{document}
