\documentclass[letterpaper,12pt,oneside,onecolumn]{article}
\usepackage[margin=1in, bottom=1in, top=1in]{geometry} %1 inch margins
\usepackage{amsmath, amssymb, amstext}
\usepackage{fancyhdr}
\usepackage{algorithm}
\usepackage{algpseudocode}
\usepackage{mathtools}

\DeclarePairedDelimiter{\ceil}{\lceil}{\rceil}
\DeclarePairedDelimiter\floor{\lfloor}{\rfloor}

%Macros
\newcommand{\A}{\mathbb{A}} \newcommand{\C}{\mathbb{C}}
\newcommand{\D}{\mathbb{D}} \newcommand{\F}{\mathbb{F}}
\newcommand{\N}{\mathbb{N}} \newcommand{\R}{\mathbb{R}}
\newcommand{\T}{\mathbb{T}} \newcommand{\Z}{\mathbb{Z}}
\newcommand{\Q}{\mathbb{Q}}
 
 
\newcommand{\cA}{\mathcal{A}} \newcommand{\cB}{\mathcal{B}}
\newcommand{\cC}{\mathcal{C}} \newcommand{\cD}{\mathcal{D}}
\newcommand{\cE}{\mathcal{E}} \newcommand{\cF}{\mathcal{F}}
\newcommand{\cG}{\mathcal{G}} \newcommand{\cH}{\mathcal{H}}
\newcommand{\cI}{\mathcal{I}} \newcommand{\cJ}{\mathcal{J}}
\newcommand{\cK}{\mathcal{K}} \newcommand{\cL}{\mathcal{L}}
\newcommand{\cM}{\mathcal{M}} \newcommand{\cN}{\mathcal{N}}
\newcommand{\cO}{\mathcal{O}} \newcommand{\cP}{\mathcal{P}}
\newcommand{\cQ}{\mathcal{Q}} \newcommand{\cR}{\mathcal{R}}
\newcommand{\cS}{\mathcal{S}} \newcommand{\cT}{\mathcal{T}}
\newcommand{\cU}{\mathcal{U}} \newcommand{\cV}{\mathcal{V}}
\newcommand{\cW}{\mathcal{W}} \newcommand{\cX}{\mathcal{X}}
\newcommand{\cY}{\mathcal{Y}} \newcommand{\cZ}{\mathcal{Z}}

\newcommand\numberthis{\addtocounter{equation}{1}\tag{\theequation}}
%Page style
\pagestyle{fancy}

\listfiles

\raggedbottom

\rhead{William Justin Toth 671 Assignment 6 - Problem Set 2} %CHANGE n to ASSIGNMENT NUMBER ijk TO COURSE CODE
\renewcommand{\headrulewidth}{1pt} %heading underlined
%\renewcommand{\baselinestretch}{1.2} % 1.2 line spacing for legibility (optional)

\begin{document}
\section*{1}
\paragraph{}
Let $K$ be a convex cone. Let $B$ be a compact base for $K$. Let $$\hat{B} = \{\hat{b} : \hat{b} = \frac{b}{||b||} \text{ and } b \in B \}.$$
We claim $\hat{B}$ is a base for $K$. It is enough to show that there is a bijection between $B$ and $\bar{B}$. Let $f: B \rightarrow \hat{B}$ given by
$$f(b) = \frac{b}{||b||}.$$
This $f$ is well defined since $0 \not\in B$ and hence $||b|| \neq 0$ for all $b \in B$.
\paragraph{}
We demonstrate $f$ is surjective. Let $\hat{b} \in \hat{B}$. Since $B$ is a base there exists $b \in B$ and scalar $\alpha > 0$ such that $$\hat{b} = \alpha b.$$
Thus $$1 = ||\hat{b}|| = \alpha ||b||$$
and so $$|| b || = \frac{1}{\alpha}.$$
Therefore $$f(b) = \frac{b}{||b||} = \alpha b = \hat{b}$$
and hence $f$ is surjective.
\paragraph{}
Now we demonstrate that $f$ is injective. Let $b_1, b_2 \in B$. Suppose that
$$f(b_1) = f(b_2) = \hat{b}.$$
Now $||\hat{b}|| = 1$ so we have $$b_1 = ||b_1|| \frac{b_2}{||b_2||}.$$
So $b_1$ lies on the ray $b_2$. Since each ray in $K$ has a single point in $B$ we thus have $b_1 = b_2$. So we may conclude that $f$ is bijective.
\paragraph{}
So $\hat{B}$ is a base for $K$. We claim that $\hat{B}$ is compact. Let $\hat{\cC}$ be a cover of $\hat{B}$. Since $0 \not \in \hat{B}$ we may assume that each open set in $\hat{\cC}$ does not contain $0$. For each $\hat{b} \in \hat{B}$ choose $C_{\hat{b}} \in \cC$ such that $\hat{b} \in C_{\hat{b}}$. Since $\cC$ covers $\hat{B}$ we may choose such $C_{\hat{b}}$. Each $\hat{b} \in \hat{B}$ is identified with some $b \in B$ by $f$.  Define $C_b$ by
$$ C_b = C_{\hat{b}} +\{b -\hat{b}\} \quad \text{such that $f(b) = \hat{b}$}.$$
That is $C_b$ is a linear translation of $C_{\hat{b}}$ sending $\hat{b}$ to $b$. Since this transformation is linear it is continuous. So since each $C_{\hat{b}}$ is open, each $C_b$ is open. Thus we have an open cover of $B$ given by:
$$\bigcup_{b\in B} C_b.$$
Since $B$ is compact this open cover contains a finite subcover. Let $I$ be the finite subset of $B$ indexing said finite subcover. That is,
$$B \subseteq \bigcup_{b\in I} C_b.$$
Let $\hat{I} = \{f(b) \in \hat{B}: b \in I\}$. We claim that $$\hat{B} \subseteq \bigcup_{\hat{b} \in I} C_{\hat{b}}.$$
Let $\hat{b} \in \hat{B}$. Then there exists $b\in B$ such that $\hat{b} = f(b)$. Since $I$ covers $B$ there exists $c \in I$ such that $b \in C_c$. Then by translation we have
$$\hat{b} \in C_{\hat{c}}$$
where $\hat{c} = f(c) \in \hat{I}$. Thus indeed $$\bigcup_{\hat{b} \in \hat{I}} C_{\hat{b}}$$ is a finite subcover of $\cC$ covering $\hat{B}$. That is, $\hat{B}$ is compact.
\paragraph{}
So now we have a compact base of $K$ consisting entirely of unit vectors. We claim this base $\hat{B}$ has no isolated points.
\end{document}
