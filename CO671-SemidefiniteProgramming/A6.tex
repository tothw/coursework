\documentclass[letterpaper,12pt,oneside,onecolumn]{article}
\usepackage[margin=1in, bottom=1in, top=1in]{geometry} %1 inch margins
\usepackage{amsmath, amssymb, amstext}
\usepackage{fancyhdr}
\usepackage{algorithm}
\usepackage{algpseudocode}
\usepackage{mathtools}

\DeclarePairedDelimiter{\ceil}{\lceil}{\rceil}
\DeclarePairedDelimiter\floor{\lfloor}{\rfloor}

%Macros
\newcommand{\A}{\mathbb{A}} \newcommand{\C}{\mathbb{C}}
\newcommand{\D}{\mathbb{D}} \newcommand{\F}{\mathbb{F}}
\newcommand{\N}{\mathbb{N}} \newcommand{\R}{\mathbb{R}}
\newcommand{\T}{\mathbb{T}} \newcommand{\Z}{\mathbb{Z}}
\newcommand{\Q}{\mathbb{Q}}
 
 
\newcommand{\cA}{\mathcal{A}} \newcommand{\cB}{\mathcal{B}}
\newcommand{\cC}{\mathcal{C}} \newcommand{\cD}{\mathcal{D}}
\newcommand{\cE}{\mathcal{E}} \newcommand{\cF}{\mathcal{F}}
\newcommand{\cG}{\mathcal{G}} \newcommand{\cH}{\mathcal{H}}
\newcommand{\cI}{\mathcal{I}} \newcommand{\cJ}{\mathcal{J}}
\newcommand{\cK}{\mathcal{K}} \newcommand{\cL}{\mathcal{L}}
\newcommand{\cM}{\mathcal{M}} \newcommand{\cN}{\mathcal{N}}
\newcommand{\cO}{\mathcal{O}} \newcommand{\cP}{\mathcal{P}}
\newcommand{\cQ}{\mathcal{Q}} \newcommand{\cR}{\mathcal{R}}
\newcommand{\cS}{\mathcal{S}} \newcommand{\cT}{\mathcal{T}}
\newcommand{\cU}{\mathcal{U}} \newcommand{\cV}{\mathcal{V}}
\newcommand{\cW}{\mathcal{W}} \newcommand{\cX}{\mathcal{X}}
\newcommand{\cY}{\mathcal{Y}} \newcommand{\cZ}{\mathcal{Z}}

\newcommand\numberthis{\addtocounter{equation}{1}\tag{\theequation}}
%Page style
\pagestyle{fancy}

\listfiles

\raggedbottom

\rhead{William Justin Toth 671 Assignment 6 - Problem Set 2} %CHANGE n to ASSIGNMENT NUMBER ijk TO COURSE CODE
\renewcommand{\headrulewidth}{1pt} %heading underlined
%\renewcommand{\baselinestretch}{1.2} % 1.2 line spacing for legibility (optional)

\begin{document}
\section*{1}
\paragraph{}
Let $K$ be a convex cone. Let $B$ be a compact base for $K$.  We will use a basic result in Analysis to show that $K$ is closed:
\begin{equation}
\text{A set $C$ is closed $\iff$ every convergent sequence in $C$ converges to a point in $C$}
\label{closed:sequence}
\end{equation}
Let $(x)_n \subseteq K$ be a convergent sequence in $K$. Then for every $x_n$ there exists $b_n \in B$ and $\alpha_n > 0$ such that
$$x_n = \alpha_n b_n.$$
Thus we have a sequence $(b)_n \subseteq B$ corresponding to the sequence $(x)_n$. Since $B$ is compact this sequence $(b)_n$ has a convergent subsequence $$(b)_{n_i}$$ that converges to $b \in B$.
Since $(x)_n$ converges, the subsequence $$(x)_{n_i}$$ corresponding to $(b)_{n_i}$ converges. Let $x$ be the limit point of $(x)_{n_i}$. Let $\alpha \geq 0$ be given by
$$\alpha = \lim_{n_i \rightarrow \infty} \frac{||x_{n_i}||}{||b_{n_i}||} = \frac{||x||}{||b||}$$
Since $b\in B$, $b\neq 0$ and thus $\alpha$ is well-defined. Observe that the sequence $(\alpha)_{n_i}$ of scalars satisfying
$$x_{n_i} = \alpha_{n_i} b_{n_i}$$
\paragraph{}
We claim that $(x)_n$ converges to $\alpha b$. Let $\epsilon > 0$.
\end{document}
