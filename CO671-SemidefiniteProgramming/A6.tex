\documentclass[letterpaper,12pt,oneside,onecolumn]{article}
\usepackage[margin=1in, bottom=1in, top=1in]{geometry} %1 inch margins
\usepackage{amsmath, amssymb, amstext}
\usepackage{fancyhdr}
\usepackage{algorithm}
\usepackage{algpseudocode}
\usepackage{mathtools}

\DeclarePairedDelimiter{\ceil}{\lceil}{\rceil}
\DeclarePairedDelimiter\floor{\lfloor}{\rfloor}

%Macros
\newcommand{\A}{\mathbb{A}} \newcommand{\C}{\mathbb{C}}
\newcommand{\D}{\mathbb{D}} \newcommand{\F}{\mathbb{F}}
\newcommand{\N}{\mathbb{N}} \newcommand{\R}{\mathbb{R}}
\newcommand{\T}{\mathbb{T}} \newcommand{\Z}{\mathbb{Z}}
\newcommand{\Q}{\mathbb{Q}}
 
 
\newcommand{\cA}{\mathcal{A}} \newcommand{\cB}{\mathcal{B}}
\newcommand{\cC}{\mathcal{C}} \newcommand{\cD}{\mathcal{D}}
\newcommand{\cE}{\mathcal{E}} \newcommand{\cF}{\mathcal{F}}
\newcommand{\cG}{\mathcal{G}} \newcommand{\cH}{\mathcal{H}}
\newcommand{\cI}{\mathcal{I}} \newcommand{\cJ}{\mathcal{J}}
\newcommand{\cK}{\mathcal{K}} \newcommand{\cL}{\mathcal{L}}
\newcommand{\cM}{\mathcal{M}} \newcommand{\cN}{\mathcal{N}}
\newcommand{\cO}{\mathcal{O}} \newcommand{\cP}{\mathcal{P}}
\newcommand{\cQ}{\mathcal{Q}} \newcommand{\cR}{\mathcal{R}}
\newcommand{\cS}{\mathcal{S}} \newcommand{\cT}{\mathcal{T}}
\newcommand{\cU}{\mathcal{U}} \newcommand{\cV}{\mathcal{V}}
\newcommand{\cW}{\mathcal{W}} \newcommand{\cX}{\mathcal{X}}
\newcommand{\cY}{\mathcal{Y}} \newcommand{\cZ}{\mathcal{Z}}

\newcommand\numberthis{\addtocounter{equation}{1}\tag{\theequation}}
%Page style
\pagestyle{fancy}

\listfiles

\raggedbottom

\rhead{William Justin Toth 671 Assignment 6 - Problem Set 2} %CHANGE n to ASSIGNMENT NUMBER ijk TO COURSE CODE
\renewcommand{\headrulewidth}{1pt} %heading underlined
%\renewcommand{\baselinestretch}{1.2} % 1.2 line spacing for legibility (optional)

\begin{document}
\section*{1}
\paragraph{}
Let $K$ be a convex cone. Let $B$ be a compact base for $K$.  We will use a basic result in Analysis to show that $K$ is closed:
$$
\text{A set $C$ is closed $\iff$ every convergent sequence in $C$ converges to a point in $C$.}
$$
Let $(x)_n \subseteq K$ be a convergent sequence in $K$. Let $x$ be the limit point of $(x)_n$. Then for every $x_n$ there exists $b_n \in B$ and $\alpha_n > 0$ such that
$$x_n = \alpha_n b_n.$$
Thus we have a sequence $(b)_n \subseteq B$ corresponding to the sequence $(x)_n$. Since $B$ is compact this sequence $(b)_n$ has a convergent subsequence $$(b)_{n_i}$$ that converges to $b \in B$.
Since $(x)_n$ converges, the subsequence $$(x)_{n_i}$$ corresponding to $(b)_{n_i}$ converges and its limit is also $x$. We also have a sequence of $(\alpha)_{n_i}$ satisfying
$$x_{n_i} = \alpha_{n_i} b_{n_i}$$
for all $n_i$.
Taking norms and dividing by $||b_{n_i}||$ we obtain
$$\alpha_{n_i} = \frac{||x_{n_i}||}{||b_{n_i}||}.$$
Since each $b_{n_i}$ is not $0$ this is well-defined. The norm $||\cdot||$ is a continuous map, so since $(x)_{n_i}$ and $(b)_{n_i}$ are convergent, the sequences of norms
$$||x_{n_i}|| \quad \text{and}\quad ||b_{n_i}||$$
are convergent. Then $(\alpha)_{n_i}$ is the quotient of two convergent sequences and this therefore continuous. Further since $b \in B$, $b \neq 0 $ and thus the limit of $\alpha_{n_i}$ is just the limit of quotients:
$$\lim_{n_i \rightarrow \infty} \alpha_{n_i} = \lim_{n_i \rightarrow \infty} \frac{||x_{n_i}||}{||b_{n_i}||} = \frac{||x||}{||b||}.$$
Let $\alpha$ denote $\frac{||x||}{||b||}$. Then $(\alpha)_{n_i}$ is convergent and converges to $\alpha$. Further $\alpha \geq 0$.
\paragraph{}
But now we have that
$$x_{n_i} = \alpha_{n_i} b_{n_i}.$$
That is, $x_{n_i}$ is the product of two convergent sequences and hence converges to the product of their limits. Thus
$$\lim_{n_i \rightarrow \infty} x_{n_i} = \lim_{n_i \rightarrow \infty} \alpha_{n_i} b_{n_i} =\alpha b.$$
Since $\alpha \geq 0$ and $b \in K$, $\alpha b \in K$. Therefore the subsequence $(x)_{n_i}$ of $(x)_n$ converges to a point in $K$. Since convergent sequences and their convergent subsequences has the same limit, $(x)_n$ converges to a point in $K$. Therefore $K$ is closed. $\blacksquare$
\end{document}
