\documentclass[letterpaper,12pt,oneside,onecolumn]{article}
\usepackage[margin=1in, bottom=1in, top=1in]{geometry} %1 inch margins
\usepackage{amsmath, amssymb, amstext}
\usepackage{fancyhdr}
\usepackage{algorithm}
\usepackage{algpseudocode}
\usepackage{mathtools}
\usepackage{bbm}

\DeclarePairedDelimiter{\ceil}{\lceil}{\rceil}
\DeclarePairedDelimiter\floor{\lfloor}{\rfloor}

%Macros
\newcommand{\A}{\mathbb{A}} \newcommand{\C}{\mathbb{C}}
\newcommand{\D}{\mathbb{D}} \newcommand{\F}{\mathbb{F}}
\newcommand{\N}{\mathbb{N}} \newcommand{\R}{\mathbb{R}}
\newcommand{\T}{\mathbb{T}} \newcommand{\Z}{\mathbb{Z}}
\newcommand{\Q}{\mathbb{Q}}
 
\newcommand{\1}{\mathbbm{1}} 

\newcommand{\cA}{\mathcal{A}} \newcommand{\cB}{\mathcal{B}}
\newcommand{\cC}{\mathcal{C}} \newcommand{\cD}{\mathcal{D}}
\newcommand{\cE}{\mathcal{E}} \newcommand{\cF}{\mathcal{F}}
\newcommand{\cG}{\mathcal{G}} \newcommand{\cH}{\mathcal{H}}
\newcommand{\cI}{\mathcal{I}} \newcommand{\cJ}{\mathcal{J}}
\newcommand{\cK}{\mathcal{K}} \newcommand{\cL}{\mathcal{L}}
\newcommand{\cM}{\mathcal{M}} \newcommand{\cN}{\mathcal{N}}
\newcommand{\cO}{\mathcal{O}} \newcommand{\cP}{\mathcal{P}}
\newcommand{\cQ}{\mathcal{Q}} \newcommand{\cR}{\mathcal{R}}
\newcommand{\cS}{\mathcal{S}} \newcommand{\cT}{\mathcal{T}}
\newcommand{\cU}{\mathcal{U}} \newcommand{\cV}{\mathcal{V}}
\newcommand{\cW}{\mathcal{W}} \newcommand{\cX}{\mathcal{X}}
\newcommand{\cY}{\mathcal{Y}} \newcommand{\cZ}{\mathcal{Z}}

\newcommand\numberthis{\addtocounter{equation}{1}\tag{\theequation}}
%Page style
\pagestyle{fancy}

\listfiles

\raggedbottom

\rhead{William Justin Toth 671 Assignment 7 - Problem Set 2} %CHANGE n to ASSIGNMENT NUMBER ijk TO COURSE CODE
\renewcommand{\headrulewidth}{1pt} %heading underlined
%\renewcommand{\baselinestretch}{1.2} % 1.2 line spacing for legibility (optional)

\begin{document}
\section*{7}
\paragraph{}
Let $A$ be the adjacency matrix of the graph $G$.  Let $\1$ denote the all ones vector. Let $a^T$ be a row of $A$ corresponding to vertex $v \in V(G)$. Let $d(v)$ denote the degree of vertex $v$. Then
$$a^T\1 = \sum_{u \sim v} 1 = d(v).$$
So labelling the vertices of $G$ as $1, \dots, n$ we have
$$A\1 = \begin{bmatrix} d(1) \\ \vdots \\ d(n)\end{bmatrix}.$$
So $G$ is a $k$-regular graph if and only if $d(v) = k$ for all $v \in V(G)$ which holds if and only if
$$ A\1 = \begin{bmatrix} d(1) \\ \vdots \\ d(n)\end{bmatrix} = \begin{bmatrix} k \\ \vdots \\ k \end{bmatrix} = k \1.$$
That is, $G$ is $k$-regular if and only if $\1$ is an eigenvector of $A$ which eigenvalue $k$, the valency of $G$.
\paragraph{}
Now suppose $G$ is a $k$-regular graph on $v$ vertices with adjacency matrix $A$. Suppose that $\tau$ is the least eigenvalue of $A$. We will show that the matrix
\begin{equation}
A - \tau I - \frac{k - \tau}{v} J \label{matrix:AIJ}
\end{equation}
is positive semidefinite by showing that all its eigenvalues are non-negative.
\paragraph{}
First we claim that $\1$ is an eigenvector of (\ref{matrix:AIJ}):
$$(A - \tau I - \frac{k - \tau}{v} J)\1 = A\1 - \tau \1 - \frac{k-\tau}{v}v\1 = (k-\tau - (k-\tau))\1 = 0 \1 .$$
Thus $\1$ is an eigenvector of (\ref{matrix:AIJ}) with eigenvalue $0$. It remains to show that all other eigenvalues of (\ref{matrix:AIJ}) are non-negative. 
\paragraph{}
Let $\lambda$ be an eigenvalue of (\ref{matrix:AIJ}) distinct from $0$. Since (\ref{matrix:AIJ}) is symmetric we may assume there is an eigenvector associated with $\lambda$, call it $x$, orthogonal to $\1$. We have:
\begin{align*}
\lambda x &= (A - \tau I - \frac{k - \tau}{v} J)x \\
&= Ax - \tau x - \frac{k-\tau}{v} \1\1^T x \\
&= Ax - \tau x - 0 &\text{ since $\1^T x = 0$}. \\
\end{align*}
Therefore
$$(\lambda+\tau)x = A x. $$
That is, $\lambda + \tau$ is an eigenvalue of $A$. Since $\tau$ is the least eigenvalue of $A$, 
$$\lambda + \tau \geq \tau$$
and hence
$$\lambda \geq 0.$$
So all eigenvalues of (\ref{matrix:AIJ}) are non-negative. Therefore (\ref{matrix:AIJ}) is positive semidefinite. $\blacksquare$
\section*{8}
\paragraph{}
Suppose that $A$ is the adjacency matrix of a $k$-regular graph ($k > 0$) $G$ with least eigenvalue $\tau$. Let $x$ be the characteristic vector of a coclique in $G$. Then
\begin{align*}
\langle xx^T, A - \tau I - \frac{k - \tau}{v} J \rangle &= tr(xx^T(A - \tau I - \frac{k - \tau}{v} J)) \\
&= x^T(A - \tau I - \frac{k - \tau}{v} J)x \\
&\geq 0 &\text{As, by problem $7$, $(A - \tau I - \frac{k - \tau}{v} J) \succcurlyeq 0$}.
\end{align*}
\paragraph{}
Now let $x$ be the characteristic vector of the largest coclique in $G$. Then 
$$\sum_{i} x_i = \alpha(G).$$
Since $x^TIx = \sum_{i} x_i^2 = \sum_{i} x_i$ we have
$$x^TIx = \alpha(G).$$
Also $\1^Tx = \sum_{i} x_i.$ and hence
$$x^TJx = (\1^Tx)^2 = \alpha(G)^2.$$
Now for a row in $A$, call it $a^T$, which corresponds to vertex $v \in V(G)$,
$$a^Tx = \sum_{u: u\sim v} x_u.$$
So then
$$x^T A x = \sum_{v \in V(G)} \sum_{u:u\sim v} x_v x_u = 0.$$
The last equality follows since $x$ is the characteristic vector of a coclique and so no adjacent vertices are both indexing a $1$ in $x$.
Thus we have
\begin{align*}
0 &\leq \langle xx^T, A - \tau I - \frac{k - \tau}{v} J \rangle \\
&= \langle xx^TA\rangle - \tau \langle xx^T, I \rangle - \frac{k - \tau}{v} \langle xx^T, J \rangle \\
&= x^TAx - \tau x^TIx -\frac{k - \tau}{v} x^TJ x \\
&= 0 - \tau \alpha(G) - \frac{k -\tau}{v} \alpha (G)^2. \\
&= \alpha(G)(-\tau - \frac{k-\tau}{v} \alpha(G)).
\end{align*}
Now by definition $\alpha(G) \geq 1$ so since 
$$ 0 \leq \alpha(G)(-\tau - \frac{k-\tau}{v} \alpha(G))$$
we have
\begin{equation}
0 \leq -\tau - \frac{k-\tau}{v} \alpha(G) \label{ineq:alpha}\end{equation}
\paragraph{}
Before rearranging (\ref{ineq:alpha}) to finish the proof, we need to understand the relationship between $\tau$ and $k$. We claim that $k$ is the maximum eigenvalue of $A$. We have previously shown that $k$ is an eigenvalue of $A$. To see that it is maximal let $\lambda$ be an eigenvalue of $A$ with eigenvector $y$. Let $y_i$ be the element of $y$ with largest absolute value. That is, $|y_i| \geq |y_j|$ for all $j$. Then
$$ |\lambda| |y_i| = |(Ay)_i| = |\sum_{j:\ j \sim i} y_j| = |d(i) y_j| = |k y_j| \leq k|y_i|$$
so $\lambda \leq k$ as desired.
\paragraph{}
Therefore $k \geq \tau$. If $k = \tau$ then the largest and smallest eigenvaule of $A$ are equal, so $A$ has one eigenvalue. Since $A$ is real symmetric, $A$ is diagonalizable as
$$A = Q^T k I Q = kQ^TQ = kI$$
for some orthogonal matrix $Q$. But as the adjacency matrix of a graph this is nonsense unless $k = 0$. But $k > 0$, a contradiction. So we may assume $k > \tau$.
\paragraph{}
Since $k > 0$, $\tau - k < 0$ so when rearranging (\ref{:ineq:alpha}) the direction of the inequality will flip. We obtain:
$$\alpha(G) \leq \tau\frac{v}{\tau-k} = \frac{v}{1-\frac{k}{\tau}}$$
as desired. $\blacksquare$
\section*{9}
\paragraph{}
Let $G$ and $H$ be graphs. Then $\overline{G} \boxtimes \overline{H}$ has the same vertex set as $\overline{G\boxtimes H}$ so to show that $\overline{G} \boxtimes \overline{H}$ is a spanning subgraph of $\overline{G\boxtimes H}$ it remains to show that $E(\overline{G} \boxtimes \overline{H})\subseteq E(\overline{G\boxtimes H})$.
\paragraph{}
Let $(u,x) (v,y) \in E(\overline{G} \boxtimes \overline{H})$. Then by definition of graph complement and strong product we have three cases, either:
\begin{enumerate}
\item $u = v$ and $x \not\sim y$ or
\item $u \not\sim v$ and $x = y$ or
\item $u \not\sim v$ and $x \not\sim y$.
\end{enumerate}
Since $u = v$ implies $u \not \sim v$ and $x = y$ implies $x \not \sim y$ we need only consider case $3$. Now for $(u,x)(v,y) \in E(\overline{G\boxtimes H})$ three conditions must be satisfied:
\begin{equation} u \neq v \text{ or } x\not\sim y \label{cond:1}\end{equation}
and
\begin{equation} u \not\sim v \text{ or } x \neq y \label{cond:2}\end{equation}
and
\begin{equation} u \not\sim v \text{ or } x\not\sim y.\label{cond:3}\end{equation}
\paragraph{}
Since we are in case $3$, $u\not\sim v$ and thus (\ref{cond:2}) and (\ref{cond:3}) are satisfied. Further since $x \not\sim y$ (\ref{cond:1}) is satisfied. Therefore $(u,x)(v,y) \in E(\overline{G\boxtimes H})$. Hence $\overline{G} \boxtimes \overline{H}$ is a spanning subgraph of $\overline{G\boxtimes H}$. $\blacksquare$
\section*{12}
\paragraph{}
Let $F_{ij}$ denote the matrix with entry $ij$ and $ji$ equal to $1$ and all other entries $0$. Let $\cA = \{ (i,j) : i,j \in \{1, \dots, n\}, \text{ and $i=j$ or $i\sim j$}\}$. The primal program for $\theta(G)$ is:
\begin{align*}
\max\quad \langle J, N \rangle&\ \\
\ni\quad \langle I, N \rangle &= 1 \\
\langle F_{ij}, N \rangle &= 0 &\text{$(i,j) \in \cA$}\\
N &\succcurlyeq 0.
\end{align*}
\paragraph{} 
Let $y_0 \in \R$ and $\bar{y} \in \R^{|\cA|}$. We optimize the dual over $y = \begin{bmatrix} y_0 \\ \bar{y} \end{bmatrix}$. The dual of our primal program is:
\begin{align*}
&\min\quad y_0&\ \\
&\ni\quad  y_0+ \sum_{(i,j) \in \cA} y_{ij}F_{ij} - J \succcurlyeq 0.
\end{align*}
\section*{13}
\paragraph{}
The primal problem from the last question of List $1$ was:
\begin{align*}
\max\ &\langle A, X \rangle \\
\ni\ &\langle I, X\rangle = 1\\
&X \succcurlyeq 0
\end{align*}
\paragraph{}
The dual program ($y \in \R$) is :
\begin{align*}
\min\ &y \\
\ni\ &yI - A \succcurlyeq 0
\end{align*}
\paragraph{}
Complementary Slackness: Let $y^*$ be dual feasible and $X^*$ be primal feasible. Then $y^*$ is dual optimal and $X^*$ is primal optimal if and only if
$$ 0 = \langle y^*I - A,X^*\rangle.$$
\section*{14}
\paragraph{}
\end{document}
