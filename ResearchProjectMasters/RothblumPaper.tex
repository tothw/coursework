\documentclass[letterpaper,12pt,oneside,onecolumn]{article}
\usepackage[margin=1in, bottom=1in, top=1in]{geometry} %1 inch margins
\usepackage{amsmath, amssymb, amstext}
\usepackage{fancyhdr}
\usepackage{algorithm}
\usepackage{algpseudocode}
\usepackage{mathtools}

\DeclarePairedDelimiter{\ceil}{\lceil}{\rceil}
\DeclarePairedDelimiter\floor{\lfloor}{\rfloor}

%Macros
\newcommand{\A}{\mathbb{A}} \newcommand{\C}{\mathbb{C}}
\newcommand{\D}{\mathbb{D}} \newcommand{\F}{\mathbb{F}}
\newcommand{\N}{\mathbb{N}} \newcommand{\R}{\mathbb{R}}
\newcommand{\T}{\mathbb{T}} \newcommand{\Z}{\mathbb{Z}}
\newcommand{\Q}{\mathbb{Q}}
 
 
\newcommand{\cA}{\mathcal{A}} \newcommand{\cB}{\mathcal{B}}
\newcommand{\cC}{\mathcal{C}} \newcommand{\cD}{\mathcal{D}}
\newcommand{\cE}{\mathcal{E}} \newcommand{\cF}{\mathcal{F}}
\newcommand{\cG}{\mathcal{G}} \newcommand{\cH}{\mathcal{H}}
\newcommand{\cI}{\mathcal{I}} \newcommand{\cJ}{\mathcal{J}}
\newcommand{\cK}{\mathcal{K}} \newcommand{\cL}{\mathcal{L}}
\newcommand{\cM}{\mathcal{M}} \newcommand{\cN}{\mathcal{N}}
\newcommand{\cO}{\mathcal{O}} \newcommand{\cP}{\mathcal{P}}
\newcommand{\cQ}{\mathcal{Q}} \newcommand{\cR}{\mathcal{R}}
\newcommand{\cS}{\mathcal{S}} \newcommand{\cT}{\mathcal{T}}
\newcommand{\cU}{\mathcal{U}} \newcommand{\cV}{\mathcal{V}}
\newcommand{\cW}{\mathcal{W}} \newcommand{\cX}{\mathcal{X}}
\newcommand{\cY}{\mathcal{Y}} \newcommand{\cZ}{\mathcal{Z}}

%Page style
\pagestyle{fancy}

\listfiles

\raggedbottom

\rhead{William Justin Toth : Rothblum Paper Notes} %CHANGE n to ASSIGNMENT NUMBER ijk TO COURSE CODE
\renewcommand{\headrulewidth}{1pt} %heading underlined
%\renewcommand{\baselinestretch}{1.2} % 1.2 line spacing for legibility (optional)

\begin{document}
\section{Stable Marriage Polytope}
\paragraph{}
A matching is given by a partial assignment matrix $x = \{x_{ij}\}_{i\in M, j\in W}$ of integers which satisfies:
\begin{align}
\sum_{j \in W} x_{ij} &\leq 1 &\text{ for all $i \in M$}\\
\sum_{i \in M} x_{ij} &\leq 1 &\text{ for all $j \in W$}\\
x_{ij} &\geq 0 &\text{for all $i \in M$ and $j \in W$}.
\end{align}
\paragraph{}
Let $A \subseteq (M \times W)$ be a set of acceptable pairs. A matching is said to be stable if it satisfies:
\begin{align}
x_ij &= 0 &\text{for all $(i,j) \in (M\times W)\backslash A$}
\end{align}
and that there exists no pair $(m,w) \in A$ for which any of:
\begin{enumerate}
\item Both $m$ and $w$ have mates and $w(x,m) <_m w$ and $m(x,w) <_w m$ \\
\item $m$ has a mate but not $w$ and $w(x,m) <_m w$\\
\item $w$ has a mate but not $m$ and $m(x,w) <_w m$\\
\item Both $m$ and $w$ have no mates
\end{enumerate}
hold.
\paragraph{Lemma 1}
Let $x$ be a matching. Then $x$ is stable if and only if $x$ satisfies $(4)$ and the following:
\begin{align}
\sum_{j>_m w} x_{mj} + \sum_{i >_w m} x_{iw} + x_{mw} &\geq 1 &\text{for all $(m,w)\in A$}.
\end{align}
\paragraph{Proof Sketch}
Provided that $x$ is integral, the set of inequalities in $(5)$ is violated if and only if there exists $(m,w) \in A$ such that $$ \sum_{j >_m w} x_{mj} = \sum_{i >_w m} x_{iw} = x_{mj} = 0. $$ From which we derive the precise four conditions which define stability and hence stability is equivalent to $(4)$ and $(5)$.$\blacksquare$
\section{Conflicting Interests Lemma}
\paragraph{}
Before stating the lemma we need some definitions. Define $$ S_M(x) = \{m\in M : \sum_{j\in W} x_{mj} > 0\}$$ and similarly $$S_W(x) = \{w\in W: \sum_{i \in M} x_{iw} > 0\}.$$ For $m \in S_M(x)$ let $w^*(x,m)$, and $w_*(x,m)$ be the most preferred and least preferred elements respectively of the set $\{ j \in W: x_{mj}>0\}$. Similarly for $w \in S_W(x)$ let $m^*(x,w)$ and $m_*(x,w)$ be the most preferred and least preferred elements respectively of the set $\{i \in M : x_{iw} > 0\}$.
\paragraph{}
Note that $x$ is integral if and only if for all $m \in S_M(x)$, $\sum_{j \in W} x_{mj}=1$ and $w^*(x,m) = w_*(x,m)$. Or equivalently, $x$ is integral if and only if for all $w \in S_W(x)$, $\sum_{i\in M} x_{iw} = 1$ and $m^*(x,w) = m_*(x,w)$.
\paragraph{Lemma 2}
Let $x = \{x_{ij}\}_{i\in M, j\in W}$ satisfy $(1) - (5)$ and let $(m,w) \in A$. Then 
\begin{align}
m\not\in S_M(x) \text{ or } (m\in S_M(x) \text{ and } w \geq_{m} w^*(x,w)) \nonumber\\\implies \sum_{i\in M} x_{iw} = 1 \text{ and } m \leq_w m_*(x,w),
\end{align}
and
\begin{align}
m\in S_M(x) \text{ and } w = w^*(x,m) \iff \sum_{i \in M} x_{iw} = 1 \text{ and } m = m_*(x,w).
\end{align}
Further, if the corresponding inequality from $(5)$ is tight for $x_{mw}$, that is 
\begin{align}
\sum_{j >_m w} x_{mj} + \sum_{i >_w m} x_{iw} + x_{mw} = 1,
\end{align}
then the converse of $(6)$ holds as well. Also 
\begin{align}
m \in S_M(x) \iff \sum_{j \in W} x_{mj} = 1.
\end{align}
\paragraph{Proof Sketch}
Starting with $(6)$, notice that in either case $\sum_{j >_m w} x_{mj} = 0$. Thus by $(5)$
\begin{align*}
1 &\leq \sum_{j >_m w} x_{mj} + x_{mj} \\
&\leq \sum_{i \in M} x_{iw} \\
&\leq 1 &\text{by $(2)$}.
\end{align*}
So every inequality holds with equality. The third equality gives part of the conclusion. Also since $\sum_{j >_m w} x_{mj} + x_{mj}=1$, by $(2)$, $x_{iw} = 0$ for all $i \in M_w$ with $i <_w m$. From this we conclude that $m \leq_w m_*(x,w)$ as otherwise there exists $i$ such that $x_{iw} > 0$.
\paragraph{}

\end{document}
