\documentclass[letterpaper,12pt,oneside,onecolumn]{article}
\usepackage[margin=1in, bottom=1in, top=1in]{geometry} %1 inch margins
\usepackage{amsmath, amssymb, amstext}
\usepackage{fancyhdr}
\usepackage{algorithm}
\usepackage{algpseudocode}
\usepackage{mathtools}

\DeclarePairedDelimiter{\ceil}{\lceil}{\rceil}
\DeclarePairedDelimiter\floor{\lfloor}{\rfloor}

%Macros
\newcommand{\A}{\mathbb{A}} \newcommand{\C}{\mathbb{C}}
\newcommand{\D}{\mathbb{D}} \newcommand{\F}{\mathbb{F}}
\newcommand{\N}{\mathbb{N}} \newcommand{\R}{\mathbb{R}}
\newcommand{\T}{\mathbb{T}} \newcommand{\Z}{\mathbb{Z}}
\newcommand{\Q}{\mathbb{Q}}
 
 
\newcommand{\cA}{\mathcal{A}} \newcommand{\cB}{\mathcal{B}}
\newcommand{\cC}{\mathcal{C}} \newcommand{\cD}{\mathcal{D}}
\newcommand{\cE}{\mathcal{E}} \newcommand{\cF}{\mathcal{F}}
\newcommand{\cG}{\mathcal{G}} \newcommand{\cH}{\mathcal{H}}
\newcommand{\cI}{\mathcal{I}} \newcommand{\cJ}{\mathcal{J}}
\newcommand{\cK}{\mathcal{K}} \newcommand{\cL}{\mathcal{L}}
\newcommand{\cM}{\mathcal{M}} \newcommand{\cN}{\mathcal{N}}
\newcommand{\cO}{\mathcal{O}} \newcommand{\cP}{\mathcal{P}}
\newcommand{\cQ}{\mathcal{Q}} \newcommand{\cR}{\mathcal{R}}
\newcommand{\cS}{\mathcal{S}} \newcommand{\cT}{\mathcal{T}}
\newcommand{\cU}{\mathcal{U}} \newcommand{\cV}{\mathcal{V}}
\newcommand{\cW}{\mathcal{W}} \newcommand{\cX}{\mathcal{X}}
\newcommand{\cY}{\mathcal{Y}} \newcommand{\cZ}{\mathcal{Z}}

%Page style
\pagestyle{fancy}

\listfiles

\raggedbottom

\rhead{William Justin Toth : Rothblum Paper Notes} %CHANGE n to ASSIGNMENT NUMBER ijk TO COURSE CODE
\renewcommand{\headrulewidth}{1pt} %heading underlined
%\renewcommand{\baselinestretch}{1.2} % 1.2 line spacing for legibility (optional)

\begin{document}
\section{Stable Marriage Polytope}
\paragraph{}
A matching is given by a partial assignment matrix $x = \{x_{ij}\}_{i\in M, j\in W}$ of integers which satisfies:
\begin{align}
\sum_{j \in W} x_{ij} &\leq 1 &\text{ for all $i \in M$}\\
\sum_{i \in M} x_{ij} &\leq 1 &\text{ for all $j \in W$}\\
x_{ij} &\geq 0 &\text{for all $i \in M$ and $j \in W$}.
\end{align}
\paragraph{}
Let $A \subseteq (M \times W)$ be a set of acceptable pairs. A matching is said to be stable if it satisfies:
\begin{align}
x_ij &= 0 &\text{for all $(i,j) \in (M\times W)\backslash A$}
\end{align}
and that there exists no pair $(m,w) \in A$ for which any of:
\begin{enumerate}
\item Both $m$ and $w$ have mates and $w(x,m) <_m w$ and $m(x,w) <_w m$ \\
\item $m$ has a mate but not $w$ and $w(x,m) <_m w$\\
\item $w$ has a mate but not $m$ and $m(x,w) <_w m$\\
\item Both $m$ and $w$ have no mates
\end{enumerate}
hold.
\paragraph{Lemma 1}
Let $x$ be a matching. Then $x$ is stable if and only if $x$ satisfies $(4)$ and the following:
\begin{align}
\sum_{j>_m w} x_{mj} + \sum_{i >_w m} x_{iw} + x_{mw} &\geq 1 &\text{for all $(m,w)\in A$}.
\end{align}
\paragraph{Proof Sketch}
Provided that $x$ is integral, the set of inequalities in $(5)$ is violated if and only if there exists $(m,w) \in A$ such that $$ \sum_{j >_m w} x_{mj} = \sum_{i >_w m} x_{iw} = x_{mj} = 0. $$ From which we derive the precise four conditions which define stability and hence stability is equivalent to $(4)$ and $(5)$.$\blacksquare$
\section{Conflicting Interests Lemma}
\paragraph{}
Before stating the lemma we need some definitions. Define $$ S_M(x) = \{m\in M : \sum_{j\in W} x_{mj} > 0\}$$ and similarly $$S_W(x) = \{w\in W: \sum_{i \in M} x_{iw} > 0\}.$$ For $m \in S_M(x)$ let $w^*(x,m)$, and $w_*(x,m)$ be the most preferred and least preferred elements respectively of the set $\{ j \in W: x_{mj}>0\}$. Similarly for $w \in S_W(x)$ let $m^*(x,w)$ and $m_*(x,w)$ be the most preferred and least preferred elements respectively of the set $\{i \in M : x_{iw} > 0\}$.
\paragraph{}
Note that $x$ is integral if and only if for all $m \in S_M(x)$, $\sum_{j \in W} x_{mj}=1$ and $w^*(x,m) = w_*(x,m)$. Or equivalently, $x$ is integral if and only if for all $w \in S_W(x)$, $\sum_{i\in M} x_{iw} = 1$ and $m^*(x,w) = m_*(x,w)$.
\paragraph{Lemma 2}
Let $x = \{x_{ij}\}_{i\in M, j\in W}$ satisfy $(1) - (5)$ and let $(m,w) \in A$. Then 
\begin{align}
m\not\in S_M(x) \text{ or } (m\in S_M(x) \text{ and } w \geq_{m} w^*(x,m)) \nonumber\\\implies \sum_{i\in M} x_{iw} = 1 \text{ and } m \leq_w m_*(x,w),
\end{align}
and
\begin{align}
m\in S_M(x) \text{ and } w = w^*(x,m) \iff \sum_{i \in M} x_{iw} = 1 \text{ and } m = m_*(x,w).
\end{align}
Further, if the corresponding inequality from $(5)$ is tight for $x_{mw}$, that is 
\begin{align}
\sum_{j >_m w} x_{mj} + \sum_{i >_w m} x_{iw} + x_{mw} = 1,
\end{align}
then the converse of $(6)$ holds as well. Also 
\begin{align*}
m \in S_M(x) \iff \sum_{j \in W} x_{mj} = 1.
\end{align*}
\paragraph{Proof Sketch}
Starting with $(6)$, notice that in either case $\sum_{j >_m w} x_{mj} = 0$. Thus by $(5)$
\begin{align*}
1 &\leq \sum_{j >_m w} x_{mj} + x_{mj} \\
&\leq \sum_{i \in M} x_{iw} \\
&\leq 1 &\text{by $(2)$}.
\end{align*}
So every inequality holds with equality. The third equality gives part of the conclusion. Also since $\sum_{j >_m w} x_{mj} + x_{mj}=1$, by $(2)$, $x_{iw} = 0$ for all $i \in M_w$ with $i <_w m$. From this we conclude that $m \leq_w m_*(x,w)$ as otherwise there exists $i$ such that $x_{iw} > 0$.
\paragraph{}
Now consider the $\Rightarrow$ implication of $(7)$. Since $m \in S_M(x)$ and $w = w^*(x,m)$, $x_{mw} > 0$. Thus $w \in S_W(x)$ and $m \geq_w m_*(x,w)$. The result now follows from $(6)$.
\paragraph{}
We claim the map $w^*(x,\cdot) : S_M(x) \rightarrow F_W(x) = \{ w \in W: \sum_{i \in M} x_{iw} = 1 \}$ is one-to-one and onto. It is one-to-one because if $w^*(x,m_1) = w^*(x,m_2)$ then by $(7) \Rightarrow$ $m_1 = m_*(x,w) = m_2$. To see that the map is onto  observe that:
$$ |F_W(x)| = \sum_{j \in F_W(x)} 1 = \sum_{j \in F_W(x)} \sum_{i\in M} x_{ij} = \sum_{i \in M} \sum_{j\in F_W(x)} x_{ij} \leq \sum_{i \in S_M(x)} 1 = |S_M(x)|.$$
Since $w^*(x,\cdot)$ is one-to-one this implies that $|S_M(x)| = |F_W(x)|$, and for all $i \in S_M(x)$, $\sum_{j \in F_W(x)} x_{ij} = 1$. This gives that $m \in S_M(x)$ if and only if $\sum_{j \in W} x_{mj} = 1$.
\paragraph{}
For the $\Leftarrow$ implication of $(7)$ as $w^*(x, \cdot)$ is onto there is a $m' \in S_M(x)$ such that $w=w^*(x,m')$. Then by $\Rightarrow (7)$, $m'=m_*(x,w)=m$, and thus $w=w*(x,m')=w*(x,m)$.
\paragraph{}
Lastly suppose $(8)$ and we will show the converse of $(6)$. Then,
\begin{align*}
\sum_{j >_m w} x_{mj} + \sum_{i >_w m} x_{iw} + x_{mw} &= 1 \\
&= \sum_{i \in M} x_{iw} \\
&= \sum_{i >_w m} x_{iw} + x_{mw} &\text{since $m \leq_w m_*(x,w)$ implies for $i <_w m$, $x_{iw} = 0$}.
\end{align*}
Therefore $\sum_{j >_m w} x_{mj} = 0$. This establishes the converse of $(6)$. $\blacksquare$
\paragraph{Lemma 3}
By symmetric arguments a similar lemma to Lemma $2$ can be established from the female perspective.
Let $x = \{x_{ij}\}_{i\in M, j\in W}$ satisfy $(1) - (5)$ and let $(m,w) \in A$. Then 
\begin{align}
w\not\in S_W(x) \text{ or } (w\in S_W(x) \text{ and } m \geq_{w} m^*(x,w)) \nonumber\\\implies \sum_{j\in W} x_{mj} = 1 \text{ and }w \leq_m w_*(x,m),
\end{align}
and
\begin{align}
w\in S_W(x) \text{ and } m = m^*(x,w) \iff \sum_{j \in W} x_{mj} = 1 \text{ and } w = w_*(x,m).
\end{align}
Further, if $(8)$ holds then the converse of $(9)$ holds as well. Also 
\begin{align*}
w \in S_W(x) \iff \sum_{i \in M} x_{iw} = 1.
\end{align*}
\section*{Main Result}
\paragraph{Theorem}
Let $C$ be the set of solutions to $(1) - (5)$. Then the integer points in $C$ are precisely its extreme points.
\paragraph{Proof Sketch}
If $x$ is an integer element of $C$ then by Birkhoff's Theorem, $x$ is an extreme point of the system defined by $(1)-(3)$ and therefore an extreme point of $C$. Now suppose that $x$ is an extreme poin of $C$.
Define the matrices $z^*$, $z_*$, and $z$ as follows:
\begin{align}
(z^*)_{mw} &= \begin{cases}
	1, &\text{if $m\in S_M(x)$ and $w = w^*(x,m)$} \\
	0, &\text{otherwise}
\end{cases} 
\\
(z_*)_{mw} &= \begin{cases}
	1, &\text{if $m\in S_M(x)$ and $w = w_*(x,m)$} \\
	0, &\text{otherwise}
\end{cases}
\\
z_{mw} &= (z^*)_{mw} - (z_*)_{mw}.
\end{align}
By invoking Lemma $2$ and Lemma $3$, $z^*$ and $z_*$ can alternatively be defined as:
\begin{align}
(z^*)_{mw} &= \begin{cases}
	1, &\text{if $w\in S_W(x)$ and $m = m_*(x,w)$} \\
	0, &\text{otherwise}
\end{cases} 
\\
(z_*)_{mw} &= \begin{cases}
	1, &\text{if $w\in S_W(x)$ and $m = m^*(x,w)$} \\
	0, &\text{otherwise}
\end{cases}
\end{align}
\paragraph{}
We claim that:
\begin{align}
\sum_{j \in W} x_{ij} = 1 &\implies \sum_{j \in W} z_{ij} = 0 &\text{ for all $i \in M$}.
\end{align}
If $i \not\in S_M(x)$ this is trivial. If $i \in S_M(x)$ then by the definitions in $(11)$ and $(12)$ we have that $$\sum_{j \in W} (z^*)_{ij} = \sum_{j \in W} (z_*)_{ij} = 1$$ and therefore the claim holds. We additionally claim that:
\begin{align}
\sum_{i \in M} x_{ij} = 1 &\implies \sum_{i \in M} z_{ij} = 0 &\text{ for all $j \in W$}.
\end{align}
This follows by a similar argument using the equivalent definitions of $z^*$ and $z_*$ given in $(14)$ and $(15)$.
\paragraph{}
Next we claim that:
\begin{align}
x_{ij} = 0 &\implies z_{ij} = 0 &\text{for all $(i,j) \in M \times W$},
\end{align}
which is immediate from the fact that if $i \in S_M(x)$ then $x_{ij} = 0$ ensures that $j \not\in \{w^*(x,i), w_*(x,i)\}$. The following claim is a consequence of $(18)$ and $(4)$ taken together:
\begin{align}
z_{ij} &= 0 &\text{for all $(i,j) \in (M \times W) \backslash A$}.
\end{align}
\paragraph{}
We need one final claim before moving to the conclusion of the proof:
\begin{align}
&\sum_{j >_m w} x_{mj} + \sum_{i >_w m} x_{iw} + x_{mw} = 1 \nonumber\\&\implies \sum_{j >_m w} z_{mj} + \sum_{i >_w m} z_{iw} + z_{mw} = 1 &\text{for all $(m,w) \in A$}.
\end{align}
This claim will be established by considering three distinct cases.
\subparagraph{Case 1}
Suppose that $m \not\in S_M(x)$ or $m \in S_M(x)$ and $w \geq_m w^*(x,m)$. Then by $(11)$ and $(12)$ we have
\begin{align}
\sum_{j >_m w} (z^*)_{mj} = \sum_{j >_m w} (z_*)_{mj} = 0.
\end{align}
Further by Lemma $2$, $w \in S_W(x)$ and $m \leq_w m_*(x,w) \leq_w m^*(x,w)$. Therefore by $(14)$ and $(15)$
\begin{align}
\sum_{i >_w m} (z^*)_{iw} + (z^*)_{mw} = \sum_{i >_w m} (z_*)_{iw} + (z_*)_{iw} = 1.
\end{align}
Thus $(21)$ and $(22)$ taken together give the desired result.
\subparagraph{Case 2}
Suppose that $w \not\in S_W(x)$ or $w \in S_W(x)$ and $m \geq_w m^*(x,w)$. This case follows similarly to case $1$ but invokes Lemma $3$ instead of Lemma $2$.
\subparagraph{Case 3}
Suppose that $m \in S_M(x)$, $w \in S_W(x)$, $w <_m w^*(x,m)$, and $m <_w m^*(x,w)$. Since $m \in S_M(x)$ and $w <_m w^*(x,m)$, we have that $\sum_{j >_m w} (z^*)_{mj} = 1$ and $(z^*)_{mw} = 0$ by definition. By the hypothesis of claim $(20)$ the converse of $(6)$ holds true, so since $m \in S_M(x)$, $w \in S_W(x)$, and $w <_m w^*(x,m)$ we have that $m >_w m_*(x,w)$. Therefore by $(14)$, $\sum_{i >_w m} (z^*)_{iw} = 0$. Thus we have
\begin{align}
\sum_{j >_m w} (z^*)_{mj} + \sum_{i >_w m} (z^*)_{iw} + (z^*)_{mw}  = 1.
\end{align}
A symmetric argument establishes that
\begin{align}
\sum_{j >_m w} (z_*)_{mj} + \sum_{i >_w m} (z_*)_{iw} + (z_*)_{mw}  = 1.
\end{align}
Therefore by $(23)$ and $(24)$
$$\sum_{j >_m w} z_{mj} + \sum_{i >_w m} z_{iw} + z_{mw}  = 0$$
as desired.
\paragraph{}
By $(16) - (20)$ there exist $\epsilon > 0$ such that $x + \epsilon z, x - \epsilon z$ satisfy $(1) - (5)$. But we have that
$$ x = \frac{1}{2}(x - \epsilon z + x + \epsilon z), $$
and so since $x$ is an extreme point we conclude that
$$ x - \epsilon z = x = x + \epsilon z. $$
That is $z = 0$ or equivalently $z^* = z_*$. Therefore for all $m \in S_M(x)$, $w^*(x,m) = w_*(x,m)$ from which integrality follows (by an earlier remark). $\blacksquare$
\end{document}
