\documentclass[letterpaper,12pt,oneside,onecolumn]{article}
\usepackage[margin=1in, bottom=1in, top=1in]{geometry} %1 inch margins
\usepackage{amsmath, amssymb, amstext}
\usepackage{fancyhdr}
\usepackage{algorithm}
\usepackage{algpseudocode}
\usepackage{mathtools}
\usepackage{theorem}

\DeclarePairedDelimiter{\ceil}{\lceil}{\rceil}
\DeclarePairedDelimiter\floor{\lfloor}{\rfloor}

%Theorem
\newtheorem{fact}{Fact}[section]
\newtheorem{lemma}[fact]{Lemma}
\newtheorem{theorem}[fact]{Theorem}
\newtheorem{definition}[fact]{Definition}
\newtheorem{corollary}[fact]{Corollary}
\newtheorem{proposition}[fact]{Proposition}
\newtheorem{claim}[fact]{Claim}
\newtheorem{exercise}[fact]{Exercise}
\newtheorem{note}[fact]{Note}
\newenvironment{proof}{{\bf Proof:  }}{\hfill\rule{2mm}{2mm}}
%Macros
\newcommand{\A}{\mathbb{A}} \newcommand{\C}{\mathbb{C}}
\newcommand{\D}{\mathbb{D}} \newcommand{\F}{\mathbb{F}}
\newcommand{\N}{\mathbb{N}} \newcommand{\R}{\mathbb{R}}
\newcommand{\T}{\mathbb{T}} \newcommand{\Z}{\mathbb{Z}}
\newcommand{\Q}{\mathbb{Q}}
 
 
\newcommand{\cA}{\mathcal{A}} \newcommand{\cB}{\mathcal{B}}
\newcommand{\cC}{\mathcal{C}} \newcommand{\cD}{\mathcal{D}}
\newcommand{\cE}{\mathcal{E}} \newcommand{\cF}{\mathcal{F}}
\newcommand{\cG}{\mathcal{G}} \newcommand{\cH}{\mathcal{H}}
\newcommand{\cI}{\mathcal{I}} \newcommand{\cJ}{\mathcal{J}}
\newcommand{\cK}{\mathcal{K}} \newcommand{\cL}{\mathcal{L}}
\newcommand{\cM}{\mathcal{M}} \newcommand{\cN}{\mathcal{N}}
\newcommand{\cO}{\mathcal{O}} \newcommand{\cP}{\mathcal{P}}
\newcommand{\cQ}{\mathcal{Q}} \newcommand{\cR}{\mathcal{R}}
\newcommand{\cS}{\mathcal{S}} \newcommand{\cT}{\mathcal{T}}
\newcommand{\cU}{\mathcal{U}} \newcommand{\cV}{\mathcal{V}}
\newcommand{\cW}{\mathcal{W}} \newcommand{\cX}{\mathcal{X}}
\newcommand{\cY}{\mathcal{Y}} \newcommand{\cZ}{\mathcal{Z}}

%Page style
\pagestyle{fancy}

\listfiles

\raggedbottom

\rhead{William Justin Toth : Stable Matching Polytope}
\renewcommand{\headrulewidth}{1pt} %heading underlined
%\renewcommand{\baselinestretch}{1.2} % 1.2 line spacing for legibility (optional)

\begin{document}
\section{Stable Matching Alternating Paths}
\paragraph{}
Let $G=(V_0 \dot\cup V_1, E)$ be a bipartite graph with strict preference lists. Let $M_1$ and $M_2$ be two stable matchings on $G$. Let $M_i(v)$ denote the vertex such that $vM_i(v) \in M_i$ for $i=1,2$.
\begin{lemma}
Let $P$ be an alternating path in the subgraph of $G$ with edge set $M_1 \triangle M_2$. Let $i \in \{0,1\}$ and let $j \in \{0,1\} \backslash \{i\}$. Let $a \in \{1,2\}$ and let $b \in \{1,2\} \backslash \{a\}$. If $P$ starts at vertex $v \in V_i$ and $M_a(v) >_v M_b(v)$ and $P$ starts with edge $vM_a(v)$ then we have:
$$\forall v_0 \in V(P) \cap V_i,\ M_a(v_0) >_{v_0} M_b(v_0)$$
and
$$\forall v_1 \in V(P) \cap V_j,\ M_b(v_1) >_{v_1} M_a(v_1).$$
\end{lemma}
\begin{proof}
\paragraph{}
Proceed by induction on $|V(P)|$. If $|V(P)| = 1$ then vacuously true. If $|V(P)| = 2$ then $P$ is precisely some edge $vv_1 \in M_a$ with $v_1 \in V_j$. If $M_a(v_1) >_{v_1} M_b(v_1)$ then $vv_1$ blocks $M_b$, a contradiction.
\paragraph{}
Now consider the inductive case. Let $v_0 \in V(P) \cap V_i$. If $v_0$ is not the end vertex of $P$ then by induction on $vPv_0$, $M_a(v_0) >_{v_0} M_b(v_0)$ as desired. Now consider the case where $P$ ends at $v_0$. Since $|V(P)| > 2$ there is a vertex $u_1$ in $V_j$ which directly precedes $v_0$ on $P$. By induction on $vPu_1$,
$$M_b(u_1) >_{u_1} M_a(u_1). $$
So if $M_b(v_0) >_{v_0} M_a(v_0)$ then $u_1v_0$ blocks $M_a$, a contradiction. Demonstrating $M_b(v_1) >_{v_1} M_a(v_1)$ for all $v_1 \in V(P) \cap V_j$ follows similarly. 
\end{proof}
\begin{lemma}
Let $C$ be a connected component of the subgraph of $G$ with edge set $M_1 \triangle M_2$. Then $C$ is a cycle.
\end{lemma}
\begin{proof}
\paragraph{}
Suppose for a contradiction that $C$ is an open path with start vertex $v_0$ and end vertex $v_1$ (that is $v_0 \neq v_1$). Let $i$ be such that $v_0 \in V_i$ and $a$ be such that $M_a(v_0) >_{v_0} M_b(v_0)$. We claim that $v_0M_a(v_0) \in E(P)$. Indeed if this is not the case then as $v_0$ is adjacent to $M_a(v_0)$ and $M_b(v_0)$, $v_0$ is not the start vertex of $P$. So this path satisfies the conditions of Lemma $1.1$.
\paragraph{}
Now either $v_1 \in V_i$ or $v_0 \in V_j$. If $v_1 \in V_i$ then by Lemma $1.1$,
$$M_a(v_1) >_{v_1} M_b(v_1). $$
But by the alternating path structure $M_b(v_1)v_1 \in E(P)$. So as $v_1$ is adjacent to $M_a(v_1)$, $v_1$ is not an endpoint of $P$, a contradiction. If $v_1 \in V_j$ the contradiction follows by a similar argument.
\end{proof}
\begin{lemma}
Let $C$ be a connected component of the subgraph of $G$ with edge set $M_1 \triangle M_2$. Then either all vertices in $V(C) \cap V_0$ prefer $M_1$ to $M_2$ and all vertices in $V(C) \cap V_1$ prefer $M_2$ to $M_1$, or vice versa.
\end{lemma}
\begin{proof}
\paragraph{}
Let $v \in V(C) \cap V_0$. We may assume without loss of generality $v$ prefers $M_1$ to $M_2$. Let $v_0 \in V(C) \cap V_0$. Since $C$ is a cycle, there are two paths from $v$ to $v_0$ along $C$. One such path satisfies the conditions of Lemma $1.1$ and therefore $v_0$ also prefers $M_1$ to $M_2$. Similarly let $v_1 \in V(C) \cap V_1$. Since $C$ is a cycle, there are two paths from $v$ to $v_1$ along $C$. One such path satisfies the conditions of Lemma $1.1$ and therefore $v_1$ prefers $M_2$ to $M_1$.
\end{proof}

\section{Stable Matching Polytope}
\paragraph{}
Let $Q(G)$ be polytope described by the set of vectors $x \in \R^E$ satisfying:
\begin{align*}
x(\delta(v)) &\leq 1 &\forall v \in V\\
x(\delta^{>w}(v))+ x(\delta^{>v}(w)) + x_{vw} &\geq 1 &\forall vw \in E \\
x_e &\geq 0 &\forall e \in E.
\end{align*}
We desire to prove this polytope is integral. Note that it is clear the integral points of $Q(G)$ describe stable matchings of $G$. The argument proceeds by induction on $|E|$. Notice in the base case $|E| = 1$, it is immediate that $Q(G) = \{1\}$.
\paragraph{}
Consider the inductive case. Let $x$ be an extreme point of $Q(G)$. We have already shown (by token/charging argument) that there exists $e=vw \in E(G)$ such that $x_e = 0$ or $x_e = 1$. 
\paragraph{}
Consider the case $x_e = 0$. If $x$ is fractional then the residual vector, $z$, acquired by restricting $x$ to $\R^{E\backslash\{e\}}$, is fractional. Notice that $z \in Q(G-e)$. Further it is not difficult to show that $x$ satisfies:
$$x(\delta^{>w}(v))+ x(\delta^{>v}(w)) + x_{vw} = 1$$
and so $z$ lies on an edge of $Q(G-e)$ (the hyperplane above intersects this edge) . Let $M_1$ and $M_2$ be the stable matchings given by the extreme points of $Q(G-e)$ which describe the edge that $z$ lies on.
\paragraph{}
Without loss of generality we may assume
$$ |M_1 \cap(\delta^{>w}(v) \cup \delta^{>v}(w))| = 2 $$
and 
$$ |M_2 \cap(\{vw\} \cup \delta^{>w}(v) \cup \delta^{>v}(w))| = 0.$$
To see this notice that if $|M_1 \cap(\delta^{>w}(v) \cup \delta^{>v}(w))| = 1$ then $M_1$ would satisfy $$x(\delta^{>w}(v))+ x(\delta^{>v}(w)) + x_{vw} = 1$$ and $z$ would be the incidence vector of $M_1$, contradicting that $z$ is fractional. Similar for $M_2$.
\end{document}
