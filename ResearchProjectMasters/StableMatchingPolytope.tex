\documentclass[letterpaper,12pt,oneside,onecolumn]{article}
\usepackage[margin=1in, bottom=1in, top=1in]{geometry} %1 inch margins
\usepackage{amsmath, amssymb, amstext}
\usepackage{fancyhdr}
\usepackage{algorithm}
\usepackage{algpseudocode}
\usepackage{mathtools}
\usepackage{theorem}

\DeclarePairedDelimiter{\ceil}{\lceil}{\rceil}
\DeclarePairedDelimiter\floor{\lfloor}{\rfloor}

%Theorem
\newtheorem{fact}{Fact}[section]
\newtheorem{lemma}[fact]{Lemma}
\newtheorem{theorem}[fact]{Theorem}
\newtheorem{definition}[fact]{Definition}
\newtheorem{corollary}[fact]{Corollary}
\newtheorem{proposition}[fact]{Proposition}
\newtheorem{claim}[fact]{Claim}
\newtheorem{exercise}[fact]{Exercise}
\newtheorem{note}[fact]{Note}
\newenvironment{proof}{{\bf Proof:  }}{\hfill\rule{2mm}{2mm}}
%Macros
\newcommand{\A}{\mathbb{A}} \newcommand{\C}{\mathbb{C}}
\newcommand{\D}{\mathbb{D}} \newcommand{\F}{\mathbb{F}}
\newcommand{\N}{\mathbb{N}} \newcommand{\R}{\mathbb{R}}
\newcommand{\T}{\mathbb{T}} \newcommand{\Z}{\mathbb{Z}}
\newcommand{\Q}{\mathbb{Q}}
 
 
\newcommand{\cA}{\mathcal{A}} \newcommand{\cB}{\mathcal{B}}
\newcommand{\cC}{\mathcal{C}} \newcommand{\cD}{\mathcal{D}}
\newcommand{\cE}{\mathcal{E}} \newcommand{\cF}{\mathcal{F}}
\newcommand{\cG}{\mathcal{G}} \newcommand{\cH}{\mathcal{H}}
\newcommand{\cI}{\mathcal{I}} \newcommand{\cJ}{\mathcal{J}}
\newcommand{\cK}{\mathcal{K}} \newcommand{\cL}{\mathcal{L}}
\newcommand{\cM}{\mathcal{M}} \newcommand{\cN}{\mathcal{N}}
\newcommand{\cO}{\mathcal{O}} \newcommand{\cP}{\mathcal{P}}
\newcommand{\cQ}{\mathcal{Q}} \newcommand{\cR}{\mathcal{R}}
\newcommand{\cS}{\mathcal{S}} \newcommand{\cT}{\mathcal{T}}
\newcommand{\cU}{\mathcal{U}} \newcommand{\cV}{\mathcal{V}}
\newcommand{\cW}{\mathcal{W}} \newcommand{\cX}{\mathcal{X}}
\newcommand{\cY}{\mathcal{Y}} \newcommand{\cZ}{\mathcal{Z}}

%Page style
\pagestyle{fancy}

\listfiles

\raggedbottom

\rhead{William Justin Toth : Stable Matching Polytope}
\renewcommand{\headrulewidth}{1pt} %heading underlined
%\renewcommand{\baselinestretch}{1.2} % 1.2 line spacing for legibility (optional)

\begin{document}
\section{Stable Matching Alternating Paths}
\paragraph{}
Let $G=(V_0 \dot\cup V_1, E)$ be a bipartite graph with strict preference lists. Let $M_1$ and $M_2$ be two stable matchings on $G$. Let $M_i(v)$ denote the vertex such that $vM_i(v) \in M_i$ for $i=1,2$.
\begin{lemma}
Let $C$ be a connected component of the subgraph of $G$ with edge set $M_1 \triangle M_2$. Then either all vertices in $V(C) \cap V_0$ prefer $M_1$ to $M_2$ and all vertices in $V(C) \cap V_1$ prefer $M_2$ to $M_1$, or vice versa.
\end{lemma}
\begin{proof}
If $C$ is a path let $v$ be an end vertex of $C$, otherwise let $v$ be an arbitrary vertex of $C$. We may assume without loss of generality that $v \in V_0$ and $M_1(v) >_v M_2(v)$. Let $w = M_1(v)$. Then $w \in V(C)$. If $M_1(w) >_w M_2(w)$ then $vw$ blocks $M_2$. So $M_2(w) >_w M_1(w)$. Let $u=M_2(w)$. Then $u \in V(C)$. If $M_2(u) >_u M_1(u)$ then $uw$ blocks $M_1$. So $M_1(u) >_u M_2(u)$. This local structure can be extended forward to hit all edges in the connected component.
\end{proof}
\paragraph{}
Notice this proof above also show that every connected component is a cycle (the graph is finite and so the described process must terminate which is only possible by returning to the start vertex).
\begin{lemma}
$M_1 \triangle J_1$ and $M_1 \triangle J_2$ are stable matchings.
\end{lemma}
\begin{proof}
Suppose for a contradiction there is a blocking pair $ab \in E$ with $a \in V_0$ and $b \in V_1$. Let $i,j \in \{1,2\}$ such that $M(a) = M_i(a)$ and $M(b) = M_j(b)$. Since $ab$ is blocking,
$$b >_a M_i(a) \text{ and } a >_b M_j(b). $$
If $i=j$ then $ab$ blocks $M_i$, a contradiction. So $i \neq j$. Since $M_i$ is stable $M_i(b) >_b a$, and since $M_j$ is stable $M_j(a) >_a b$. That is,
$$M_j(a) >_a M_i(a) \text{ and } M_i(b) >_b M_j(b).$$
Thus $a,b \in V(J_j)$, which implies $i=j=2$, a contradiction.
\end{proof}
\section{Stable Matching Polytope}
\paragraph{}
Let $Q(G)$ be polytope described by the set of vectors $x \in \R^E$ satisfying:
\begin{align*}
x(\delta(v)) &\leq 1 &\forall v \in V\\
x(\delta^{>w}(v))+ x(\delta^{>v}(w)) + x_{vw} &\geq 1 &\forall vw \in E \\
x_e &\geq 0 &\forall e \in E.
\end{align*}
We desire to prove this polytope is integral. Note that it is clear the integral points of $Q(G)$ describe stable matchings of $G$. The argument proceeds by induction on $|E|$. Notice in the base case $|E| = 1$, it is immediate that $Q(G) = \{1\}$.
\paragraph{}
Consider the inductive case. Let $x$ be an extreme point of $Q(G)$. We have already shown (by token/charging argument) that there exists $e=vw \in E(G)$ such that $x_e = 0$ or $x_e = 1$. 
\paragraph{}
In the case $x_e = 1$, Kostya notes that we want a concise way to delete from $G$ the vertices $v,w$ and the residual vector from $x$ should be a vertex of $Q(G-\{v,w\})$. I haven't treated this yet.
\paragraph{}
Consider the case $x_e = 0$. If $x$ is fractional then the residual vector, $z$, acquired by restricting $x$ to $\R^{E\backslash\{e\}}$, is fractional. Notice that $z \in Q(G-e)$. Further it is not difficult to show that $x$ satisfies:
$$x(\delta^{>w}(v))+ x(\delta^{>v}(w)) + x_{vw} = 1$$
and so $z$ lies on an edge of $Q(G-e)$ (the hyperplane above intersects this edge) . Let $M_1$ and $M_2$ be the stable matchings given by the extreme points of $Q(G-e)$ which describe the edge that $z$ lies on.
\paragraph{}
Without loss of generality we may assume
$$ |M_1 \cap(\delta^{>w}(v) \cup \delta^{>v}(w))| = 2 $$
and 
$$ |M_2 \cap(\{vw\} \cup \delta^{>w}(v) \cup \delta^{>v}(w))| = 0.$$
To see this notice that if $|M_1 \cap(\delta^{>w}(v) \cup \delta^{>v}(w))| = 1$ then $M_1$ would satisfy $$x(\delta^{>w}(v))+ x(\delta^{>v}(w)) + x_{vw} = 1$$ and $z$ would be the incidence vector of $M_1$, contradicting that $z$ is fractional. Similar for $M_2$.
\paragraph{}
From Lemma $1.1$, $M_1 \triangle M_2$ is a collection of cycles which can be partitioned into two sets $J_1$ and $J_2$ as follows. Let $J_1$ contain the cycles whose vertices in $V_0$ prefer $M_1$ to $M_2$, and let $J_2$ contain the cycles whose vertices in $V_1$ prefer $M_1$ to $M_2$. Since $|M_1 \cap(\delta^{>w}(v) \cup \delta^{>v}(w))| = 2$, neither $J_1$ nor $J_2$ are empty. Therefore $M_1 \triangle J_1$ and $M_2 \triangle J_2$ are distinct matchings from $M_1$ and $M_2$.
\paragraph{}
Further it can be seen that
\begin{align*}
\chi(M_1 \triangle J_1) + \chi(M_1 \triangle J_2) &= (\chi(M_1) + \chi(J_1) - 2\chi(M_1 \cap J_1)) + (\chi(M_1) + \chi(J_2) -2\chi(M_1 \cap J_2))\\
&=2\chi(M_1) + \chi(J_1) + \chi(J_2) -2(\chi(M_1 \cap J_1) + \chi(M_2 \cap J_2)) \\
&= 2\chi(M_1) + \chi(M_1 \triangle M_2) - 2(\chi(M_1) + \chi(M_1 \cap M_2)) \\
&= \chi(M_1 \triangle M_2) + 2\chi(M_1 \cap M_2) \\
&=\chi(M_1) + \chi(M_2).
\end{align*}
\paragraph{}By Lemma $1.2$, $M_1 \triangle J_1$ and $M_1 \triangle J_2$ are stable. This contradicts that $M_1$ and $M_2$ are described by extreme points defining at edge of $Q(G-e)$. $\blacksquare$
\end{document}
