\documentclass[letterpaper,12pt,oneside,onecolumn]{article}
\usepackage[margin=1in, bottom=1in, top=1in]{geometry} %1 inch margins
\usepackage{amsmath, amssymb, amstext}
\usepackage{fancyhdr}
\usepackage{algorithm}
\usepackage{algpseudocode}
\usepackage{mathtools}

\DeclarePairedDelimiter{\ceil}{\lceil}{\rceil}
\DeclarePairedDelimiter\floor{\lfloor}{\rfloor}

%Macros
\newcommand{\A}{\mathbb{A}} \newcommand{\C}{\mathbb{C}}
\newcommand{\D}{\mathbb{D}} \newcommand{\F}{\mathbb{F}}
\newcommand{\N}{\mathbb{N}} \newcommand{\R}{\mathbb{R}}
\newcommand{\T}{\mathbb{T}} \newcommand{\Z}{\mathbb{Z}}
\newcommand{\Q}{\mathbb{Q}}
 
 
\newcommand{\cA}{\mathcal{A}} \newcommand{\cB}{\mathcal{B}}
\newcommand{\cC}{\mathcal{C}} \newcommand{\cD}{\mathcal{D}}
\newcommand{\cE}{\mathcal{E}} \newcommand{\cF}{\mathcal{F}}
\newcommand{\cG}{\mathcal{G}} \newcommand{\cH}{\mathcal{H}}
\newcommand{\cI}{\mathcal{I}} \newcommand{\cJ}{\mathcal{J}}
\newcommand{\cK}{\mathcal{K}} \newcommand{\cL}{\mathcal{L}}
\newcommand{\cM}{\mathcal{M}} \newcommand{\cN}{\mathcal{N}}
\newcommand{\cO}{\mathcal{O}} \newcommand{\cP}{\mathcal{P}}
\newcommand{\cQ}{\mathcal{Q}} \newcommand{\cR}{\mathcal{R}}
\newcommand{\cS}{\mathcal{S}} \newcommand{\cT}{\mathcal{T}}
\newcommand{\cU}{\mathcal{U}} \newcommand{\cV}{\mathcal{V}}
\newcommand{\cW}{\mathcal{W}} \newcommand{\cX}{\mathcal{X}}
\newcommand{\cY}{\mathcal{Y}} \newcommand{\cZ}{\mathcal{Z}}

%Page style
\pagestyle{fancy}

\listfiles

\raggedbottom

\rhead{William Justin Toth : Iterative Rounding} 
\renewcommand{\headrulewidth}{1pt} %heading underlined
%\renewcommand{\baselinestretch}{1.2} % 1.2 line spacing for legibility (optional)

\begin{document}
\section{Rank Lemma}
\paragraph{Definitions} Let $P = \{x : Ax \geq b, x \geq 0\}$. For $x \in P$, $A^=$ denotes the submatrix of $A$ restricted to rows $a_i^T$ of $A$ for which $a_i^Tx = b_i$ (that is, the rows at equality at $x$). Let $A^=_x$ denote the submatrix of $A^=$ consisting of colums $a_j$ for which $x_j > 0$ (that, the columns of $A^=$ corresponding to nonzeros in $x$).

\paragraph{Lemma 1} Let $P = \{x : Ax \geq b, x \geq 0\}$. Then $x$ be an extreme point of $P$ if and only if $A_x^=$ has linearly independent columns.
\paragraph{Proof Sketch} First $(\impliedby)$ direction. If $x$ is not an extreme point then there exists $y \neq 0$ such that $x+y, x-y \in P$. So $A^=y = 0$ and thus $A^=_y$ has linearly dependent columns. Since $x_j = 0 \implies y_j=0$, $A_y^=$ is a submatrix of $A_x^=$. Therefore the columns of $A_x^=$ are linearly dependent.
\paragraph{}
Now the $(\implies)$ direction. Suppose $A_x^=$ has linearly dependent columns. Then there exists $y \neq 0$ such that $A_x^=y = 0$. Complete $y$ to the same dimension as $x$ with zeroes such that $A^=y = 0$ (in this construction $x_j = 0 \implies y_j = 0$). Now we can choose $\epsilon > 0$ small enough that $x+\epsilon y \geq 0$, $x - \epsilon y \geq 0$ and $A(x+\epsilon y) = Ax + \epsilon Ay \geq b$ (similarly $A(x -\epsilon y) \geq b$). Therefore $x$ is not an extreme point of $P$. $\blacksquare$
\paragraph{Rank Lemma}
Let $P = \{x : Ax \geq b, x \geq 0\}$. Let $x$ be an extreme point of $P$ such that $x>0$. Then any maximal number of linearly independent tight constraints of the forms $A_ix = b_i$ for some row $i$ of $A$ equals the number of variables.
\paragraph{Proof Sketch}
Since $x > 0$, $A_x^= = A^=$. Therefore by Lemma $1$, $A^=$ has linearly independent columns. Therefore the rank of $A^=$ is equal to the number of variables, and hence any maximal number of linearly independent tight constraints of the forms $A_ix = b_i$ for some row $i$ of $A$ equals the number of variables. $\blacksquare$
\section{Stable Matching LP Forumlation}
\paragraph{Defintions}
Let $G = (V, E)$ be a bipartite graph with $V = V_1 \dot\cup V_2$. We may weight the edges $e \in E$ with weights $c_e$. If we assign for each vertex a list of preferences with respect to that vertex's neighbours then we have a (weighted) stable marriage instance. Let $v,w \in V$, define $\delta_v^>(w) = \{vj \in \delta(v): j \text{ is strictly preferred to } w \text{ by } v\}$. Note we can also define similar sets for relations like $=, \geq, <$ etc. We may order the elements of the neighbourhood of $v$ with respect to the $>$ relation. Let $n^*(v)$ ($n_*(v)$) denote the most (least respectively) preferred elements of the neighbourhood of $v$.
\paragraph{LP formulation}
Let $LP_S(G)$ denote the following linear program:
\begin{align*}
\max\ c^Tx \\
\text{s.t.} \sum_{e\in \delta(v)} x_e &\leq 1 &\forall v \in V\\
\sum_{e \in \delta_v^>(w)} x_e + \sum_{e \in \delta_w^>(v)} x_e + x_{vw} &\geq 1 &\forall vw \in E \\
x_e &\geq 0 &\forall e \in E.
\end{align*}
It is not difficult to show that integral extreme point solutions to $LP_S(G)$ are equivalent to characteristic vectors of stable matchings of $G$.
\paragraph{Lemma 2}
Let $x$ be an extreme point solution to $LP_S(G)$ such that $x>0$. Then there exists $W \subseteq V$ and $T \subseteq E$ such that the following hold:
\begin{enumerate}
\item $\sum_{e\in \delta(v)} x_e = 1$, $\forall v \in W$.
\item $\sum_{e \in \delta_v^>(w)} x_e + \sum_{e \in \delta_w^>(v)} x_e + x_{vw} = 1$, $\forall vw \in T$.
\item The vectors in $\{\chi(\delta(v)) : v \in W\}$ together with the vectors in $\{\chi(\delta_v^>(w)) + \chi(\delta_w^>(v)) + \chi(vw) : vw \in T\}$ are all linearly independent.
\item $|W| + |T| = |E|$.
\end{enumerate}
\paragraph{Proof}
Immediate from Rank Lemma. $\blacksquare$
\paragraph{Lemma 3}
Let $x$ be an extreme point solution of $LP_S(G)$ such that $0 < x < 1$. Let $W \subseteq V$, and $T \subseteq E$ be sets that together satisfy $1-4$ of Lemma $2$. If $W$ and $T$ are chosen so that $|W|$ is maximized then for any $v \in V$ if $w = n^*(v)$ then $vw \not\in T$.
\paragraph{Proof}
Let $v \in V$, let $w = n^*(v)$ and suppose for a contradiction that $vw \in T$. Then 
$$\sum_{e \in \delta_w^>(v)} x_e + \sum_{e \in \delta_v^>(w)} x_e + x_{vw} = 1. $$
Since $w = n^*(v)$, $\delta_v^>(w) = \emptyset$ (also notice $\chi(\delta_v^>(w)) = 0$). Therefore in fact we have that
$$\sum_{e \in \delta_w^\geq(v)} x_e = 1.$$
Since $\sum_{e \in \delta(w)} x_e \leq 1$ and $\delta(w) = \delta_w^\geq(v) \dot\cup \delta_w^<(v)$, $$\sum_{e \in \delta_w^<(v)} x_e = 0. $$
Therefore, since $x_e > 0$ for each $e \in \delta_w^<(v)$, $$\chi(\delta_w^<(v)) = 0.$$
Then we have that
\begin{align*}
\chi(\delta(w)) &= \chi(\delta_w^\geq(v)) + \chi(\delta_w^<v(v)) \\
&= \chi(\delta_w^\geq(v)) \\
&= \chi(\delta_w^>(v)) + \chi(\delta_v^>(v)) + \chi(vw).
\end{align*}
So if $w \in W$ this contradicts condition $3$ of Lemma $2$, the linear independence. But if $w \not\in W$ then by the equivalence just demonstrated the sets $W' = W \cup \{w\}$ and $T' = T \backslash \{vw\}$ also satisfy $1-4$ of Lemma $2$ yet $|W'| > |W|$ contradicting the maximality of $|W|$. $\blacksquare$
\paragraph{Lemma 4}
Let $x$ be an extreme point solution of $LP_S(G)$. Then there exists $e \in E$ such that $x_e = 0$ or $x_e = 1$.
\paragraph{Proof}
Suppose for contradiction that $0 < x_e < 1$ for all $e \in E$. By Lemma $2$ there exists $W, T$ satisfying properties $1-4$ of Lemma $2$. Choose such $W$ and $T$ so that $|W|$ is maximized. First notice that
$$d_E(v) \geq 2 $$
for all $v \in W$. This follows since $\sum_{e \in \delta(v)} x_e = 1$ and $x_e < 1$ for each $e \in \delta(v)$.
\paragraph{}
Let $v \in W$. Let $w = n_*(v)$. We claim that $vw \not\in T$. Suppose for a contradiction that $vw \in T$. Then we would have:
$$  \sum_{e \in \delta_v^>(w)} x_e + \sum_{e \in \delta_w^>(v)} x_e + x_{vw} = 1. $$
But since $w=n_*(v)$, $\delta_v^<(w) = \emptyset$ (also notice that $\chi(\delta_v^<(w)) = 0$). Thus $\delta_v^\geq(w) = \delta(v)$. Therefore
$$\sum_{e \in \delta_v^>(w)} x_e + x_{vw} = \sum_{e \in \delta(v)} x_e = 1 \text{ since $v \in W$}. $$
So we have that $$ \sum_{e \in \delta_w^>(v)} x_e = 0. $$ That is $\delta_w^>(v) = \emptyset$ since $x_e > 0$ for $e \in E$. Then $$\chi(\delta_w^>(v)) = 0.$$
Therefore
\begin{align*}
\chi(\delta_v^>(w)) + \chi(\delta_w^>(v)) + \chi(vw) &= \chi(\delta_v^>(w)) + \chi(vw) \\
&= \chi(\delta(v)).
\end{align*}
Since $v \in W$ this tells us that condition $3$ of Lemma $2$ is violated, a contradiction. Therefore $vw \not\in T$.
\paragraph{}
Now for any $v \in V$, by Lemma $3$, $vn^*(v) \not\in T$. Since $d_E(v) \geq 2$ implies that $n^*(v) \neq n_*(v)$ we thus have that $d_{E\backslash T}(v) \geq 2$ for all $v \in W$.
\paragraph{}
We are now ready to give a fractional token counting argument to obtain the desired contradiction. To each $e \in E$ we will distribute one token. The tokens will be redistributed to elements of $W$ and $T$ such that each element receives one token. In showing that some fractional token remains with an edge $e \in E$ after redistributing we conclude that $|W| + |T| \neq |E|$, contradicting condition $4$ of Lemma $2$.
\paragraph{}
Redistribute the tokens according to the following rules. For any $e=vw \in E$ do:
\begin{enumerate}
\item If $e \in T$ then $e$ keeps its token,
\item otherwise:
\subitem if $v \in W$ then $e$ gives $\frac{1}{2}$ token to $v$,
\subitem if $w \in W$ then $e$ give $\frac{1}{2}$ token to $w$.
\end{enumerate}
Then after redistribution by rule $1$ every $e \in T$ has a token. Let $v \in W$. By our previous arguments $d_{E\backslash T}(v) \geq 2$. So there exist at least two edges which each give $\frac{1}{2}$ token to $v$. Therefore $v$ has at least one token after redistribution. It remains to show that there is an edge not in $T$ still holding some token fraction after redistribution.
\paragraph{}
We claim that there exist $v \in V\backslash W$. Suppose not, that is $W = V = V_1 \dot\cup V_2$. Then
\begin{align*}
\sum_{v \in V_1} \chi(\delta(v)) &= \chi(\delta(V_1)) \\
&= \chi(\delta(V_2)) &\text{since $G$ is bipartite} \\
&= \sum_{v \in V_2} \chi(\delta(v)).
\end{align*}
Since $V_1, V_2 \subseteq W$ this contradicts Lemma $2$ part $3$. Therefore there exists $v \in V \backslash W$. Let $w = n^*(v)$. Let $e = vw \in E$. By Lemma $3$, $e \not \in T$. Since $v \not\in W$, $e$ keeps at least $\frac{1}{2}$ token. Therefore $|E| > |W| + |T|$ contradicting condition $4$ of Lemma $2$. $\blacksquare$
\section{Iterative Algorithm}
\paragraph{}
In this section we will show an iterative method which solves the weighted stable marriage problem, and in doing so prove that the extreme points of $LP_S(G)$ are integral. 

\begin{algorithm} \begin{algorithmic}[1]
\Procedure{Iterative Stable Matching Algorithm}{}
\State Initialize $M \gets \emptyset$.
\While{$E(G) \neq \emptyset$}
\State Find an extereme point optimal solution $x$ to $LP_S(G)$.
\State Remove every edge $e$ from $G$ where $x_e = 0$.
\If{$\exists e=vw \in E$ with $x_e = 1$}
\State $M \gets M \cup \{e\}$
\State $\forall u \in N(v)$ (the neighbourhood of $v$) such that $v$ prefers $u$ to $w$ remove every edge in $\delta_u^<(v)$ from $G$
\State $\forall u \in N(w)$ such that $w$ prefers $u$ to $v$ remove every edge in $\delta_u^<(w)$ from $G$.
\State $G \gets G - v - w$.
\EndIf
\EndWhile
\State \Return $M$
\EndProcedure
\end{algorithmic} \end{algorithm}

\paragraph{Lemma 5}
If the above algorithm finds in every iteration an edge $e$ such that $x_e = 0$ in step $5$ or an edge $e$ with $x_e = 1$ in step $6$ then it returns a stable matching $M$ of weight at least the optimal solution to $LP_S(G)$.
\paragraph{Proof}
Proceed by induction on the number of iterations of the algorithm. In the base case the algorithm runs for exactly one iteration and the result is trivial.
\paragraph{}
Now consider the inductive case. Suppose the algorithm finds an edge $e$ with $x_e = 0$ in step $5$. The residual problem is to find a stable matching in $G' = G - e$. Let $x_r$ be the residual solution, that is $x$ restricted to $E \backslash \{e\}$. Notice that $x_r$ is a feasible solution to $LP_S(G')$. By the inductive hypothesis, the algorithm returns a stable matching $M' \subseteq E(G')$ with weight at least that of the optimal solution to $LP_S(G')$. Since $c(M') \geq c^T_rx_r = c^Tx$ the result holds in this case.
Now suppose the algorithm finds an edge $e=vw$ with $x_e = 1$ in step $6$. The residual problem in this case is to find a stable matching that contains $e$. Consider the residual graph built by the algorithm:
$$ G' = (G\backslash(\bigcup_{uv \in \delta_v^>(w)}\delta_u^<(v) \cup \bigcup_{uw \in \delta_w^>(v)} \delta_u^<(w) ))- v - w.$$
The algorithm will return a matching $M'$ of weight at least the optimal solution to $LP_S(G')$. Let $x_r$ be the solution $x$ restricted to $E(G')$. Notice that each entry removed from $x$ to obtain $x_r$ with the exception of $x_e$ is $0$ (otherwise $x$ is not feasible). Then $x_r$ is a feasible solution to $LP_S(G')$. So $c(M') \geq c^T_rx_r$. 
\paragraph{}
The matching returned by the algorithm is $M = M' \cup \{e\}$. We claim this matching is stable. Indeed there is no blocking pair among vertices of edges in $M'$ since $M'$ is a stable matching on $G'$. The only possible blocking pair would be $v$ or $w$ paired with a vertex endpoint of an edge in $M'$. Say without loss of generality $(v, u)$ is such a pair. Then $v$ prefers $u$ to $w$ and $u$ prefers $v$ to their partner, say $y$. But then $uv \in \delta_v^>(w)$ and $uy \in \delta_u^<(v)$ so $uy$ is not an edge in $G'$, and hence cannot be an edge in $M'$. Therefore the matching $M$ is stable. So we have
\begin{align*}
c(M) &= c(M') + c_e \\
&\geq c^T_rx_r + c_e &\text{since $c(M') \geq c^T_rx_r$}\\
&= c^Tx &\text{since $x_e = 1$}
\end{align*} 
and thus the weight of the stable matching $M$ returned by the algorithm is at least the weight of the optimal solution to $LP_S(G)$. $\blacksquare$
\paragraph{Theorem}
$LP_S(G)$ is integral.
\paragraph{Proof}
Let $x$ be an extreme point of $LP_S(G)$. Suppose that $x$ is not integral. Choose weight function $c$ such that $x$ is the unique optimal solution to $LP_S(G)$ with objective function $c$. By Lemma $5$ the above algorithm returns a stable matching $M$ such that $c(M) \geq c^Tx$. The characteristic vector of this stable matching, call it $x'$, is an integral feasible solution to $LP_S(G)$. Since $c^Tx' \geq c^Tx$, $x'$ is an optimal solution to $LP_S(G)$. Since $x$ is not integral, $x \neq x'$. Therefore $x$ is not the unique optimal solution to $LP_S(G)$, a contradiction. So $LP_S(G)$ is integral. $\blacksquare$
\end{document}
