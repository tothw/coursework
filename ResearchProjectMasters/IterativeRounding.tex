\documentclass[letterpaper,12pt,oneside,onecolumn]{article}
\usepackage[margin=1in, bottom=1in, top=1in]{geometry} %1 inch margins
\usepackage{amsmath, amssymb, amstext}
\usepackage{fancyhdr}
\usepackage{algorithm}
\usepackage{algpseudocode}
\usepackage{mathtools}

\DeclarePairedDelimiter{\ceil}{\lceil}{\rceil}
\DeclarePairedDelimiter\floor{\lfloor}{\rfloor}

%Macros
\newcommand{\A}{\mathbb{A}} \newcommand{\C}{\mathbb{C}}
\newcommand{\D}{\mathbb{D}} \newcommand{\F}{\mathbb{F}}
\newcommand{\N}{\mathbb{N}} \newcommand{\R}{\mathbb{R}}
\newcommand{\T}{\mathbb{T}} \newcommand{\Z}{\mathbb{Z}}
\newcommand{\Q}{\mathbb{Q}}
 
 
\newcommand{\cA}{\mathcal{A}} \newcommand{\cB}{\mathcal{B}}
\newcommand{\cC}{\mathcal{C}} \newcommand{\cD}{\mathcal{D}}
\newcommand{\cE}{\mathcal{E}} \newcommand{\cF}{\mathcal{F}}
\newcommand{\cG}{\mathcal{G}} \newcommand{\cH}{\mathcal{H}}
\newcommand{\cI}{\mathcal{I}} \newcommand{\cJ}{\mathcal{J}}
\newcommand{\cK}{\mathcal{K}} \newcommand{\cL}{\mathcal{L}}
\newcommand{\cM}{\mathcal{M}} \newcommand{\cN}{\mathcal{N}}
\newcommand{\cO}{\mathcal{O}} \newcommand{\cP}{\mathcal{P}}
\newcommand{\cQ}{\mathcal{Q}} \newcommand{\cR}{\mathcal{R}}
\newcommand{\cS}{\mathcal{S}} \newcommand{\cT}{\mathcal{T}}
\newcommand{\cU}{\mathcal{U}} \newcommand{\cV}{\mathcal{V}}
\newcommand{\cW}{\mathcal{W}} \newcommand{\cX}{\mathcal{X}}
\newcommand{\cY}{\mathcal{Y}} \newcommand{\cZ}{\mathcal{Z}}

%Page style
\pagestyle{fancy}

\listfiles

\raggedbottom

\rhead{William Justin Toth : Iterative Rounding} %CHANGE n to ASSIGNMENT NUMBER ijk TO COURSE CODE
\renewcommand{\headrulewidth}{1pt} %heading underlined
%\renewcommand{\baselinestretch}{1.2} % 1.2 line spacing for legibility (optional)

\begin{document}
\section{Rank Lemma}
\paragraph{Definitions} Let $P = \{x : Ax \geq b, x \geq 0\}$. For $x \in P$, $A^=$ denotes the submatrix of $A$ restricted to rows $a_i^T$ of $A$ for which $a_i^Tx = b_i$ (that is, the rows at equality at $x$). Let $A^=_x$ denote the submatrix of $A^=$ consisting of colums $a_j$ for which $x_j > 0$ (that, the columns of $A^=$ corresponding to nonzeros in $x$).

\paragraph{Lemma 1} Let $P = \{x : Ax \geq b, x \geq 0\}$. Then $x$ be an extreme point of $P$ if and only if $A_x^=$ has linearly independent columns.
\paragraph{Proof Sketch} First $(\impliedby)$ direction. If $x$ is not an extreme point then there exists $y \neq 0$ such that $x+y, x-y \in P$. So $A^=y = 0$ and thus $A^=_y$ has linearly dependent columns. Since $x_j = 0 \implies y_j=0$, $A_y^=$ is a submatrix of $A_x^=$. Therefore the columns of $A_x^=$ are linearly dependent.
\paragraph{}
Now the $(\implies)$ direction. Suppose $A_x^=$ has linearly dependent columns. Then there exists $y \neq 0$ such that $A_x^=y = 0$. Complete $y$ to the same dimension as $x$ with zeroes such that $A^=y = 0$ (in this construction $x_j = 0 \implies y_j = 0$). Now we can choose $\epsilon > 0$ small enough that $x+\epsilon y \geq 0$, $x - \epsilon y \geq 0$ and $A(x+\epsilon y) = Ax + \epsilon Ay \geq b$ (similarly $A(x -\epsilon y) \geq b$). Therefore $x$ is not an extreme point of $P$. $\blacksquare$
\paragraph{Rank Lemma}
Let $P = \{x : Ax \geq b, x \geq 0\}$. Let $x$ be an extreme point of $P$ such that $x>0$. Then any maximal number of linearly independent tight constraints of the forms $A_ix = b_i$ for some row $i$ of $A$ equals the number of variables.
\paragraph{Proof Sketch}
Since $x > 0$, $A_x^= = A^=$. Therefore by Lemma $1$, $A^=$ has linearly independent columns. Therefore the rank of $A^=$ is equal to the number of variables, and hence any maximal number of linearly independent tight constraints of the forms $A_ix = b_i$ for some row $i$ of $A$ equals the number of variables. $\blacksquare$
\section{Stable Matching LP Forumlation}
\paragraph{Defintions}
Let $G = (M \cup W, E)$ be a stable marriage instance. Let $m \in M$, define $\delta_m^>(w) = \{mj \in E: j >_m w, j \in W\}$. For $w \in W$, define $\delta_w^>(m)$ similarly. Note we can also define similar sets for relations like $=, \geq, <$ etc.
\paragraph{LP formulation}
Let $LP_S(G)$ denote the following linear program:
\begin{align*}
\max\ c^Tx \\
\text{s.t.} \sum_{e\in \delta(v)} x_e &\leq 1 &\forall v \in M\cup W\\
\sum_{e \in \delta_m^>(w)} x_e + \sum_{e \in \delta_w^>(m)} x_e + x_{mw} &\geq 1 &\forall mw \in E \\
x_e &\geq 0 &\forall e \in E.
\end{align*}
\paragraph{Lemma 2}
Let $x$ be an extreme point solution to $LP_S(G)$ such that $x>0$. Then there exists $V \subseteq M \cup W$ and $T \subset E$ such that the following hold:
\begin{enumerate}
\item $\sum_{e\in \delta(v)} x_e = 1$, $\forall v \in V$.
\item $\sum_{e \in \delta_m^>(w)} x_e + \sum_{e \in \delta_w^>(m)} x_e + x_{mw} = 1$, $\forall mw \in T$.
\item The vectors in $\{\chi(\delta(v)) : v \in V\}$ together with the vectors in $\{\chi(\delta_m^>(w)) + \chi(\delta_w^>(m)) + \chi(mw) : mw \in T\}$ are all linearly independent.
\item $|V| + |T| = |E|$.
\paragraph{Proof}
Immediate from Rank Lemma. $\blacksquare$
\paragraph{Lemma 3}
Let $x$ be an extreme point solution of $LP_S(G)$. Then there exists $e \in E$ such that $x_e = 0$ or $x_e = 1$.
\paragraph{Proof}
Suppose for contradiction that $0 < x_e < 1$ for all $e \in E$. By Lemma $2$ there exists $V, T$ satisfying properties $1-4$ of Lemma $2$. Choose such $V$ and $T$ so that $|V|$ is maximized. First notice that
$$d_E(v) \geq 2 $$
for all $v \in V$. This follows since $\sum_{e \in \delta(v)} x_e = 1$ and $x_e < 1$ for each $e \in \delta(v)$.
\paragraph{}
Let $w$ be the element of $W$ least preferred by $m$ such that $mw \in E$. We claim that $mw \not\in T$. Suppose for a contradiction that $mw \in T$. Then we would have:
$$  \sum_{e \in \delta_m^>(w)} x_e + \sum_{e \in \delta_w^>(m)} x_e + x_{mw} = 1. $$
But since $w$ is the least favourite of $m$, $\delta_m^>(w) \cup mw = \delta(m)$. Therefore
$$\sum_{e \in \delta_w^>(m)} x_e + x_{mw} = \sum_{e \in \delta(m)} x_e = 1 \text{ since $m \in V$}. $$
So we have that $$ \sum_{e \in \delta_w^>(m)} x_e = 0. $$ That is $\delta_w^>(m) = \emptyset$ since $x_e > 0$ for $e \in E$. Then $$\chi(\delta_w^>(m)) = 0.$$
Therefore
\begin{align*}
\chi(\delta_m^>(w)) + \chi(\delta_w^>(m)) + \chi(mw) &= \chi(\delta_m^>(w)) + \chi(mw) \\
&= \chi(\delta(m)).
\end{enumerate}
Since $m \in V$ this tell us that condition $3$ of Lemma $2$ is violated, a contradiction. Therefore $mw \not\in T$.
\paragraph{}
Now let $w$ be the element of $W$ most preferred by $m$ such that $mw \in E$. We claim that $mw \not\in T$. Suppose for a contradiction that $mw \in T$. Then we have that $\delta_m^>(w) = \emptyset$. So then
\begin{align*}
1 &= \sum_{e \in \delta_m^>(w)} x_e + \sum_{e \in \delta_w^>(m)} x_e + x_{mw} \\
 &= \sum_{e \in \delta_w^>(m)} x_e + x_{mw} \\
&= \sum_{e \in \delta_w^\geq(m)} x_e.
\end{align*}
If $w \in V$ then $\sum_{e \in \delta(w)} x_e = 1$ and therefore $\sum_{e \in \delta_w^<(w)} = 0$.
\end{document}
