%======================================================================
\chapter{Introduction}
%======================================================================
This chapter gives motivation, history of related work, and how our results fit into the current body of knowledge.

%----------------------------------------------------------------------
\section{Motivation and Related Work}

\paragraph{}
These days online dating applications are all the rage for eligible singles, but before these apps made finding a date as easy as a swipe of the finger one way people met potential suitors was through their social circle. Imagine you really like playing cupid and you happen to know $n$ heterosexual male and heterosexual female friends who are all single and looking to for dates. Since you're such a good friend you know each of your friends preferences over who they would like to be paired with in the opposite gender. You would like to set up your single friends together in $n$ pairs and you have a goal: you don't want your friends to mutually circument your dating suggestions. Say some are your friends are Adam, Bob, Amy, and Brianne. If you pair Adam and Amy together, and Bob and Brianne, but Adam and Brianne mutually prefer each other to Amy and Bob respectively then if they might want to ditch their suggested dates for each other. You hope to avoid this.
\paragraph{}
In the previous dating game your single friends with their preferences form what is called by economists a two-sided matching market \cite{roth1992two}. The hope that your friends don't mutually want to ditch their dates is the condition of stability which is a highly desireable property in matching markets. Other (perhaps more practical) examples two-sided matching markets is the matching of medical school residents with hospitals for internships, college applicants with colleges, and labour markets between employers and employees. The critical features of two-sided matching markets are the prescence of two distinct groups without crossover who have interests in forming some mutual relationship with a member of the opposite group, and not only that but have preferences who they become involved with in the opposite group.
\paragraph{}
The abstract mathematical model of these phenomena is called a stable matching problem. Stable matching was first introduced by Gale and Shapley to investigate college admissions \cite{gale1962college}. Their groundbreaking work earned them the Nobel Prize in Economics. Perhaps more significant than that they spawned a field of inquiry that brings together mathematicians, computer scientists, and economists in studying stable matching problems and their variants.
\paragraph{}
In Gale and Shapley's work they prove that stable matchings always exist when preferences are strict through giving their famous deferred acceptance algorithm to compute stable matchings. They also show that stable matchings need not exist if we remove the "sides" from the market and allow anyone to be matched with anyone (called the Stable Roomates problem). After Gale and Shapley's work the next major landmark in stable matching was the work of Knuth \cite{knuthmariages} in which the connection between stable matchings and distributive lattices is introduced, a result Knuth attributes to Conway. Since then interest in stable matchings has exploded, with people coming up with all sorts of variations on the problems and studying them in turn.
\paragraph{}
One such variation we are interested in is the question wherein each possible pairing in a stable matching problem is weighted by some possible utility, and our goal is to find a stable matching of maximum total weight. Vande Vate was the first to formulate stable matchings as the feasible region of a linear program and prove that the extreme points of his formulation were integer-valued vectors \cite{vate1989linear}. In doing so he implies a means of solving the weighted stable matching variant through the use of linear programming techniques. Vande Vate's formulation was not for the most general case, and his proof was long and complicated. Rothblum gave a formulation for the more general case of weighted stable matching in bipartite graphs and a simpler proof that Vande Vate's via an extreme point argument \cite{rothblum1992characterization}.
\paragraph{}
In Knuth's work on stable matching he proposes a variant where we consider three-sided markets instead of two. One important application for this type of model would be three-way kidney exchange in the context of organ transplant procedures in hospitals \cite{saidman2006increasing}. Knuth asks the question: for when do stable matchings exist in this context? As late as 2006 Eriksson et al \cite{eriksson2006three} show that for complete preference lists with at most 4 agents in each side there always exists a stable matching. In 2010 Biro et al \cite{biro2010three} show a counterexample for with 6 agents in each side but with incomplete preferences allowed. Later Farczadi et al \cite{farczadi2014stable} prove that a natural approach through extending Gale and Shapley's deferred acceptance may be intractable by proving that the problem of deciding if any matching of two sides of a three-sided market can be extended to a stable three-sided matching is $NP$-complete.

\section{Our Contributions and Outline}

\paragraph{}
Our first significant contribution to this body of theory is a simpler view of the linear programming formulation of weight stable matching studied by Vande Vate and Rothblum. We present a short proof that Rothblum's formulation has integer-valued vectors for extreme points which uses only some elementary theory of polytopes and Conway's famous lattice theorem for stable matchings. Our proof is inspired by the common paradigm of iterative rounding \cite{lau2011iterative}, yet takes a slightly different direction as you will see in Chapter $3$. We are hopeful that our proof technique can be used in the investigation of linear programming formulations for other stable matching type problems in the future since its simlicity leaves it wide open for extension.
\paragraph{}
Our other contributions are in the study of cyclic complete three-sided stable matchings as proposed by Knuth. We began our study in hopes of applying modern computing power to push the best known size for which it was known there is always a stable matching higher. As we previously mentioned it currently stands at $4$ and that result belongs to Eriksson and his collaborators. We propose a computer search framework for testing all instances of the problem for a given size. The framework is interesting because it allows for the cutting off of many instances simultaneously by not needing to explore their full preferences. It does this through the use of lemmas which are sufficient for proving existence of stable matchings, and eliminating from consideration instances that are "symmetric"(in a sense we will explain in Chapter $4$) to previously checked instances. 
\paragraph{}
We provide some new lemmas which are sufficient for concluding that an instance of the three-sided problem have stable matchings. Many of these lemmas are about finding a particular partial matching and applying "induction" to get a stable matching in a smaller instance. Other lemmas try to satisfy all agents of a given side in a way that they are unwilling to prevent the matching from being stable. In the study of symmetry, we provide a definition of when two instances of the three-sided problem are symmetric and study the structure of equivalence classes under this symmetry.
\paragraph{}
In the next chapter we will give the reader the necessary background to understand our contributions, taking them through many of the the famous results previously mentioned along the way. They will also learn about linear programming and the methods of iterative rounding, and previous structural results in stable matching theory. The two chapters following preliminaries will detail our contributions to firstly polyhedral theory in stable matching, and secondly to three-sided stable matching problems. After that we conclude with a summary of what was studied, and provide many open avenues for the interested reader to begin exploring research problems in the theory of stable matchings based off our contributions in this thesis.
