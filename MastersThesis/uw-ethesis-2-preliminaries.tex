%======================================================================
\chapter{Preliminaries}
%======================================================================
This chapter gives background on the tools used in the main body of work. Most of this material, with possible exceptions being iterative rounding and the structure of stable matching instances, should be familiar to a student with comparable background to a Combinatorics and Optimization undergraduate at the University of Waterloo. We will emphasize the results in the covered areas that we will need later, but it is important to point out that these areas go much deeper than what is written here and so the end of each section includes suggested texts for further reading.

\section{Matching Theory}
\subsection{Graphs and Matchings}
\paragraph{Graphs} A graph $G$ is an ordered pair $(V,E)$ where $V$ is called the vertex set and $E \subseteq \{\{a,b\} : a,b \in V, a \neq b\}$ is called the edge set. The clause $a \neq b$ forbids self-loops and insisting that $E$ is a proper set forbids parallel edges. Some authors choose to work with more generality but for ease of exposition we will not. As defined above, $G$ is an undirected graph, but if one were to change $E$ to a set of ordered pairs in $V \times V$ then $G$ would be called a directed graph. We will work here with undirected graphs in most chapters. When $V$ and $E$ are not explicit we can used $V(G)$ and $E(G)$ to refer to respectively the vertex and edge sets of $G$.
\paragraph{Paths and Adjacency} In a graph $G$, two vertices $a,b \in V(G)$ are said to be adjacent, denoted $a \sim b$, if $ab \in E$ (notice $ab \in E$ is shorthand for $\{a,b\} \in E$). For any vertex $a$ we call $\delta(a) = \{b \in V(G): b \sim a\}$ the neighbourhood of $a$. The degree of $a$, $d(a)$, is defined as $d(a) = |\delta(a)|$. A path $P \subseteq E$ is a set of edges that describe a sequence of vertices $v_0, v_1, \dots, v_n$ such that for any $i \in \{0,\dots, n-1\}$, $v_iv_{i+1} \in P$. That is a path is a set of edges describing a sequence of adjacent vertices. A cycle is a path with $v_0 = v_n$.
\paragraph{Bipartite Graphs} A graph $G = (V,E)$ is said to be bipartite if there exists a partition of $V$ into $V_0, V_1$ such that for every edge $ab \in E$, $a \in V_0$ and $b \in V_1$. We will restrict our attention to bipartite graphs in this section as most of our work on stable matchings presupposes this case.
\paragraph{Matching} Given a graph $G = (V,E)$, a matching on $G$ is any $M \subseteq E$ satisfying that for all $e_1, e_2 \in M$, $e_1 \cap e_2 = \emptyset$. We use $V(M) = \{v \in V: \exists e \in M, v \in e\}$ to denote the set of vertices matched in $M$. Intuitively our definition of matching means that each vertex matched in $M$ (those in $V(M)$) is matched to exactly one "partner" (ie $|\delta(v) \cap M| = 1$). The partner of vertex $v$, denoted $M(v)$, is the vertex such that $vM(v) \in M$. When it is clear from contex we may also use $M$ to refer to the graph "induced" by $M$, by which we mean the graph $(V(M), M)$.
\subsection{Maximum Cardinality Matching}
\paragraph{Problem} One classical problem in Matching Theory is the question of finding a maximum cardinality matching given a graph. This question was first was investigated by Berge in TODOCITE. He gave a characterization of when a matching isa maximum. To understand the theorem statement we will need just a bit more definitions.
\paragraph{Alternating Paths} A path $P$ in graph $G$ is said to be $M$-alternating for matching $M$ if its sequence of edges alternate being in $M$ and not being in $M$. More precisely if we order the edges of $P$ as $e_0, \dots, e_{n}$ then for any $i \in \{0, \dots, n-1\}$, edge $e_i \in M$ if and only if $e_{i+1} \in E \backslash M$. An $M$-alternating path $P$ is called $M$-augmenting if the first and last edges of $P$ are not in $M$. The name purposefully evokes the image of flow augmenting paths in network flow theory as there is a way to increase the cardinality of a matching $M$ by augmenting (taking symmetric difference, defined below) it with an $M$-augmenting path.
\paragraph{Symmetric Difference} For any sets $S,T \subseteq U$ the symmetric difference of $S$ and $T$, denoted $S \triangle T$, is given by $S \triangle T = (S \cup T) \backslash (S \cap T)$. 
\paragraph{Augmenting Path Theorem (Berge TODOCITE)} A matching $M$ in graph $G$ is maximum if and only if there does not exist any $M$-augmenting path $P$.
\paragraph{Proof} First suppose to the contrary that there exists $M$-augmenting path $P$. We claim that $M \triangle P$ is a matching of greater cardinality than $M$. To see that $M \triangle P$ is a matching let $v \in V(M \triangle P)$. If $v \in V(M)$ and $v \not\in V(P)$ then the edge incident upon $v$ has not changed, and no new edges incident to $v$ were added by symmetric difference with $P$. If $v \in V(M)$ and $v \in V(P)$ then as $P$ is an $M$-augmenting path, $vM(v) \in P$. Hence $vM(v) \notin M \triangle P$, and further there is $u \in V(G)$ such that $vu \in P\backslash M$. Thus $|\delta(v) \cap (M\triangle P)| =1$) as desired. Lastly if $v \not\in V(M)$ then $v$ is one of the endpoints of $P$ and is matched along its edge in $P$. So $M \triangle P$ is a matching. Further since every vertex in $V(M)$ is matched in $M \triangle P$, and the start and end vertices of $P$ are also matched, $M \triangle P$ matches more vertices that $M$ and thus is of greater cardinality.
\paragraph{}
Now suppose that $M$ is a matching which is not of maximum cardinality. That is there exists some matching $M'$ such that $|M'| > |M|$. Consider their symmetric difference $J = M' \triangle M$. Observe that each vertex in the graph $(V(G), J)$ has degree at most two (attaining this if it is matched along difference edges in $M$ and $M'$). Therefore $(V(G), J)$ consists only of vertex disjoint paths and cycles. The edges of said paths and cycles alternate belonging to $M'$ and to $M$ (observe that if this claim were to fail there would be a vertex with two edges incidenct upon it in the same matching, a contradiction). So the cycles are even in number of edges and contain the same number of edges from each of $M'$ and $M$. But since $|M'| > |M|$ there is a path with more edges in $M'$ than in $M$, $P$. This follows from counting edges in $M$ and $M'$ noticing that cycles contribute the same number to each. The path $P$ is $M$-augmenting. $\blacksquare$
\paragraph{}
We cover this proof not only because it is instrinsically interesting, but also because the structure of $J$, the symmetric difference of two matchings, will arise in the future when we study the structure of stable matchings.
\paragraph{Further Reading}
This problem is very well understood. For instance Tutte's classic min-max theorem TODOCITE, and Edmond's Blossom algorithm TODOCITE. A textbook appropriate for advanced undergraduate or beginner graduate students is Combinatorial Optimization by Cook, Cunningham, Pulleybank, and Schrijver TODOCITE which contains a chapter covering the results mentioned here.
\subsection{Maximum Weight Matching}
\section{Iterative Rounding}
\subsection{Linear Programming Tools}
\subsection{Strategy}
\subsection{Application to Matching}
\section{Stable Matching}
\subsection{Problem}
\subsection{Gale Shapley Algorithm}
\subsection{Structure}

