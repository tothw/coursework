%======================================================================
\chapter{Cyclic Stable Matching}
%======================================================================
This chapter describes the cyclic stable matching problem as posed by Knuth. We give some sufficient conditions for a matching to be stable in this context. Then we proceed to describe a computer search procedure for deciding if all preference systems of a certain size have a cyclic stable matching or finding a counterexample should one exist. 
\section{The Problem}
\paragraph{}
In Knuth's writing on stable matching \cite{knuth1997stable} he talks about some extensions of the classical stable matching problems as discussed so far. One such avenue for generalization is increasing the number of "genders", which is to say increase instead of matching pairs, we consider matching triples in tripartite graphs or matching $k$-tuples in $k$-partite graphs. We will make this notion formal through the terminology of hypergraphs.
\begin{definition}
We $G=(V,E)$ is a hypergraph if $V$ is some discrete set and $E \subseteq 2^V$ ($2^V$ denotes the set of all subsets of elements in $V$). We can think of adjacency in this case as $v \sim w$ provided there exists $e \in E$ such that $v, w \in e$. The notions of degree $d$ and edges incident upon a vertex $\delta$ extend naturally. A matching in this context is $M \subseteq E$ such that for all $m_1, m_2 \in M$, $m_1 \cap m_2 = \emptyset$. We call $G$ $k$-uniform if for all $e \in E$, $|e| = k$. We call $k$-uniform $G$ $k$-partite if there exists $V_1, \dots, V_k$ which partition $V$ and for all $e=(v_1, \dots, v_k) \in E$, $v_i \in V_i$ for $i = 1,\dots, k$. Lastly we say $k$-partite graph $G$ is balanced if $\size{V_1} = \dots = \size{V_k}$.
\end{definition}
\paragraph{}
With our description hypergraphs complete we are ready to describe a generalization of stable matching.
\begin{definition}
In an instance of cyclic $k$-gender stable matching ($CkGSM$), we are given a $k$-partite hypergraph $G=(V_0 \cup \dots \cup V_{k-1}, E)$ and for each $i \in \{0,\dots,k-1\}$, each $v_i \in V_i$ is equipped with a preference order over $V_{i+1} \cap V(\delta(v_i))$ where addition of indices is taken modulo $k$ (so $V_{k-1}$ vertices have preferences over $V_0$ vertices, hence the descriptor cyclic). That each vertex in a given gender has preferences over vertices acceptable to them in another gender. We say our instance, and naturally the hypergraph $G$, is complete if for all $v_0 \in V_0$, $\dots$, $v_{k-1} \in V_{k-1}$, $(v_0,\dots,v_{k-1}) \in E$ (all possible tuples are acceptable to use in matchings). A solution to an instance of $CkGSM$ is a matching $M \subseteq E$ that is stable in the sense that no blocking tuple exists. A tuple $e=(v_0,\dots, v_{k-1}) \in E$ blocks $M$ if $e\not\in M$ and for each $i \in \{0,\dots, k-1\}$, $v_{i+1} >_{v_i} M(v_i)$ where $M(v_i)$ denotes the vertex $v_{i+1} \in V_{i+1}$ such that there exists an edge $m \in M$ for which $v_i, v_{i+1} \in m$.
\end{definition}
\begin{note}
We may assume that instances of $CkGSM$ are balanced. To see this observe that adding dummy vertices does not affect the solutions. Thus if $\size{V_i} < \size{V_j}$ we can add $\size{V_j} - \size{V_i}$ dummy vertices to $V_i$ to balance their sizes. If incomplete preferences are allowed then dummy vertices can simply have no incoming edges. If preferences are required to be complete then we may simply place dummy vertices as the least preferred vertices among the relevant preference lists and give the dummy vertices arbitrary orders over "real" vertices they face. We that an instance of $CkGSM$ is of size $n$ if it is balanced with each $V_i$ having size $n$.
\end{note}
\paragraph{}
We will focus on instances where $k=3$ to simplify things somewhat. Knuth asked if instances of $C3GSM$ always have solutions. To this day this question is unresolved for complete instances. Some, but not many, results are known and we will present them in the next section.
\begin{conjecture}\label{conj:stab}
For all sizes $n$, instances of $C3GSM$ of size $n$ have a stable matching.
\end{conjecture} 
\section{Known Results}
\paragraph{}
First we will present a counterexample that demonstrates that when instances are not complete there need not always be a stable matching solution. This example is due to Biro et al \cite{biro2010three} and shows that when you have incomplete preferences there is a hypergraph with each $V_i$ have size $6$ and preference lists that admit no stable matching.
\begin{note}\label{ex:R6}
For our counterexample consider the tripartite hypergraph $R6 = (A \cup B \cup C, E)$ where $$A = \{a_1,a_2,a_3,a_1',a_2',a_3'\},\ B = \{b_1,b_2,b_3,b_1',b_2',b_3'\},\ \text{and}\ C = \{c_1,c_2,c_3,c_1',c_2',c_3'\}.$$ Below we present the preference orders of each vertex, where vertices omitted from a list are deemed unacceptable and cannot form edges with the given vertex whose list is under consideration:
\begin{align*}
a_0 &: b_0 > b_0' &b_0: c_0 > c_0' &\ &c_0:a_1 > a_1'\\
a_1 &: b_1 > b_1' &b_1: c_1 > c_1' &\ &c_1:a_2 > a_2'\\
a_2 &: b_2 > b_2' &b_2: c_2 > c_2' &\ &c_2: a_0> a_0'\\
a_0' &: b_2 &b_0': c_2 &\ &c_0':a_0 \\
a_1'&: b_0 &b_1': c_0 &\ &c_1':a_1\\
a_2' &:b_1 &b_2': c_1 &\ & c_1':a_2.
\end{align*}
We will adopt the notation of the original authors calling the agents $\{a_i, b_i, c_i: 1 \leq i \leq 3 \}$ the inner agents and $\{a_i',b_i',c_i': 1\leq i \leq 3\}$ the outer agents.
\end{note}.
\begin{lemma}
The problem instance of $C3GSM$ described in note \ref{ex:R6} has no stable matching.
\end{lemma}
\begin{proof}
Let $M$ be a matching in $R6$. We observe that the possible triples in $M$ have very particular forms. Let $m \in M$ be a triple. Suppose that inner vertex $a_i \in m$. Then by $a_i$'s preferences either $b_i \in m$ or $b_i' \in m$. If $b_i \in m$ then as $a_i$ is not acceptable to $c_i$, the triple $m$ is of the form $a_ib_ic_i'$. If $b_i' \in m$ then $c_{i-1} \in m$ (recall index addition and subtraction is done modulo $3$) and hence the triple $m$ is of the form $a_i b_i'c_{i-1}$. Otherwise if outer vertex $a_i' \in m$ then the triple $m$ is of the form $a_i'b_{i-1}c_{i-1}$. That is the only triples in $M$ are those of the form
$$a_ib_ic_i' \quad\text{or}\quad a_ib_i'c_{i-1} \quad\text{or}\quad a_i'b_{i-1}c_{i-1}.$$
Then triples in $M$ contain exactly two inner agents and one outer agent each. Since each agent can only be matched at most once, and inner agents are matched in pairs then $M$ matches an even number of inner agents. But there are an odd number ($9$) of inner agents. Hence at least one inner agent is unmatched in $M$. By the symmetry of the problem instance we may assume without loss of generality that this inner agent is $a_i$. Consider the triple $T=(a_i, b_{i}', c_{i-1})$. If $c_{i-1}$ is matched in $M$ then, as $a_i$ is unmatched, $M(c_{i-1}) = a_i'$. Hence $a_i >_{c_{i-1}} M(c_{i-1})$. Again since $a_i$ is unmatched, $b_{i}'$ is unmatched and thus prefers $T$ to $M$. Hence the triple $T$ blocks $M$, and since $M$ was an arbitrary matching no matching of $R6$ is stable.
\end{proof}
\paragraph{}
Since we have a counterexample to stability in the case where incomplete preferences are allowed this motivates us to restrict the problem to the situation where problem instances are complete. It is in this setting that we will work going forward. As of yet we are not aware of any counterexample instance that has no stable matching for complete $C3GSM$, nor of any proof that all instances admit a stable matching.
\paragraph{}
One natural angle towards a proof would be to attempt to extend the Deferred-Acceptance algorithm of Gale and Shapley to $C3GSM$. Farczadi et al provide some insight into the feasibility of this route in \cite{farczadi2014stable}. Here they claim a particular approach to extending Deferred-Acceptance is $NP$-complete. For readers without a complexity theory background this result gives strong evidence that there is no polynomial time algorithm for the approach to extending Deferred-Acceptance that they examine.
\begin{theorem}
(Farczadi et al): Consider an instance of complete $C3GSM$ with hypergraph $G=(A\cup B\cup C, E)$. Fix a perfect matching of $A$ to $B$. The problem of deciding if this matching can be extended to a stable matching of $G$ is $NP$-complete.
\end{theorem}
\paragraph{}
So in lieu of algorithmic insight into the problem, researchers have presented ad-hoc methods to demonstrate when instances of $C3GSM$ are stable for fixed sizes $n$. It has been shown by Eriksson et al \cite{eriksson2006three} that complete instances of size $n \leq 4$ always have a stable matching. First observe that cases of size $n \leq 2$ are not interesting as blocking triples are impossible since a blocking triple must draw each of its vertices from a distinct triple in the matching (no vertices in instances of $C3GSM$ are unmatched in a stable matching, as they would come in threes and prefer to be together than unmatched). The case for $n=3$ is straightforward and requires a bit of case analysis, but Eriksson et al have shown something stronger that we will prove instead. Before we state the lemma, we will give some quick notation that will make discussing the lemma and proof much more convenient.
\begin{definition}
Let $G=(A\cup B \cup C, E)$ be the hypergraph for an instance of $C3GSM$. We will define the following functions for $A$ but we will use analogous versions for $B$ and $C$ throughout. Let $S\subseteq B$. Let $a \in A$. Then $f(a, S) = b$ provided $b \in B$ and for all $b' \in B \backslash S$, $b>_a b'$. In other words $f(a,S)$ maps $a$ and $S$ to the most preferred vertex in $B$ ignoring those in $S$. Let $s \in \{1,\dots,n-1\}$. If $s=1$ define $f(a,s) = b$ provided $f(a,\{b\}) = b$. If $s>1$ define $f(a,s) = b$ provided $f(a,\{b\}\cup\{f(a,t): t\in\{1,\dots,s-1\}) = b$. Intuitively $f(a,s)$ is $a$'s $s$-th choice among vertices in $B$.
\end{definition}
\begin{definition}
Let $G = (A \cup B \cup C, E)$ be the hypergraph for an instance of $C3GSM$. We say $G$ has an $i,j,k$ triple if there exist $a \in A$, $b\in B$, and $c \in C$ such that $f(a,i) = b$, $f(b,j) = c$, and $f(c,k) = a$. Of particular interest are $1,1,1$ triples. If a matching contains a $1,1,1$ triple then none of the vertices in the triple will participate in any blocking triple against the matching.
\end{definition}
\begin{lemma}\label{lemma:n3}
(Eriksson et al) \cite{eriksson2006three}: Let $G=(A\cup B \cup C, E)$ be the hypergraph for an instance of $C3GSM$ of size $3$. Let $a \in A$. Then $G$ has a stable matching $M$ where either $M(a)= f(a,1)$ or, $M(a) = f(a,2)$ and $M(b) = f(b,1)$ where $b = f(a,1)$.
\end{lemma}
\begin{proof}
\paragraph{} First consider the case where $G$ has a $1,1,1$ triple $(a',b',c')$. Construct the matching $M$ as follows. Add $a'b'c'$ to $M$. Now $a',b',c'$ will not participate in any blocking triples against $M$. Hence there are only $6$ vertices remaining, so regardless of how we choose the next two triples for $M$ the resulting matching will be stable. If $a = a'$ then we are done, so we may assume $a\neq a'$. If $b = b'$ then complete $M$ arbitrarily as long as $M(a) = f(a,2)$. Otherwise, $b\neq b'$ and we may complete $M$ arbitarily ensuring $M(a) = b$.
\paragraph{} Now consider the case where $G$ has no $1,1,1$ triple. Let $c = f(b,1)$. Since there is no $1,1,1$ triple, $a \neq f(c,1)$. Let $a' = f(c,1)$. Again since there is no $1,1,1$ triple $b \neq f(a',1)$. Let $b' = f(a',1)$. Similarly $c \neq f(b',1)$ and et $c' = f(b',1)$. Let $a'', b'', c''$ be the remaining vertices in $A,B,C$ respectively. Form the matching $M = \{abc,a'b'c',a''b''c''\}$. We claim $M$ is stable. Since $a$ and $a'$ were matched to their first choices they will not participate in any blocking triple. Hence $a''$ is the only vertex in $A$ willing to participate in a blocking triple. But $a'$ is only unmatched to $b$ and $b'$, both of which are matched to their first choices and hence are unwilling to participate in any blocking triple. Thus there are no blocking triples against $M$ and hence $M$ is stable. Further since $M(a) = b$ we are done. 
\end{proof}
\paragraph{}
Using lemma \ref{lemma:n3} Eriksson et al show the following theorem. We ommit the proof as, by the authors own admission, it is a technical case analysis.
\begin{theorem}(Eriksson et al) \label{theorem:n4} \cite{eriksson2006three}: The conjecture \ref{conj:stab} holds for $n\leq 4$.
\end{theorem}
\section{Our Contributions}
\paragraph{}
In this section we will describe some work done towards resolving if $C3GSM$ instances always have solutions for higher sizes $n$. We present first some sufficient conditions for problem instances to have stable matchings. Next we study the symmetry of problem instances and describe when problem instances are equivalent with respect to having a stable matching. All this work is building towards a proposal for a computer search procedure to decide if all instances of $C3GSM$ for a given size $n$ have a stable matching, or present a counterexample if they do not.
\subsection{Sufficient Conditions}\label{subsec:sufficient}
\begin{definition}
Let $G$ be the hypergraph for an instance of $C3GSM$, and let $M$ be a matching in $G$. We denote by $\cB(G,M)$ the set of blocking triples in $G$ against $M$. Formally $\cB(G,M) = \{abc \in E(G): b>_a M(a), c>_b M(b), a>_c M(c)\}$.
\end{definition}
\begin{note}
If we let $S \subseteq V(G)$ and denote by $G[S]$ the hypergraph $G$ restricted to vertices in $S$ and edges between vertices in $S$ ($G[S]$ denotes the subgraph induced by $S$) then we may use $\cB(G[S],M)$ to describe the blocking triples of $G$ among vertices in $S$. In particular if we wish to study blocking triples present among agents matched in $M$ we study the set $\cB(G[V(M)], M)$.
\end{note}
\begin{lemma}\label{lemma:partialstab}
Let $G = (A\cup B \cup C, E)$ be the hypergraph for an instance of $C3GSM$ of size $n$ where $n-1$ is the best known size such that  satisfies conjecture \ref{conj:stab}. Suppose that $G$ has a matching $M$ (note $M$ need not be perfect) for which $\cB(G[V(M)],M) = \emptyset$. If for all  $v \in V(M)$, for all $w \in V(G)$ such that $w >_v M(v)$, we have that $w \in V(M)$ then $G$ has a stable matching.
\end{lemma}
\begin{proof}
\paragraph{}
Let $G' = G[V(G)\backslash V(M)]$ be the hypergraph obtained by removing vertices matched in $M$ from $G$. Consider the instance of $C3GSM$ with underlying hypergraph $G'$ and preference orders obtained from $G$. Since the size the instance with $G' \leq n=1$ we can find a stable matching of $G'$, call it $M'$. Consider the perfect matching in $G$ given by $S = M \cup M'$. We claim that $S$ is stable for $G$. Suppose for a contradiction that there exists $(a,b,c) \in \cB(G, S)$. Since $M'$ is stable for $G'$ and $\cB(G[V(M)], M) = \emptyset$, the blocking triple $(a,b,c)$ must use both agents matched in $M$ and $M'$. That is, at least one of $a,b,c \in V(M')$ and at least one of $a,b,c \in V(M)$. Bear in mind $V(M) \cap V(M') = \emptyset$. Without loss suppose $a \in V(M)$. Since $(a,b,c)$ is a blocking triple, $b >_a S(a) = M(a)$. Then by our hypothesis $b$ is matched in $M$. Again, since $(a,b,c)$ is a blocking triple, $c >_b S(b) = M(b)$ and by our hypothesis $c$ is matched in $M$. So $a,b,c \in V(M)$, a contradiction as this implies $a,b,c \not\in M'$ violating that at least one of $a,b,c \in M'$.
\end{proof}
\begin{corollary}\label{cor:1cycle}
Let $G$ be the hypergraph for an instance of $C3GSM$ and $n$ be as in lemma \ref{lemma:partialstab}. If there exists $a,b,c \in V(G)$ such that $b = f(a,1)$, $c = f(b,1)$, and $a = f(c,1)$ then $G$ has a stable matching.
\end{corollary}
\begin{proof}
\paragraph{}
Invoke lemma \ref{lemma:partialstab} with $M = \{(a,b,c)\}$.
\end{proof}
\begin{corollary}
Let $G=(A\cup B \cup C, E)$ be the hypergraph for an instance of $C3GSM$ with $n$ as in lemma \ref{lemma:partialstab}. If one of $A,B,C$, say without loss $A$, is such that for all $a,a' \in A$, $f(a,1) = f(a',1)$ then $G$ has a stable matching.
\end{corollary}
\begin{proof}
\paragraph{}
Let $a_1, \dots, a_n \in A$. Suppose that $b = f(a_1,1) =  \dots = f(a_n,1)$. Let $c = f(b,1)$. Then there exists some $a_k$ such that $a_k = f(c,1)$. Since $a_k \in A$ we may invoke corollary \ref{cor:1cycle} with $(a_k, b, c)$ to complete the proof.
\end{proof}
\begin{lemma} Let $G=(A\cup B \cup C, E)$ be the hypergraph underlying an instance of $C3GSM$ of size $n$ where $n-1$ is the best known size satisfying conjecfture \ref{conj:stab}. Let $M$ be a matching in $G$. Suppose that for all $b\in B\cap V(M)$ and $c\in C\cap V(M)$, $b,c \not\in V(\cB(G,M))$. Also suppose that for all $a \in A\cap V(M)$ if $abc \in \cB(G,M)$ then $b \not\in V(M)$. If $b = f(a,1)$ for all $a \in A\backslash V(M)$ then $G$ has a stable matching.
\end{lemma}
\begin{proof}
\paragraph{}
Let $c = f(b,V(M))$ be $b$'s first choice unmatched in $M$. Let $a=f(c,V(M))$. Then $b = f(a,1) = f(a,V(M))$. Let $T = (a,b,c)$. Observe that for all $v \in V(M \cup \{T\})$,  $v \not\in V(\cB(G,M\cup\{T\})$. Consider the hypergraph $G' = G[V(G)\backslash \{V(M \cup \{T\}\}]$ (the graph $G$ removing vertices matching in $M \cup\{T\}$) with preference orders derived from $G$. Since the size of the $C3GSM$ instance with $G'$ is less than $n$, $G'$ has a stable matching $M'$. Let $S = M \cup \{T\} \cup M'$. Then $S$ is a perfect matching of $G$. We claim that $S$ is stable in $G$. Since $M'$ is stable in $G'$, for all $abc \in \cB(G,S)$, there exists $v \in V(M \cup \{T\})$, with $v \in abc$. But such $v \not\in V(\cB(G,M\cup\{T\}) \supseteq \cB(G,S)$. Hence no such $v$ form blocking triples and therefore $S$ is stable. 
\end{proof}
\begin{lemma}
Let $G=(A\cup B \cup C, E)$ be the hypergraph underlying an instance of $C3GSM$ of size $n$. Let $M$ be a matching in $G$. Suppose that for all $c \in C \cap V(M)$, $c \not\in V(\cB(G, M))$, and there exists $b \in B \backslash V(M)$ such that for all $a \in A \cap V(M)$, if $a\in e \in \cB(G,M)$ then $b \in e$, and $b = f(a',V(M))$ for all $a'\in A \backslash V(M)$. Suppose also that there exists $c \in C\backslash V(M)$ such that for all $b \in B \cap V(M)$, if $b \in e \in \cB(G,M)$ then $c \in e$. If $\size{M} \geq n-3$ then $G$ has a stable matching. Intuitively this lemma says that if there is a matching large enough where all agents in $A$ form their blocking triples through some $b \in B$ outside $M$ that is the first choice of all unmatched $a \in A$, and that there is an unmatched $c \in C$ that all matched agents in $B$ form their blocking triples through then there exists a stable matching in this instance.
\end{lemma}
\begin{proof}
\paragraph{}
We may assume without loss that $\size{M} = n-3$, as $|M| \geq n-2$ is trivial. Let $c' = f(b, V(M))$ be the first choice of $b$ unmatched in $M$. Let $a' = f(c, V(M))$. If $c' = c$ then adding the triple $(a', b,c)$ to matching $M$ stabilizes $M(A), M(B), M(C)$ and $a', b, c'$ in the sense that they form no blocking triples. Formally $M(A),M(B),(C), a',b,c' \not\in V(\cB(G, M \cup \{ a'bc\}))$. Any matching of the unmatched agents extends this to a stable matching of $G$. So consider when $c' \neq c$. Let $a = f(c, V(M) \cup\{a'\})$ be first choice of $c'$ among agents not matched in $M$ and not equal to $a'$. Form the triple $T = (a, b, c')$. Then $b$ is stable for having $c'$, $a$ is stable for having $b$, and $M(A)$ is stable for $b$ being stable. It is possible that $c'$ is unstable through $a'$ if $a' >_{c'} a$ (otherwise $c'$ is stable). Form the triple $T' = (a', b', c)$ where $b'=f(a',V(M(B))\cup\{b\})$. Then $a'$ is is stable as $b$ and $M(B)$ are stable, and $c$ is stable for having $a'$. Hence $c'$ is stable for having $a$ with $a'$ stable. Form the complete matching $S$ from $M$, $T$, $T'$ and the triple matching the three remaining unmatched agents. The only unstable agents under $S$ are $b'$ and possibly agents among the last three to be matched, but that is not a sufficient number of agents enough to form a blocking triple and hence $S$ is a stable matching for $P$. 
\end{proof}
\paragraph{Stabilizing $A,B,C$}
The previous lemmas have the common theme of building matchings that have no blocking triples in the hypergraph then using induction to make the matching perfect. The following lemma and corollary take a different strategy. Whereas before we tried to create a matching where matched vertices have no blocking triples even with unmatched vertices, alternatively we can attempt to pair all the vertices in one of $A,B,C$ with a partner in such a way that none of them will participate in a blocking triple.
\begin{lemma}\label{lemma:genderstab}
Let $G=(A\cup B\cup C, E)$ be the hypergraph underlying an instance of $C3GSM$. Let $M$ be a perfect $A$-$B$ matching (that is a perfect matching of the underlying complete bipartite graph between $A$ and $B$). Let $B' = \{ b \in B : \exists a \in A : b >_a M(a) \}$. Hence $B'$ is the set of vertices through which agents of $A$ can form blocking triples. If for all $b, b' \in B'$, $f(b,1) \neq f(b',1)$ then $G$ has a stable matching.
\end{lemma}
\begin{proof}
\paragraph{}
Let $C' = \{ c \in C : \exists b\in B', c = f(b,1)\}$. Since each $f(\cdot,1)$ is a bijection between $B'$ and $C'$ we can form the matching $S_1 = \{abc \in E: a = M(b)\text{ and } c = f(b,1)\}$. Let $S_2$ be an arbitrary perfect matching of $G[V(G) \backslash V(S_1)]$, the remaining vertices unmatched in $S = S_1 \cup S_2$. Then $S$ is a complete matching on $G$. We claim that $S$ is stable. Suppose for a contradiction there exists blocking triple $a,b,c$. Then $b >_a M(a)$ and hence $b \in B'$. Thus $M(b) = f(b,1)$ and so $M(b) \geq_b c$ by definition of $f(\cdot,1)$, but this contradicts that $a,b,c$ is a blocking triple since that requires $c >_b M(b)$.
\end{proof}
\begin{corollary}
If $f(a,1) \neq f(a',1)$ for all $a,a' \in A$ in a hypergraph $G=(A\cup B \cup C, E)$ underlying a $C3GSM$ instance then $G$ has a stable matching.
\end{corollary}
\begin{proof}
\paragraph{}
Observe that $M = \{ab : b = f(a,1)\}$ is a perfect $A$-$B$ matching since $f(\cdot, 1)$ is a bijection between $A$ and $B$. In this case we have $B' = \emptyset$ by the definition of $f(\cdot,1)$. Thus we may invoke lemma \ref{lemma:genderstab} to obtain a stable matching of $G$.
\end{proof}
\begin{lemma}\label{lemma:fixing}
Let $G = (A\cup B \cup C, E)$ be the hypergraph underlying an instance of $C3GSM$ of size $n$. Let $M$ be a matching in $G$ with $\size{M} = n-2$. Suppose there exists $a \in A \cap V(M)$ such that for all $e \in \cB(G, V(M))$ if there exists $v \in V(M)$ such that $v \in e$ then $v = a$. That is, $a$ is the only vertex matched in $M$ that forms blocking triples against $G$. If there exists $b \in B\backslash V(M)$, for all $b' \in B$, $b' \geq_a b$ (that is, the last choice of $a$ is unmatched in $M$) then $G$ has a stable matching.
\end{lemma}
\begin{proof}
\paragraph{}
By the size of $M$ there exists $b,b' \in B$ such that $\{b,b'\} = B \backslash V(M)$. Without loss say that $b$ is the last choice of $a$. Let $c' = f(b', V(M))$ be the first choice of $b'$ not matched in $M$. Let $a' \in A\backslash V(M)$. Form triple $T = a'b'c'$ and let $T'$ be the triple containing $b$ and the remaining two vertices unmatched in $M$ or $T$. Then $S = M \cup \{T, T'\}$ is a perfect matching of $G$. We claim that $S$ is stable.
\paragraph{}
Suppose for a contradiction that there exists $e \in \cB(G,S)$. First consider the case where for all $v \in V(M)$, $v \not\in e$. Then $e$ uses only vertices of the triples $T$ and $T'$, but that is an insufficient number of vertices to form a blocking triple. Now consider when there exists $v \in V(M)$, $v \in e$. Since $M \subseteq S$, $\cB(G,S) \subseteq \cB(G,M)$ and hence $e \in \cB(G,M)$. So by our assumptions $a=v$. Since all $v' \neq a \in V(M)$ are such that $v' \not\in e$. We have either $b \in e$ or $b' \in e$. Since $a \in e$, $b$ is the last choice of $a$, and $e$ is a blocking triple, we know $b \not\in e$ (otherwise $b >_a S(a)$ which is a contradiction as $S$ is a perfect matching). Thus $b' \in e$. Let $c \in e \cap V(C)$. Now since $S(b') = c' = f(b',V(M))$ we have $c >_{b'} c'$. Then by our choice of $c'$, $c \in V(M)$. But this contradicts that $c \not\in e$. Therefore $S$ is stable.
\end{proof}
\subsection{Symmetry in Problem Instances}\label{subsec:symmetry}
\paragraph{}
We will now turn our attention towards formulating an equivalence relation on instances of $C3GSM$. In doing so we hope to answer the question ``when does knowledge that one instance of $C3GSM$ has a stable matching translate to knowing that another instance does?". This question is motivated by our efforts to design a computer search protocol in the next subsection. In checking if all $C3GSM$ instances of a certain size have a stable matching it is desirable to avoid repeated work by not checking instances``symmetric" to one already checked.
\begin{definition}
Let $G$ and $H$ be two hypergraphs underlying distinct instances of $C3GSM$. We say that $G$ and $H$ are equivalent in the sense that they denote symmetric problem instances, written $G \equiv H$ if there exists a bijection $\phi: V(G) \rightarrow V(H)$ such that for all $v \in V(G)$,
 \begin{equation}\label{cond:order}
 w >_v u \text{ if and only if } \phi(w) >_{\phi(v)} \phi(u).
 \end{equation}
 \end{definition}
\begin{note}
It is clear that $\equiv$ is an equivalence relation. Through the identity bijection we have reflexivity, through the inverse bijection we have symmetry, and through composition of bijections we have transitivity. Thus $\equiv$ is an equivalence relation as desired.
\end{note}
 \begin{lemma}
 If $G \equiv H$ then $G$ has a stable matching if and only if $H$ has a stable matching
 \end{lemma}
 \begin{proof}
 \paragraph{}
 By symmetry it suffices to prove the sufficiency direction. Let $M$ be a stable matching of $G$. Define the matching $\phi(M)$ in $H$ as
 $$\phi(M) = \{ \phi(a)\phi(b)\phi(c): abc \in M \}.$$
 It is not hard to see that since $M$ is a matching and $\phi$ is a bijection that $\phi(M)$ is a matching. Suppose for a contradiction that $\phi(M)$ is not stable for $H$. Let $xyz$ be a blocking triple of $\phi(M)$ against $H$. Let $abc = \phi^{-1}(x)\phi^{-1}(y)\phi^{-1}(z)$. By our definition of $\phi(M)$, $abc \not\in M$. Further since $y >_x \phi(M)(x)$, by the definition of $\phi$ we have $b >_a M(a)$. Similarly $c>_b M(b)$ and $a>_c M(c)$. Thus $abc$ is a blocking triple of $M$, a contradiction.
 \end{proof}
 \begin{note}\label{note:labels}
 Since we require that $\phi$ is a bijection it is necessary that $G$ and $H$ underly instances of $C3GSM$ of the same size, say $n$. Let $V(G)$ be partitioned as $V(G)_1 \cup V(G)_2 \cup V(G)_3$ since $G$ is tripartite. Let $i \in \{0,1,2\}$. Since each $V(G)_i$ is of size $n$ we may label the elements of $V(G)_i$ as $(i,0), \dots, (i,n-1)$. We may induce the same labelling on $V(H)$. Then $\phi$ is a bijection from $Z_3 \times Z_n$ to itself which preserves the each vertex's order of its neighbours (and consequently the graph incidence structure). So we can say a few things about $\phi$ from this perspective.
 \end{note}
 \paragraph{Observations} Firstly by $\ref{cond:order}$ we must preserve the graph structure. Hence if $\phi(a,i) = \phi(b, j)$ then for all $k \in \{0,\dots,a-1,a+1,\dots,n-1\}$ there exists some $\ell$ such that $\phi(a,k) = \phi(b,\ell)$. That is to say, $\phi$ maps elements of all elements of one part to a single other part. Furthermore if $\phi(a,\cdot) = \phi(b,\cdot)$ then $\phi(a+1,\cdot) = \phi(b+1, \cdot)$ with addition taken modulo $3$. Again this follows from condition \ref{cond:order}.
\paragraph{}
The previous observations lead us to see that there exists $r \in \{0,1,2\}$ such that $\phi(a, \cdot) = \phi(a+r,\cdot)$. Hence we can specify $\phi$ by $(r,\Pi)$ where $\Pi = \{\pi_0, \pi_1,\pi_2\}$ is a family of three permutations $\pi_i$ on $\Z_5$, one for each part $V(G)_i$. In specifying $\phi=(r,\Pi)$ we require that $(r,\Pi)$ satisfying a translation of condition \ref{cond:order} for all $(i,a) \in V(G)$
\begin{equation}\label{cond:orderT}
(i+1,b) >_{(i,a)} (i+1, b') \text{ if and only if } (i+1+r, \pi_{i+1}(b)) >_{(i+r,\pi_i(a))} (i+1+r, \pi_{i+1}(b')).
\end{equation}
\paragraph{}
Of interest to us are the equivalence classes of a given problem instance under $\equiv$. For a hypergraph $G$ underlying an instance of $C3GSM$ we will denote the equivalence class containing $G$ by $[G]$. In a procedure to test whether instances of $C3GSM$ of a certain size have stable matchings one can hope to avoid unnecessary repeated work by only testing one representative from $[G]$ instead of all instances in $[G]$. We will now define a strict total order on instances of $C3GSM$ of size $n$, with the goal of obtaining a method to test only the minimum element with respect to this order for each equivalence class $[G]$.
\begin{definition}
We will use the symbol $>_{lex}$ to denote our lexicographical order of instances of $C3GSM$ of a given size $n$. Let $G$ and $H$ be hypergraphs underlying size $n$ instances of $C3GSM$. We will translate $G$ and $H$ to finite sequences and apply the natural lexicographical order to said sequences. The sequence $seq(G)$ will consist of appending sequences for each of its vertex partitions as $seq(G) = seq(G,0)seq(G,1)seq(G,2)$ to be defined as follows. Let $i \in \{0,\dots, 2\}$.  We define $seq(G,i)$ by appending sequences for each vertex of $V(G)_i$. Formally $seq(G,i) = seq(G,i,0)\dots seq(G,i,4)$ where for each $j \in \{0,\dots, n-1\}$,
$$seq(G,i,j) = f((i,j),1)f((i,j),2)\dots f((i,j),n-1).$$
So for instance $seq(G,0, 0)$ is obtained by appending the first through $n$-th choices of vertex $0$ in $V(G)_0$. We sometimes will refer to $seq(G,0,0)$ as $(0,0)$'s So $seq(G)$ consists of appending the preference lists of vertices $0$ through $n-1$ of $V(G)_0$ followed by appending the same for $V(G)_1$ and finally $V(G)_2$.
We say $G >_{lex} H$ is $seq(G)$ is lexicographically larger than $seq(H)$. That is there exists some $k$ such that
$$seq(G)_k > seq(H)_k$$
and for all $i < k$
$$seq(G)_i = seq(H)_i.$$
We compare vertices $seq(G)_k$ and $ seq(H)_k$ as if they are natural numbers corresponding to their labels. That is if $seq(G)_k = (i,a)$ and $seq(H)_k = (j,b)$ then we simply say $seq(G)_k > seq(H)_k$ if and only if $a>b$. 
\end{definition}
\paragraph{}
Unfortunately we do not have a full characterization of the lexicographic minimum of a given equivalence class under the order $>_{lex}$, which would be desirable. Instead we do have some necessary conditions which we collect in the following lemma. These are useful for finding instances of $C3GSM$ that we can avoid testing in a computer search since we are certain we are at least testing the lexicographic minimum of each equivalence class.
\begin{lemma}\label{lemma:nec-symmetry}
Let $G$ be the hypergraph underlying an instance of $C3GSM$ of size $n$. If $G$ is the lexicographic minimum of $[G]$ then the following are satisfied:
\begin{enumerate}
\item $seq(G,0,0) = 0,1,\dots,n-1$,
\item $seq(G,1,0) = 0,1,\dots,4$,
\item let $a, b \in [0,n-1] \cap \Z$ and let $k \in [1,n] \cap \Z$. If $f((0,a), k) < f((0,b), k)$ and for all $j \in [0,k) \cap \Z$, $f((0,a),j) =f((0,b),k)$ then $a < b$,
\item $seq(G)$ is lexicographically at most $seq(G,1)seq(G,2)seq(G,0)$ and $seq(G,2)seq(G,0)seq(G,1)$.
\end{enumerate}
\end{lemma}
\begin{proof}
\paragraph{}
We first prove $(1)$. Let $\phi : V(G) \rightarrow V(G)$ be described by $(0,e,\pi_1, e)$ where $e$ denotes the identity permutation and $\pi_1$ is the permutation which sends $seq(G,0,0)$ to $0,1,\dots,n-1$. Let $H$ be the hypergraph underlying an instance of $C3GSM$ such that $G \equiv H$ under $\phi$. Since $G$ is the lexicographic minimum of $[G]$ and $H \in [G]$, $seq(G,0,0) \leq_{lex} seq(H,0,0) = 0,1,\dots,n-1$. Since $0,1,\dots,n-1$ is the lexicographic minimum sequence possible we have $seq(G,0,0) = 0,1,\dots,n-1$.
\paragraph{}
To prove $(2)$ we do something similar to $(1)$. Let $\phi$ be described by $(0,e,e,\pi_2)$ where $\pi_2$ is the permutation which sends $seq(G,1,0)$ to $0,1,\dots,n-1$. Let $H$ be the hypergraph underlying an instance of $C3GSM$ such that $G \equiv H$ under $\phi$. Observe that since $r = 0$ and $\pi_0 = \pi_1 = e$, $seq(G,0) = seq(H,0)$. Since $G$ is the lexicographic minimum of $[G]$ and $H \in [G]$, $seq(G,1,0) \leq_{lex} seq(H,1,0) = 0,1,\dots,n-1$. Since $0,1,\dots,n-1$ is the lexicographic minimum sequence possible we have $seq(G,1,0) = 0,1,\dots,n-1$.
\paragraph{}
We now turn our attention to proving $(3)$. Observe that $seq(G,0,a) <_{lex} seq(G,0,b)$. Let $\phi$ be described by $(0,\pi_0,e,e)$ where $\pi_0$ which transposes $a$ and $b$. Let $H$ be the hypergraph underlying an instance of $C3GSM$ such that $G \equiv H$ under $\phi$. Suppose for a contradiction that $b<a$. Since $r=0$, $\pi_1=\pi_2=e$, and $\pi_0$ simply transposes $a$ and $b$, this yields 
$$seq(H,0) = seq(G,0,0) \dots seq(G,0,b-1)seq(G,0,a)seq(G,0,b+1) \dots seq(G,0,a-1) seq(G,0,b) seq(G,0,a+1) \dots seq(G,0,n-1) <_{lex} seq(G,0).$$
This contradicts the lexicographic minimality of $G$.
\paragraph{}
Finally we prove $(4)$. Let $\phi$ be described by $(1,e,e,e)$. We will prove $seq(G)$ is lexicographically at most $seq(G,1)seq(G,2)seq(G,0)$. The proof that $G$ is lexicographically smaller that $seq(G,2)seq(G,0)seq(G,1)$ folllws similarly using $(2,e,e,e)$. Let $H$ be the hypergraph underlying an instance of $C3GSM$ such that $G \equiv H$ under $\phi$. Since $\pi_0=\pi_1=\pi_2=e$ and $r=1$ we have that
$$seq(H) = seq(G,1)seq(G,2)seq(G,0)$$
and thus by the lexicographic minimality of $G$, $seq(G) \leq_{lex} seq(H)$ and the result follows.
\end{proof}
\subsection{Computer Search}\label{sec:computersearch}
\paragraph{}
Our goal now is to describe a computer search procedure to test all instances of $C3GSM$ for a given size $n$ under the assumption that for $k<n$ it is known that all instances of $C3GSM$ have a stable matching (in our case we study $n=5$). A major challenge towards solving this problem computationally is that enumerating all instances of $C3GSM$ would be infeasible. For each vertex there is one possible preference order over the vertices they face for each permutation of $\{1,\dots,n\}$. There are $3n (=15$ in our case) such vertices, and thus there are $(5!)^{15}$ possible problem instances for $C3GSM$ with size $5$. 
\paragraph{}
So what we resolve to do is attempt to eliminate preference systems from consideration using our symmetry and sufficient checks theory from the previous subsections. Our approach is based on the idea that if we relax the total orders to partial orders giving only the first few preferences of each vertex and can find a stable matching then any $C3GSM$ instance that results from completing the partial order will also have said stable matching. The following theory formalizes this idea.
\begin{definition}
A partially specified instance of $C3GSM$ again provides a complete $3$-partite hypergraph $G$, but now we do not know the total orders for each of the vertices of $G$. Herein and henceforth we use the ordered pair labelling for vertices of $G$ as outlined in note \ref{note:labels}. In our partially specified instance there exists some ordered triple of integers $(i,j,k)$, called the indicator of this instance, such that, for all vertices $(x,a)$ where $x<i$ or, $x=i$ and $a\leq j$ we have that $f((x,a), 1), \dots f((x,a), k)$ are the only choices $(x,a)$ has specified. For all other vertices $(x,a)$ (those with $x=i$ and $a > j$ or, $x>i$) we have that $f((x,a), 1), \dots, f((x,a),k-1)$ are the only choices $(x,a)$ has specified.
\end{definition}
\paragraph{}
We give a few quick examples. A partially specified instance of $C3GSM$ with indicator $(i,jk)$ where any of $i,j,k <0$ has specified no choices for any vertex. A partially specified instance with indicator$(2,4,5)$ of size $5$ is in fact a completely specified instance of $C3GSM$ of size $5$. A partially specified instance with indicator $(0,4,1)$ has each vertex in $V(G)_0$ specifying their first choice and no other choice specified.
\begin{definition}
A partially specified instance of $C3GSM$, $I'$, of size $n$ with indicator $(i',j',k')$ is called an extension of partially specified instance of $C3GSM$, $I$,of size $n$ with indicator $(i,j,k)$ if  they both have underlying hypergraph $G$ and the following is satisfied:
\begin{equation} \label{cond:indicator}
(i',j',k') = \begin{cases}
(0,0,k+1), \text{if $i=2$ and $j=n-1$} \\
(i+1, 0, k), \text{else, if $j = n-1$} \\
(i,j+1, k), \text{otherwise},
\end{cases}
\end{equation}
and for all $v \in V(G)$, the order specified for $v$  in $I$ is contained in the order specified for $v$ in $I'$.
\end{definition}
\paragraph{}
For example consider the partially specified instance of $C3GSM$ of size $5$ where the first four vertices of $V(G)_0$ specify their first choice to be $(1,0)$, and the remaining vertices specify no choices. This instance has indicator $(0,3,1)$. The partially specified instances of $C3GSM$ where the first four vertices of $V(G)_0$ specify their first choice to be $(1,0)$ and the vertex $(0,4)$ specifies $f((0,4),1) = k$ where $k \in \{(1,0), \dots, (1,4)\}$ are extensions of our original partially specified instance. In this example the indicator of our original instance satisfies the third case of equation \ref{cond:indicator}. By the second case of \ref{cond:indicator} any extensions of our above mentioned extensions would next have vertex $(1,0)$ specify their first choice. By third case of \ref{cond:indicator} an extension of a system where all first choices are specified and no others would have $(0,0)$ specifying their first two choices and and all others specifying their first choice.
\paragraph{Computer Search Algorithm}
We now describe our procedure for deciding if each instance of $C3GSM$ of size $n=5$ has a stable matching. For any partially specified instance of $C3GSM$, $I$, let $\cE(I)$ denote the list of partially specified instances of $C3GSM$ which are extensions of $I$ and let $\cP(I)$ denote the unique instance that $I$ is an extension of. For uniqueness $\cP$ of a preference system with no preferences specified is itself and its canonical indicator is $(0,-1,1)$. The algorithm is as follows:
\begin{enumerate}
\item Let $I$ be the partially specified instance of $C3GSM$ of size $5$ with indicator $(0,-1,1)$.
\item While $\cE(I) \neq \emptyset$ or $\cP(I) \neq I$
	\begin{enumerate}
	\item If $I$ satisfies a lemma of \ref{subsec:sufficient} which implies $I$ has a stable matching or the graph underlying $I$ is not the lexicographical minimum of its equivalence class by lemma \ref{lemma:nec-symmetry} then set $\cE(I) \leftarrow \emptyset$ and $I \leftarrow \cP(I)$, and continue to the next iteration of step $2$.
	\item Else If $\cE(I) = \emptyset$ then return $I$ as a counterexample to conjecture \ref{conj:stab}
	\item Otherwise let $E \in \cE(I)$. Remove $E$ from $\cE(I)$ and set $I \leftarrow E$. 
	\end{enumerate}
\item If While loop terminates with no counterexample found then conjecture \ref{conj:stab} is proven for $n=5$ and return that it is so.
\end{enumerate}
\paragraph{}
The intuition behind this approach is that we can often conclude that a problem instance has a stable matching without looking at all the preferences of each vertex. Recall, for instance that if there is a cycle of vertices, all respective first choices of each other, then there is a stable matching. The goal is to develop enough theory of lemmas that imply stable matchings as in subsection \ref{subsec:sufficient} that we can eliminate many instances in step $2(a)$ early on in execution. To see the benefit of this suppose we eliminate in step $2(a)$ a partially specified instance that has only specified each vertex's first choice. There are $(4!)^{15}$ possible fully specified instances of $C3GSM$ that can be obtained from said instance via a sequence of extensions that we do not need to check now.
\paragraph{Notes on using lemmas of subsection\ref{subsec:sufficient}}
In step $2(a)$ it important to observe that saying $I$ has a stable matching is a bit of an abuse of notation. What we mean is that any completely specified instance that can be obtained from $I$ by a sequence of extensions has a stable matching. This is actually easier to verify in many cases than in immediately apparent. If you can form a matching in $I$ only  matches vertices with those that they have specified a ranking for (first choice, second choice, etc.) then it is easy to see which vertices they prefer to the matching. When you specify the $k$-th choice of a vertex in an extension all their $k-1$-th choices have previously been specified. Most lemmas only need knowledge of the vertices preferred by a vertex to their partner, and so for them just match as above and check the lemma's conditions directly. There is a notable exception in lemma \ref{lemma:fixing} which we use in a slightly different way.
\paragraph{Fixing Choices With Lemma \ref{lemma:fixing}}
Lemma \ref{lemma:fixing} requires knowledge of a vertex's last choice in order to conclude that the given problem instance has a stable matching. Unfortunately lemmas which rely on last choices don't naturally lend themselves to cutting instances from consideration early on via step $2(a)$. There is an alternative approach to using them though. Suppose you have a partially specified instance $I$ and matching $M$ for which each vertex except for $a$ that is matched in $I$ is matched to a partner that they have already specified their ranking for in $I$, and all vertices $a$ has specified a ranking for are matched in $M$. Also suppose that $M$ satisfies all the requirements of lemma \ref{lemma:fixing} except for the requirement that $a$'s last choice is unmatched in $M$ (as a consequence of not knowing $a$'s last choice). Then by lemma \ref{lemma:fixing} any fully specified instance of $C3GSM$ extended from $I$ wherein $M(a)$ is not the last choice of $a$ is one where $a$'s last choice is unmatched in $M$ and hence has a stable matching. Thus we know we need only consider extensions where $M(a)$ is the last choice of $a$. If you can find another such matching $M'$ with $M'(a) \neq M(a)$ then we know $I$ has a stable matching since the last choice of $a$ cannot simultaneously be $M'(a)$ and $M(a)$.
\paragraph{Results of Computational Experiments}
We implemented our computer search algorithm in Java and attempt to run it. We were unsuccessful in getting the algorithm to terminate but we were able to eliminate many preferences. A week of running showed the algorithm had gotten down to considering the C3GSM instance:
TODO-Show-a-Preference-System
\paragraph{}
TODO-What-to-say-about-results?
