\documentclass[letterpaper,12pt,oneside,onecolumn]{report}
\usepackage{amsmath, amssymb, amstext}
\usepackage{fancyhdr}
\usepackage{algorithm}
\usepackage{algpseudocode}
\usepackage{mathtools}

\DeclarePairedDelimiter{\ceil}{\lceil}{\rceil}

\pagestyle{fancy}


\listfiles

\setlength{\hoffset}{0pt}			% 1 inch left margin
\setlength{\oddsidemargin}{0pt}		% 1 inch left margin
\setlength{\voffset}{0pt}			% 1 inch top margin
\setlength{\marginparwidth}{0pt}	% no margin notes
\setlength{\marginparsep}{0pt}		% no margin notes
\setlength{\textwidth}{6.375in}

\raggedbottom

\rhead{William Justin Toth 642 3} %CHANGE n to ASSIGNMENT NUMBER ijk TO COURSE CODE
\renewcommand{\headrulewidth}{0pt}
%\renewcommand{\baselinestretch}{1.2} % 1.2 line spacing for legibility (optional)

\begin{document}
\section*{1}
\paragraph{}
Let $G$ be a planar graph of girth at least $7$, and minimum degree at least $2$. Suppose for contradiction that every $v \in V(G)$ such that $d(v) = 2$ is not a adjacent to a vertex of degree at most $3$. We may assume without loss of generality that $G$ is connected, as otherwise we may consider in turn each connected component of $G$. Now since $G$ is connected, by the minimum degree of $G$, $G$ is $2$-connected. Thus every face of $G$ is bounded by a cycle by Proposition $4.2.6$ in the text. So by the minimum degree of $G$ every vertex is adjacent to at least two faces.
\paragraph{}
We proceed by a face discharging argument. For every $v \in V(G)$ let $ch(v)$ denote the initial charge of $v$. Set $ch(v) = 2d(v) - 6$ for every $v \in V(G)$. For every $f \in F(G)$ (the set of faces of $G$), denote the initial charge of $f$ by $ch(f)$.  Set $ch(f) = |f| - 6$ for every $f \in F(G)$. Now $\sum_{v\in V(G)} ch(v) + \sum_{f\in F(G)} ch(f) = 2\sum_{v \in V(G)}d(v) - 6|V(G)| +  2\sum_{f \in F(G)}|f| - 6|F(G)| = 4|E(G)| - 6|V(G)| + 2|E(G)| - 6|F(G)| = -6(|V(G)| - |E(G)| + |F(G)|) = -12$. Therefore the total charge is negative.
\paragraph{}
Discharge according to the following rules:
\begin{align}
&\text{Every vertex of degree at least four sends $\frac{1}{2}$ charge to each adjacent face,}\\
&\text{and every face sends one charge to each adjacent vertex of degree two.}
\end{align}
\paragraph{}
Let $ch_F(v)$ denote the final charge of vertex $v$. If $d(v) = 2$ then by rule $(2)$ $v$ receives at least $2$ charge because it is adjacent to at least two faces, and $v$ does not send any charge by rule $(1)$ since its degree is less that four. Thus $ch_F(v) \geq ch(v) + 2 = -2 + 2 = 0$.If $d(v) = 3$ then $v$ does not receive charge by rule $(2)$ nor send charge by rule $(1)$ so $ch_F(v) = ch(v) = 2(3) - 6 = 0$.
\paragraph{} 
If $d(v) \geq 4$ then $v$ does receive any charge by rule $(2)$, but $v$ sends charge by rule $(1)$. The vertex $v$ is adjacent to at most $d(v)$ faces, and thus sends at most $\frac{d(v)}{2}$ charge by rule $(2)$. Hence $ch_F(v) \geq ch(v) - \frac{d(v)}{2} = 2d(v) - 6 - \frac{d(v)}{2} = \frac{3}{2}d(v) - 6 \geq \frac{3}{2}4 - 6 = 0$. Therefore every $v \in V(G)$ has non-negative final charge.
\paragraph{}
Let $ch_F(f)$ denote the final charge of face $f$. Let $f \in F(G)$. Let $t$ denote the number of degree two vertices adjacent to $f$. Let $l$ denote the number of degree four vertices adjacent to $f$. Suppose $|f| = 7$. Then $t \leq 3$ since if $t \geq 4$ then $l\leq 3$ and thus there is a pair of adjacent degree $2$ vertices contradicting our assumption for contradiction. By rules $(1)$ and $(2)$, $f$ sends $t$ charge and receives $\frac{l}{2}$ charge. So $ch_F(f) = ch(f) - t + \frac{l}{2}$. If $t = 3$ then $l = 4$ as otherwise there is a vertex of degree $2$ adjacent to a vertex of degree less than four. In this case $ch_F(f) = 7 - 6 - 3 + \frac{4}{2} = -2 + 2 = 0$. If $t \leq 2$ then $l \geq t$, thus $ch_F(f) = 7 - 6 - t + \frac{l}{2} \geq 1 - t + \frac{t}{2} = 1 -\frac{t}{2} \geq 1 - 1 = 0$.
\paragraph{}
Suppose $|f| \geq 8$. Then $t \leq \frac{|f|}{2}$ by our assumption for contradiction. This can be seen by noticing that degree $2$ vertices have at least one vertex of degree at least four separating them along $f$. Further this implies $l \geq t$. By rule $(1)$, $f$ receives $\frac{l}{2}$ charge. By rule $(2)$, $f$ sends $t$ charge. Therefore $ch_F(f) = ch(f) - t + \frac{l}{2} \geq ch(f) - t + \frac{t}{2} = ch(f) - \frac{t}{2} \geq ch(f) - \frac{|f|}{4} = |f| - 6 - \frac{|f|}{4} = \frac{3}{4}|f| - 6 \geq \frac{24}{4} - 6 = \frac{24}{4} - \frac{24}{4} = 0$. Therefore every $f \in F(G)$ has non-negative final charge, and so the total final charge in non-negative contradicting the initial total charge being negative. $\blacksquare$

\section*{2}
\paragraph{}
Let $G$ be a $k$-critical graph with $k\geq 4$ such that $\Delta \geq 3$. Define $p(G) = (k-1)|V(G)| - 2|E(G)|$. Suppose that $G$ is minimal such that $p(G) = 0$ and $G$ is not a clique. Note then that $G$ is $k-1$-regular. Further since $G$ is not a clique, there exists a vertex with two nonadjacent neighbours.
\subsection*{a}
\paragraph{}
Let $R \subsetneq V(G)$. Then $R$ has a $k-1$-coloring, $\phi$, since $G$ is $k$-critical. Now $R$ has a critical extension $R'$ with extender $W$ and core $X$. Choose $R'$ to be the maximal such critical extension. So by the Potential Extension Lemma, $p(R') \leq p(R) + p(W) - p(X)$. Rearranging we see that $p(R) \geq p(R') + p(X) - p(W)$. Now since $W$ is $k$-critical, the min degree of vertices in $W$ is $k-1$. So $p(W) = (k-1)|V(W)| - 2|E(W)| = (k-1)|V(W)| - \sum_{v \in V(W)} deg(v) \leq (k-1)|V(W)| -(k-1)|V(W)| = 0$. Thus $p(R) \geq p(R') + p(X) + 0 = p(R') + p(X)$.
\paragraph{}
If $V(R') \neq V(G)$ then $V(R') \subsetneq V(G)$. Thus $R'$ has a $k-1$-coloring, $\phi'$, since $G$ is $k$-critical. Thus $R'$ has a critical extension $R''$ with extender $W'$ and core $X'$. Notice that $W' \neq \emptyset$ and $X' \neq \emptyset$ since $W' \cap X' \neq \emptyset$. So $|V(R'')| > |V(R')|$. Now let $W'' = W \cup W'$ and $X'' = X \cup X'$. Then $R'' = R \cup W'' - X''$. Therefore $R''$ is a critical extension of $R$ of size larger than $R'$, contradicting the maximality of $R'$.
\paragraph{}
So $V(R') = V(G)$. Then $|X| = k-1$ so as to delete all identified vertices in $G(\phi,R)$. So $p(R) \geq p(R') + p(X) = (k-1)|V(R')| - 2|E(R')| + (k-1)|V(X)| - 2|E(X)| \geq (k-1)|V(X)| - \sum_{v \in V(X)} deg(v)$. Since $|V(X)| = k-1$, $\Delta(X) \leq k-2$. Thus $p(R) \geq (k-1)|V(X)| - (k-2)|V(X)| = |V(X)| = k-1$. $\blacksquare$
\subsection*{b}
\paragraph{}
Let $z \in V(G)$ such that $x,y \in N(z)$ and $xy \not\in E(G)$. Let $G'$ be the graph formed by identifying the vertices $x$ and $y$ as $v$, and deleting $z$. If $G'$ is $k-1$-colorable then such a coloring could be extended to a $k-1$-coloring of $G$, a contradiction. Thus $G'$ is not $k-1$-colorable. So $G'$ has a $k$-critical subgraph. Let $H'$ be a $k$-critical subgraph of $G'$. If $v \not\in V(H')$ then $H'$ is a $k$-critical subgraph of $G$, contradicting that $G$ is $k$-critical. Thus $v \in V(H')$.
\paragraph{}
Since $H'$ is $k$-critical, the minimum degree of vertices in $H'$ is $k-1$. Therefore $|N(v) \cap V(H'1)| \geq k-1$. For all $w\in V(H')$ such that $w \neq v$, $deg(w) \leq k-1$ in $H'$, since $G$ is $k-1$-regular and no vertices not equal to $v$ can increase in degree in $H'$. Thus $deg(w) = k-1$ in $H'$ since its minimum and maximum degree in $H'$ is $k-1$.  So we have:
\begin{align*}
p(H') &= (k-1)|V(H')| - 2|E(H')| \\
&= (k-1)|V(H')| - sum_{w \in V(H')} deg_{H'}(w)\\
&= (k-1)|V(H')| - (k-1)|V(H') - 1| - deg_{H'}(v)\\
&= k-1 - deg_{H'}(v) 
\end{align*}
\paragraph{}
Now suppose for contradiction that $deg(v) > k-1$ in $H'$. Then $p(H') \leq -1$. Let $K = H' - v + x + y$. Then $V(K) \subseteq V(G-z) \subsetneq V(G)$. So by $(a)$, $p(K) \geq k-1$. But in fact:
\begin{align*}
p(K) &= (k-1)|V(K)| - 2|E(K)| \\
&= (k-1)|V(H')| -(k-1) + k-1 + k-1 -2|E(H')|\\
&+ 2|N(v) \cap V(H')| - 2|N(x) \cap V(H')| - 2|N(y) \cap V(H')| \\
&\leq p(H') + k-1 + 2|N(v) \cap V(H')| - 2|N(x) \cup N(y) \cap V(H')|\\
&\leq p(H') + k-1 + 2|N(v) \cap V(H')| - 2|N(v) \cap V(H')|\\
&= p(H') + k-1\\
&\leq -1 + k - 1\\
&= k-2. 
\end{align*}
Thus we have $p(K) \geq k-1$ and $p(K) \leq k-2$, a contradiction. Therefore $deg(v) = k-1$ in $H'$.
\paragraph{}
Now we have $p(H') = k-1 - deg_{H'}(v) = k-1 - k-1 = 0$. Since $|V(H')| < |V(G)|$ and $H'$ is $k$-critical, by minimality $p(H') = 0$ if and only if $H'$ is a clique. Since $H'$ is $k-1$-regular, $H'$ is in fact a $k$-clique. Further since $|v \cup (N(v) \cap V(H')| = k$, $v \cup (N(v) \cap V(H') = V(H')$. Therefore $H = H' - v$ is $k-1$-clique in $G\backslash \{x,y,z\}$.
\paragraph{}
Suppose for contradiction there exists, $w \in V(H')$ such that $w \in N(x) \cap N(y)$. Then since $H'$ is $k$-critical, $H' -w$ has a $k-1$-coloring, $\phi$. We can extend $\phi$ to a $k-1$-coloring of $H' -w - v + x + y$ by giving $x$ and $y$ the color $\phi(v)$. Now $w$ sees at most $k-2$ colors in $\phi$ so we can extend $\phi$ to a $k-1$ coloring of $H' - v + x + y$.  But $\phi(x) = \phi(y) = \phi(v)$  so we can re-identify $x$ and $y$ as $v$ to obtain a $k-1$ coloring of $H'$, contradicting $H'$ is $k$-critical.
\paragraph{}
Thus $(N(x) \cap N(y)) \cap V(H') = \emptyset$, so $(N(x) \cap N(y)) \cap V(H) = \emptyset$. Further $V(H) = N(v) \cap V(H') = N(x) \cap V(H) \cup N(y) \cap V(H)$. Therefore $H$ is a $k-1$-clique in $G\backslash \{x,y,z\}$ and $N(x) \cap V(H)$, $N(y) \cap V(H)$ partitions $V(H)$. $\blacksquare$
\subsection*{c}
\paragraph{}
Let $H$ be a $(k-1)$-clique by $(b)$ for non-adjacent $x,y$. Let $G' = G\backslash V(H) + xy$. Suppose for contradiction that $G'$ is $k-1$-colorable. Call such a coloring $\phi_{G'}$. Since $H$ is a $k-1$-clique, $H$ is $k-1$-colorable. Call such a coloring $\phi_{H}$. 
\paragraph{}
Before we proceed note that $N(x) \cap V(H) \neq \emptyset$ and $N(y) \cap V(H) \neq \emptyset$. As otherwise since $z \in N(x) \cap N(y)$ and $z \not \in V(H)$ and thus $N(x)$ and $N(y)$ each contribute at most $k-2$ vertices to $H$, neither set can contribute $0$ vertices. If either set contributed $0$ vertices to $V(H)$, then $H$ would have at most $k-2$ vertices, but it has $k-1$ vertices.
\paragraph{}
Let $C_x = \{\phi_H(v) : v \in N(x) \cap V(H) \}$ and let $C_y = \{\phi_H(v) : v \in N(y) \cap V(H) \}$. If $\phi_{G'}(x) \not\in C_x$ and $\phi_{G'}(y) \not\in C_y$ then we can extend $\phi_{G'}$ to a $k-1$-coloring of $G + xy$ and therefore $G$ using $\phi_H$.