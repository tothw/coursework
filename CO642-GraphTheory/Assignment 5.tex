\documentclass[letterpaper,12pt,oneside,onecolumn]{article}
\usepackage[margin=1in, bottom=1in, top=1in]{geometry} %1 inch margins
\usepackage{amsmath, amssymb, amstext}
\usepackage{fancyhdr}
\usepackage{algorithm}
\usepackage{algpseudocode}
\usepackage{mathtools}

\DeclarePairedDelimiter{\ceil}{\lceil}{\rceil}
\DeclarePairedDelimiter\floor{\lfloor}{\rfloor}

%Macros
\newcommand{\A}{\mathbb{A}} \newcommand{\C}{\mathbb{C}}
\newcommand{\D}{\mathbb{D}} \newcommand{\F}{\mathbb{F}}
\newcommand{\N}{\mathbb{N}} \newcommand{\R}{\mathbb{R}}
\newcommand{\T}{\mathbb{T}} \newcommand{\Z}{\mathbb{Z}}
\newcommand{\Q}{\mathbb{Q}}
 
 
\newcommand{\cA}{\mathcal{A}} \newcommand{\cB}{\mathcal{B}}
\newcommand{\cC}{\mathcal{C}} \newcommand{\cD}{\mathcal{D}}
\newcommand{\cE}{\mathcal{E}} \newcommand{\cF}{\mathcal{F}}
\newcommand{\cG}{\mathcal{G}} \newcommand{\cH}{\mathcal{H}}
\newcommand{\cI}{\mathcal{I}} \newcommand{\cJ}{\mathcal{J}}
\newcommand{\cK}{\mathcal{K}} \newcommand{\cL}{\mathcal{L}}
\newcommand{\cM}{\mathcal{M}} \newcommand{\cN}{\mathcal{N}}
\newcommand{\cO}{\mathcal{O}} \newcommand{\cP}{\mathcal{P}}
\newcommand{\cQ}{\mathcal{Q}} \newcommand{\cR}{\mathcal{R}}
\newcommand{\cS}{\mathcal{S}} \newcommand{\cT}{\mathcal{T}}
\newcommand{\cU}{\mathcal{U}} \newcommand{\cV}{\mathcal{V}}
\newcommand{\cW}{\mathcal{W}} \newcommand{\cX}{\mathcal{X}}
\newcommand{\cY}{\mathcal{Y}} \newcommand{\cZ}{\mathcal{Z}}

%Page style
\pagestyle{fancy}

\listfiles

\raggedbottom

\rhead{William Justin Toth 642 A5} %CHANGE n to ASSIGNMENT NUMBER ijk TO COURSE CODE
\renewcommand{\headrulewidth}{1pt} %heading underlined
%\renewcommand{\baselinestretch}{1.2} % 1.2 line spacing for legibility (optional)

\begin{document}
\section*{1}
\subsection*{a}
\paragraph{}
Let $k \in \Z$ such that $k \geq 0$. Choose $n = R(k)$. By Ramsey's Theorem $n$ exists. Let $T$ be a tournament on $n$ vertices. Number the vertices arbitrarily. That is write $V(T) = \{1, \dots, n\}$. Color the edges of $T$ as follows: for every $(i,j) \in E(T)$, if $i<j$ color $(i,j)$ red and otherwise (that is, if $i>j$) color $(i,j)$ blue. Now by the definition of $R(k)$, $T$ contains either a red $K_k$ or blue $K_k$.
\paragraph{}
Suppose that $T$ contains a red $K_k$. Let $T'$ be directed subgraph of $T$ induced by the red $K_k$. We claim that $T'$ is an acyclic tournament. Clearly, $T'$ is an orientation of $K_k$ so it remains to show $T'$ is acyclic. Suppose for a contradiction that $T'$ has a directed cycle $C$. Let $(i,j) \in E(C)$. Since $(i,j)$ is red, $i < j$. Since $(i,j) \in E(C)$, $P = C -(i,j)$ is a directed $j-i$ path.
\paragraph{Claim 1.1}
For all $i,j \in V(T')$, if $P$ be a directed monochromatic red $j-i$ path then $j<i$.
\paragraph{Proof of Claim 1.1}
Proceed by induction on $|E(P)|$. If $|E(P)| = 1$ then $(j,i)$ is a red arc. So $j<i$. Suppose for induction that for all directed monochromatic red $j-l$ paths $P'$ if $1\leq |E(P')| < |E(P)|$ then $j<l$. Let $(l,i) \in E(P)$. Then $l \neq j$ since $1 < |E(P)|$. Since $(l,i)$ is a red arc, $l<i$. Now $jPl$ is a directed monochromatic red $j-l$ path such that $1\leq |E(jPl)| < |E(P)|$, so by the induction hypothesis $j<l$. Thus by transitivity $j < i$. Therefore by the principle of mathematical induction for all $i,j \in V(T')$, if $P$ be a directed monochromatic red $j-i$ path then $j<i$. $\blacksquare$
\paragraph{}
Now $P$ is a directed monchromatic red $j-i$ path, so by Claim $1.1$, $j<i$. But then we have $i<j$ and $j<i$. That is, by transitivity $i<i$, a contradiction. Thus $T'$ has no directed cycle. Therefore $T'$ is acyclic.
\paragraph{}
Suppose that $T$ contains a blue $K_k$. This case is symmetric to the red case, but it is included for completeness. Let $T'$ be directed subgraph of $T$ induced by the blue $K_k$. We claim that $T'$ is an acyclic tournament. Clearly, $T'$ is an orientation of $K_k$ so it remains to show $T'$ is acyclic. Suppose for a contradiction that $T'$ has a directed cycle $C$. Let $(i,j) \in E(C)$. Since $(i,j)$ is blue, $i > j$. Since $(i,j) \in E(C)$, $P = C -(i,j)$ is a directed $j-i$ path.
\paragraph{Claim 1.2}
For all $i,j \in V(T')$, if $P$ be a directed monochromatic blue $j-i$ path then $j>i$.
\paragraph{Proof of Claim 1.1}
Proceed by induction on $|E(P)|$. If $|E(P)| = 1$ then $(j,i)$ is a blue arc. So $j>i$. Suppose for induction that for all directed monochromatic blue $j-l$ paths $P'$ if $1\leq |E(P')| < |E(P)|$ then $j>l$. Let $(l,i) \in E(P)$. Then $l \neq j$ since $1 < |E(P)|$. Since $(l,i)$ is a blue arc, $l>i$. Now $jPl$ is a directed monochromatic blue $j-l$ path such that $1\leq |E(jPl)| < |E(P)|$, so by the induction hypothesis $j>l$. Thus by transitivity $j > i$. Therefore by the principle of mathematical induction for all $i,j \in V(T')$, if $P$ be a directed monochromatic blue $j-i$ path then $j>i$. $\blacksquare$
\paragraph{}
Now $P$ is a directed monchromatic blue $j-i$ path, so by Claim $1.2$, $j>i$. But then we have $i>j$ and $j>i$. That is, by transitivity $i>i$, a contradiction. Thus $T'$ has no directed cycle. Therefore $T'$ is acyclic. Thus for all nonnegative integers $k$ there is an integer $n$ such that every tournament on $n$ vertices contains an acyclic tournament on $k$ vertices. $\blacksquare$
\subsection*{b}
\paragraph{}
We proceed by induction on $k$ to show that for all $k\in \Z$ such that $k\geq 0$, if $n = \ceil{2^{k-1}}$ and $T$ is a tournament on $n$ vertices then $T$ contains an acyclic tournament on $k$ vertices. If $k=0$ then $n = \ceil{2^{-1}} = 0$ and the result holds trivially. If $k=1$ then $n = \ceil{2^0} =1$. Any tournament on one vertex is acyclic so again the result holds trivially. 
\paragraph{}
Now let $k \in \Z$ and suppose for induction that for all $j \in \Z$ such that $0 \leq j < k$, if $n = \ceil{2^{j-1}}$ and $T$ is a tournament on $n$ vertices then $T$ contains an acyclic tournament on $j$ vertices. Let $T$ be a tournament on $n=\ceil{2^{k-1}} = 2^{k-1}$ vertices. Let $v \in V(T)$. Let $N^+(v) = \{(v,u):u \in V(T)\}$ and let $N^-(v) = \{(u,v): u\in V(T)\}$. Let $$N^*(v) = argmax_{N \in \{N^+(v), N^-(v)\}}\{|N|\}.$$
Then $\ceil{\frac{n}{2}} \leq |N^*(v)| \leq \ceil{n}-1$. That is, $\ceil{2^{k-2}} \leq |N^*(v)| \leq \ceil{2^{k-1}} - 1$. Let $N$ be a size $\ceil{2^{k-2}}$ subset of $N^*(v)$. Then strictly either all arcs from $v$ are forward to vertices in $N$ or all arcs from $v$ are backward to vertices in $N$. Let $T' = T[N]$. Then $0\leq |V(T')| = \ceil{2^{k-2}} < n$ and so by the induction hypothesis $T'$ contains an acyclic tournament on $k-1$ vertices. Let $T^*$ be such a tournament. Then $T^* + v$ is a tournament on $k$ vertices.
\paragraph{}
Suppose for a contradiction that $T^* + v$ contains a directed cycle $C$. Since $T^*$ is acyclic, $v \in V(C)$. Consider the arcs $(u,v), (v,w) \in E(C)$ for $u,w \in V(T^*) = N$. Such arcs exists since $C$ is a directed cycle containing $v$. But then there is a forward arc from $v$ to a vertex in $N$ and a backward arc from $v$ to a vertex in $N$, a contradiction since all arcs from $v$ to vertices of $N$ were chosen to have the same direction. Therefore $T^* + v$ is an acyclic tournament on $k$ vertices. Thus by the principle of mathematical induction for all $k\in \Z$ such that $k\geq 0$, if $n = \ceil{2^{k-1}}$ and $T$ is a tournament on $n$ vertices then $T$ contains an acyclic tournament on $k$ vertices. Therefore for all nonnegative integers $k$ there is an integer $n$ such that every tournament on $n$ vertices contains an acyclic tournament on $k$ vertices. $\blacksquare$

\section*{2}
\paragraph{Lemma 1 (from class)}
If $G$ is a graph, then $\alpha(G) \geq \sum_{v\in V(G)} \frac{1}{d(v) +1}$.
\subsection*{a}
\paragraph{}
Suppose that $G$ is a disjoint union of cliques. Let $k = \kappa(G)$. Let $G_1,\dots, G_k$ denote the connected components of $G$. Since $G$ is a disjoint union of cliques, $\alpha(G) = k$. This follows from observing that each vertex is only independent of vertices in other connected components, and therefore an independent set contains at most one vertex from each connected component (that is, $\alpha(G) \leq k$). This bound is attained by choosing one vertex from each connected component and hence $\alpha(G) = k$. Further we have that:
\begin{align*}
\sum_{v \in V(G)} \frac{1}{d(v) +1} &= \sum_{i=1}^k \sum_{v \in V(G_i)} \frac{1}{d(v) + 1} &\text{since $G_i$ are disjoint}\\
&= \sum_{i=1}^k \sum_{v \in V(G_i)} \frac{1}{(|V(G_i)| -1) + 1} &\text{since $G_i$ is a clique} \\
&= \sum_{i=1}^k \sum_{v \in V(G_i)} \frac{1}{|V(G_i)|} \\
&= \sum_{i=1}^k \frac{|V(G_i)|}{|V(G_i)|} \\
&= \sum_{i=1}^k 1 \\
&= k \\
&=\alpha(G). &\text{since $\alpha(G) = k$ by preceding discussion}
\end{align*}
Therefore if $G$ is a disjoint union of cliques then $\alpha(G) = \sum_{v \in V(G)} \frac{1}{d(v) +1}$.
\paragraph{}
Now suppose to the contrary that $G$ is not a disjoint union of cliques. Then there exists a triple of vertices $x, y, z \in V(G)$ such that $xy \in E(G)$ and $yz \in E(G)$ but $xz \not\in E(G)$. Let $\pi: V \backslash \{x,y,z\} \rightarrow \{4, \dots, n\}$ be a random ordering of $V\backslash \{x,y,z\}$.  Let $$X = \{v \in V(G)\backslash \{x,y,z\}:\text{ for all }w\in V(G)\backslash \{x,y,z\}\text{ if }vw \in E(G)\text{ then }\pi(v) < \pi(w)\}.$$ Then by calculation in proof of Lemma $1$, $E[|X|] \geq \sum_{v\in V(G)\backslash \{x,y,z\}}\frac{1}{d(v)+1}$
\paragraph{}
Let $\pi_1$ be an ordering of $V$ given by:
$$
\pi_1(v) = \begin{cases}
1, &\text{if $v=x$}\\
2, &\text{if $v=y$}\\
3, &\text{if $v=z$}\\
\pi(v), &\text{otherwise}.
\end{cases}
$$
Let $\pi_2$ be an ordering of $V$ given by:
$$
\pi_2(v) = \begin{cases}
1, &\text{if $v=z$}\\
2, &\text{if $v=x$}\\
3, &\text{if $v=y$}\\
\pi(v), &\text{otherwise}.
\end{cases}
$$
For all possible orderings of $V$, $\pi_i$, let $$X_i = \{v \in V(G):\text{ for all }w\in V(G)\text{ if }vw \in E(G)\text{ then }\pi_i(v) < \pi_i(w)\}.$$ Then $x \in X_1$ as it is positioned first under $\pi_1$, $y \not\in X_1$ as $y$ is adjacent to $x$, and $z \not\in X_1$ as $z$ is adjacent to $y$.  Further $z \in X_2$ as it is positioned first under $\pi_2$, $x \in X_2$ as $x$ is not adjacent to $z$, and $y \not\in X_2$ as $y$ is adjacent to $z$. Now we have that $|X_1| = 1 + |X|$ and $|X_2| = 1 + |X_1|$. Then $|X_2| > |X_1|$.
\paragraph{}
Now suppose for a contradiction that for all orderings of $V$, $\pi_i$, that $|X_i| \leq \sum_{v \in V(G)} \frac{1}{d(v)+1}$. By our calculations in the proof of Lemma $1$ in class,  $E[|X_i|] \geq \sum_{v \in V(G)} \frac{1}{d(v)+1}$. But we have that $|X_1| < |X_2| \leq \sum_{v \in V(G)} \frac{1}{d(v)+1}$, and thus
\begin{align*}
E[|X_i|] &= \sum_{\pi_i} Pr(\pi_i) |X_i| \\
&= Pr(\pi_1)|X_1| +  \sum_{\pi_i: i\neq 1} Pr(\pi_i)|X_i| \\
&\leq Pr(\pi_1)|X_1| + \sum_{\pi_i:i\neq 1} Pr(\pi_i)(\sum_{v \in V(G)} \frac{1}{d(v)+1}) \\
&< Pr(\pi_1)(\sum_{v \in V(G)} \frac{1}{d(v)+1}) + \sum_{\pi_i:i\neq 1} Pr(\pi_i)(\sum_{v \in V(G)} \frac{1}{d(v)+1})  \\
&= \sum_{v \in V(G)} \frac{1}{d(v)+1}.
\end{align*}
Therefore $E[|X_i|] <  \sum_{v \in V(G)} \frac{1}{d(v)+1}$. This contradicts that $E[|X_i|] \geq \sum_{v \in V(G)} \frac{1}{d(v)+1}$. Thus there exists an ordering of $V$, $\pi_i$, for which $|X_i| > \sum_{v \in V(G)} \frac{1}{d(v)+1}$. As $|X_i|$ is an independent set this implies that $\alpha(G) > \sum_{v \in V(G)} \frac{1}{d(v)+1}$. Therefore if $G$ is not a disjoint union of cliques then $\alpha(G) > \sum_{v \in V(G)} \frac{1}{d(v)+1}$. So by contraposition if $\alpha(G) = \sum_{v\in V(G)} \frac{1}{d(v) + 1}$ then $G$ is a disjoint union of cliques. Therefore $\alpha(G) = \sum_{v\in V(G)} \frac{1}{d(v) + 1}$ if and only if $G$ is a disjoint union of cliques. $\blacksquare$
\subsection*{b}
\paragraph{Lemma 2.1}
If $G$ is a graph on $n$ vertices then $\omega(G) \geq \sum_{v \in V(G)} \frac{1}{n-d(v)}$.
\paragraph{Proof of Lemma 2.1}
Suppose $G$ is a graph on $n$ vertices. Let $\bar{d}(v)$ denote the degree of $v$ in $\bar{G}$ for all $v \in V(G)$. Note that $\bar{d}(v) = n-1-d(v)$ for all $v \in V(G)$ by the definition of $\bar{G}$. Then we have that:
\begin{align*}
\omega(G) &= \alpha(\bar{G}) \\
&\geq \sum_{v \in V(\bar{G})} \frac{1}{\bar{d}(v) + 1} &\text{By Lemma $1$}\\
&= \sum_{v \in V(G)} \frac{1}{\bar{d}(v) + 1} &\text{Since $V(G) = V(\bar{G})$}\\
&= \sum_{v \in V(G)} \frac{1}{n - 1 - d(v) + 1} &\text{Since $\bar{d}(v) = n-1-d(v)$}\\
&=\sum_{v \in V(G)} \frac{1}{n-d(v)}.
\end{align*}
Thus $\omega(G) = \sum_{v \in V(G)} \frac{1}{n-d(v)}$ as desired. $\blacksquare$
\paragraph{Lemma 2.2}
If $G$ is a graph on $n$ vertices then $\omega(G) = \sum_{v \in V(G)} \frac{1}{n-d(v)}$ if and only if $G$ is $\omega(G)$-partite.
\paragraph{Proof of Lemma 2.2}
Suppose $G$ is a graph on $n$ vertices. Then $\omega(G) = \alpha(\bar{G})$ so $\omega(G) = \sum_{v \in V(G)} \frac{1}{n-d(v)}$ if and only if $\alpha(\bar{G}) = \sum_{v \in V(G)} \frac{1}{n-d(v)}$. But $\alpha(\bar{G}) = \sum_{v \in V(G)} \frac{1}{n-d(v)}$ if and only if $\alpha(\bar{G}) = \sum_{v \in V(G)} \frac{1}{\bar{d}(v) + 1}$. Then by $2a$, $\alpha(\bar{G}) = \sum_{v \in V(G)} \frac{1}{\bar{d}(v) + 1}$  if and only if $\bar{G}$ is a disjoint union of $\alpha(\bar{G})$ cliques. Now $\bar{G}$ is a disjoint union of $\alpha(\bar{G})$ cliques if and only if $G$ is $\alpha(\bar{G})$-partite. Since $\omega(G) = \alpha(\bar{G})$, $G$ is $\alpha(\bar{G})$-partite if and only if $G$ is $\omega(G)$-partite. Therefore $\omega(G) = \sum_{v \in V(G)} \frac{1}{n-d(v)}$ if and only if $G$ is $\omega(G)$-partite. $\blacksquare$
\paragraph{}
Let $G$ be an extremal graph on $n$ vertices such that $G$ is $K_r$ free. We will show that $G = T_{r-1}(n)$ (the Turan graph).
\end{document}
