\documentclass[letterpaper,12pt,oneside,onecolumn]{report}
\usepackage{amsmath, amssymb, amstext}
\usepackage{fancyhdr}
\usepackage{algorithm}
\usepackage{algpseudocode}
\usepackage{mathtools}
\usepackage{centernot}

\DeclarePairedDelimiter{\ceil}{\lceil}{\rceil}

\newcommand{\adjacent}[1][.7em]{\mathrel{\rule[.5ex]{#1}{.4pt}}}
\newcommand{\notadjacent}[1][.7em]{\mathrel{\centernot{\adjacent[#1]}}}

\pagestyle{fancy}

\listfiles

\setlength{\hoffset}{0pt}			% 1 inch left margin
\setlength{\oddsidemargin}{0pt}		% 1 inch left margin
\setlength{\voffset}{0pt}			% 1 inch top margin
\setlength{\marginparwidth}{0pt}	% no margin notes
\setlength{\marginparsep}{0pt}		% no margin notes
\setlength{\textwidth}{6.375in}
\raggedbottom

\rhead{William Justin Toth 642 4} %CHANGE n to ASSIGNMENT NUMBER ijk TO COURSE CODE
\renewcommand{\headrulewidth}{0pt}
%\renewcommand{\baselinestretch}{1.2} % 1.2 line spacing for legibility (optional)

\begin{document}
\section*{1}
\paragraph{}
Let $G$ be a graph with $n$ vertices and minimum degree $\delta > 1$. Let $X \subseteq V(G)$ chosen at random such that for any $v \in V(G)$, $Pr(v \in X) = \frac{ln(\delta)}{\delta + 1}$. For convenience let $p = \frac{ln(\delta)}{\delta + 1}$. Then $Pr(|X| = k ) = {n \choose k} p^k (1-p)^k$, and $E[|X|] = np$. Let $u  \adjacent v$ denote that $u$ is adjacent to $v$ for any $u,v \in V(G)$. Since $G$ has minimum degree $\delta$, for any $u, v \in V(G)$, $Pr(u \adjacent v) \geq \frac{\delta}{n}$. Since $G$ has $n$ vertices, $Pr(u = v) = \frac{1}{n}$. Now $Pr(u \notadjacent v \text{ and } u \neq v) = 1 - (Pr(u \adjacent v) + Pr(u = v)) = \frac{n - 1- \delta}{n}$.
\paragraph{}
Now for any $y \in V(G)$, $Pr(y \not \in N(X) \text{ and } y \not\in X) = \Pi_{x \in X} Pr(y \notadjacent x \text{ and } y \neq x) \leq (\frac{n - 1- \delta}{n})^{|X|}$. Let $Y = \{y \in V(G) | y \not \in N(X) \text{ and } y \not\in X)\}$. Then
\begin{align*}
E[|Y|] &= E[y \mid y \not\in N(x) \text{ and } y \not\in X] \\
&\leq \sum_{y \in V(G)} E[(\frac{n - 1- \delta}{n})^{|X|}] \\
&= n \sum_{k = 1}^n Pr(|X| = k) (\frac{n - 1- \delta}{n})^{k} \\
&= n \sum_{k=1}^n {n \choose k} p^k (1-p)^{n-k} (\frac{n - 1- \delta}{n})^{k} \\
&= n(1-p)^n \sum_{k=1}^n {n \choose k} (\frac{p(n - 1- \delta)}{(1-p)n})^{k} \\
&=  n(1-p)^n [(\frac{p(n - 1- \delta)}{(1-p)n})^{n} - 1] &\text{ since $\sum_{k=1}^n {n \choose k} (x)^{k} = x^n - 1$ for any $x$}\\
&= n(1-p)^n (\frac{n - (\delta + 1)p}{(1-p)n})^n - n(1-p)^n \\
&= n(1 + \frac{-(d+1)p}{n})^n - n(1-p)^n \\
&\leq ne^{-(d+1)p} &\text{ since $n(1-p)^n \geq 0$ and $(1 + \frac{x}{n})^n \leq e^x$ for any $x$.}
\end{align*}
\paragraph{}
Now by our choice of $X$ and $Y$, $X \cup Y$ is a dominating set for $G$. We will proceed by computing the expectation of $|X \cup Y|$ and concluding that there is a set $X \cup Y$ that at most realizes this size. We have:
\begin{align*}
E[|X \cup Y|] &\leq E[|X|] + E[|Y|] \\
&\leq  np + ne^{-(d+1)p} &\text{ by our computations of $E[|X|]$ and $E[|Y|]$}\\
&= \frac{n ln(\delta)}{\delta + 1} + ne^{-\frac{(\delta + 1)ln(\delta)}{\delta+1}} \\
&= n\frac{1 + ln(\delta + 1)}{\delta + 1}.
\end{align*}
\paragraph{}
Therefore there exists a dominating set of size at most $n\frac{1 + ln(\delta + 1)}{\delta + 1}$ as desired. $\blacksquare$
\section*{2}
\paragraph{}
Let $G$ be a graph on $n$ vertices with $nd/2$ edges, where $d \geq 1$. Let $X \subseteq V(G)$ chosen at random such that for any $v \in V(G)$, $Pr(v \in X) = \frac{1}{\delta}$. For convenience let $p = \frac{1}{\delta}$. Then for any $uv \in E(G)$, $Pr(uv \subseteq X) = Pr(u \in X \text{ and } v \in X) = Pr(u \in X)Pr(v \in X)$, with the last equality holding by independence. Therefore $Pr(uv \subseteq X) = p^2$.
\paragraph{}
Let $C$ be a random variable counting the number of edges contained in $G[X]$. Then we have:
\begin{align*}
E[C] &= \sum_{uv \in E(G)} Pr(uv \subseteq X) \\
&= |E(G)|p^2 \\
&= \frac{nd}{2}p^2.
\end{align*}
So there exists an $X \subseteq V(G)$ such that $|E(G[X])| \leq \frac{nd}{2} p^2$. Let $Y$ be a set containing exactly one vertex of each edge in $X$. Then $X \backslash Y$ is an independent set. Now we have:
\begin{align*}
E[|X \backslash Y|] &\geq E[|X|] - E[|Y|] \\
&= np - E[|Y|] &\text{since $E[|X]| = np$}\\
&\geq np - \frac{nd}{2} p^2 &\text{since $E[|Y|] \leq E[C] = \frac{nd}{2} p^2$}\\
&= \frac{n}{d} - \frac{nd}{2d^2} \\
&= \frac{n}{2d}.
\end{align*}
Now since $E[|X \backslash Y|] \geq \frac{n}{2d}$ there exists an independence set of at least this size, as desired. $\blacksquare$

\section*{3}
\paragraph{}
Let $G$ be a graph and let $L$ be a $10d$-list assignment for some $d \geq 1$. Suppose that for each $v \in V(G)$ and $c \in L(v)$ that $|\{ u \in N(v) : c \in L(v) \}| \leq d$. Let $\phi$ be a (possibly improper) random list coloring of the vertices of $G$ such that the vertices are colord uniformly at random from their list assignments. That is, for all $v \in V(G)$, for all $c \in L(v)$, $Pr(\phi(v) = c) = \frac{1}{10d}$, and for all colors $c \not\in L(v)$, $Pr(\phi(v) = c) = 0$.
\paragraph{}
For each $e = uv \in E(G)$, and color $c$, let $A_{e,c}$ be the event that $\phi(u) = \phi(v) = c$. Then 
\begin{align*}
Pr(A_{e,c}) &= Pr(\phi(u) = c \text{ and } \phi(v) = c) \\
&= Pr(\phi(u) = c)Pr(\phi(v) = c) &\text{ by independence} \\
&\leq \frac{1}{10d}\frac{1}{10d} \\
&= \frac{1}{100d^2}.
\end{align*}
\paragraph{}
The event $A_{uv,c}$ depends on the events $A_{uw, c}$ for each $w \in N(v)$ and the events $A_{wv,c}$ for each $w \in N(u)$. For each vertex $v \in V(G)$, for each $c \in L(v)$ there are at most $d$ vertices in $N(v)$ sharing color $c$, and $|L(v)| = 10d$ so $|N(v)| \leq 10d\cdot d = 10d^2$. Each edge connects two vertices, so the event $A_{e,c}$ depends on at most $2\cdot 10d^2 = 20d^2$ other events. Now $4(\frac{1}{100d^2})(20d^2) = \frac{4}{5} \leq 1$, so by the Local Lemma the $Pr(\text{ no } A_{e,c} \text{ occurs for any }e\in E(G)\text{ and color }c) > 0$. Therefore there exists a proper $L$-coloring of $G$ (as otherwise the aforementioned probability would be $0$, not $>0$). $\blacksquare$
\section*{4}
\subsection*{a}
\paragraph{}
This proof follows the techniques of Diestel in the textbook. We reproduce it here with some added detail for completeness.
\paragraph{Definition}
Let $G$ be a graph. Let $\mathcal{P}_{i,j}$ denote the property that for any disjoint vertex sets $U, W \subseteq V(G)$ satisfying $|U| \leq i$ and $|W| \leq j$, there exists a vertex $v \not\in U \cup W$ that is adjacent to all the vertices in $U$ but to none in $W$.
\paragraph{Lemma 11.3.2} For every constant $p \in (0,1)$ and $i,j \in \mathbb{N}$, asymptotically almost surely $G_{n,p}$ has the property $\mathcal{P_{i,j}}$.
\paragraph{Proof of Lemma 11.3.2}
Let $p \in (0,1)$. Let $i,j \in \mathbb{N}$. Let $U, W \subseteq V(G_{n,p})$ be disjoint and satisfy $|U| \leq i$, and $|W| \leq j$. Let $v \in G - (U \cup W)$. Let $A$ be the event that $v \text{ is adjacent to all the vertices in $U$ but to none in $W$}$. Then 
\begin{align*}
Pr(A) &= p^{|U|}(1-p)^{|W|} \\
&\geq p^i(1-p)^j.
\end{align*}
Hence, the probability that no such $v$ exists is 
\begin{align*}
(1 - p^{|U|}(1-p)^{|W|})^{n-|U|-|W|} &\leq (1-p^i(1-p)^j)^{n-i-j}. &\text{(for $n \geq i + j$)}
\end{align*}
This follows since the corresponding events are independent for different $v$. As there are at most $n^{i+j}$ pairs of such sets $U,W$ in $V(G)$ (encoding sets $U$ of fewer than $i$ vertices as non-injective maps $\{0,\dots,i-1\} \rightarrow \{0, \dots, n-1\}$, etc.), the probability that any such pair $U$, $W$ has no suitable $v$ satisfying the lemma is at most $n^{i+j}(1-p^i(1-p)^j)^{n-i-j}$. Since $1-p^i(1-p)^j < 1$, this probability tends to $0$ as $n \rightarrow \infty$. $\blacksquare$
\paragraph{Main Result (Corollary 11.3.3 in the text)}
Let $k \geq 1 \in \mathbb{Z}$. Let $p \in (0,1)$ be a constant. By Lemma $11.3.2$, $G_{n,p}$ asymptotically almost surely has the property $\mathcal{P}_{2,k-1}$. We will show this implies $k$-connectedness. Any graph satisfying $\mathcal{P}_{2,k-1}$ has order at least $k+2$.  If $W$ is a set of fewer than $k$ vertices, then by the definition of $\mathcal{P}_{2,k-1}$, any two vertices $x,y \in V(G_{n,p}) - W$ can be taken as $U$ (ie, $U = \{x,y\}$ and hence $|U| \leq 2$) and thus have a common neighbour $v \in V(G_{n,p}) - U - W$. In particular, every $x,y \not\in W$ has a common neighbour $v \not\in W$. That is, $W$ does not separate $x$ from $y$, and hence $G_{n,p}$ is $k$-connected asymptotically almost surely. $\blacksquare$
\subsection*{b}
\paragraph{}
Suppose that $c$ is a constant such that $c > 1$, and $p(n) = c\frac{ln(n)}{n}$. Let $A$ be the event that $G_{n,p}$ has an isolated vertex. Let $A_v$ be the event that $v \in V(G_{n,p})$ is an isolated vertex of $G_{n,p}$. Then $A \subseteq \bigcup_{v \in V(G_{n,p})} A_v$. Then for all $v \in V(G_{n,p})$,
\begin{align*}
Pr(A_v) &= Pr(\text{for all } u \in V(G_{n,p})\backslash \{v\}, uv \not\in G_{n,p}) \\
&= (1- p(n))^{n-1} \\
&= (1- c\frac{ln(n)}{n})^{n-1}.
\end{align*}
\paragraph{}
Now by the union bound, $Pr(A) \leq \sum_{v\in V(G_{n,p})} Pr(A_v) = n(1- c\frac{ln(n)}{n})^{n-1}$. We now compute the limit as $n \rightarrow \infty$:
\begin{align*}
\lim_{n \to \infty} Pr(A) &\leq \lim_{n\to \infty} n(1- c\frac{ln(n)}{n})^{n-1} \\
&= \lim_{n\to \infty} \frac{n-cln(n))^{n-1}}{n^{n-2}} \\
&=  \lim_{n\to \infty} e^{ln(\frac{n-cln(n))^{n-1}}{n^{n-2}})} \\
&=  \lim_{n\to \infty} \frac{e^{(n-1)ln(n - cln(n))}}{e^{(n-2)ln(n)}}\\
&=  \lim_{n\to \infty} \frac{e^{(n)ln(n - cln(n))}e^{2ln(n)}}{e^{(n)ln(n)}e^{cln(n)}}\\
&\leq \lim_{n\to \infty} e^{-n}e^{(2-c)ln(n)} &\text{ since $n-cln(n) < n$} \\
& = 0.
\end{align*}
\paragraph{}
Therefore asymptotically almost surely $A$ does not happen, as desired. $\blacksquare$
\section*{5}
\paragraph{}
Let $\mathcal{S} = \{S \subseteq V(G_{n,p}) : |S| = 4\}$. For each $S \in \mathcal{S}$, let $X_S$ be the event that $G_{n,p}[S]$ is isomorphic to $K_4$. Let $X$ be a random variable counting the number of $K_4$ subgraphs of $G_{n,p}$. That is $X = \sum_{S \in \mathcal{S}} X_S$. Now $Pr(X_S) = p^6$ as for all $S \in \mathcal{S}$. So we have:
\begin{align*} 
E[X] &= E[\sum_{S \in \mathcal{S}} X_S] \\
&= \sum_{S \in \mathcal{S}} E[X_S] \\
&= \sum_{S \in \mathcal{S}} Pr(X_S) \\
&= {n \choose 4} p^6.
\end{align*}
\subsection*{a}
\paragraph{}
Suppose that $p(n) << n^{-2/3}$. Then 
\begin{align*}
E[X] &= {n \choose 4} p^6 \\
&\leq n^4 p^6 \\
&<< n^4 n^-\frac{2}{3}6 \\
&= n^0 \\
&= 1.
\end{align*}
Therefore 
\begin{align}
\lim_{n\to \infty}E[X] &= 0.
\end{align}
By Markov's Inequality, $Pr(X \geq 1) \leq E[X]$, and therefore by $(1)$, 
\begin{align*}
\lim_{n \to \infty}Pr(X \geq 1) &= 0.
\end{align*}
Thus $X = 0$ asymptotically almost surely, that is $G_{n,p}$ has no $K_4$ subgraph asymptotically almost surely.
\subsection*{b}
\paragraph{}
Now suppose that $p(n) >> n^{-2/3}$. We consider $E[X^2]$ with the intent of applying the second moment method. We have:
\begin{align}
E[X^2] &= E[(\sum_{S\in \mathcal{S}} X_S)(\sum_{S \in \mathcal{S}} X_S)] \nonumber\\
&= E[\sum_{S_1,S_2 \in \mathcal{S}} X_{S_1}X_{S_2}] \nonumber\\
&= \sum_{S_1,S_2 \in \mathcal{S}} Pr(X_{S_1} \text{ and } X_{S_2}) \nonumber\\
&= \sum_{i = 0}^4 \sum_{S_1,S_2 \in \mathcal{S} : |S_1 \cap S_2| = i} Pr(X_{S_1} \text{ and } X_{S_2}).
\end{align}
\paragraph{}
So by $(2)$ we need to consider the sizes of the intersections of $S_1$ and $S_2$.
\subparagraph{Case $i = 0$}
If $i = 0$ then $Pr(X_{S_1} \text{ and } X_{S_2}) = p^6p^6 = p^{12}$ by independence, and so $\sum_{S_1,S_2 \in \mathcal{S} : |S_1 \cap S_2| = 0} Pr(X_{S_1} \text{ and } X_{S_2}) = {n \choose 8}p^{12}$.
\subparagraph{Case $i=1$}
If $i=1$ then again $Pr(X_{S_1} \text{ and } X_{S_2}) = p^{12}$ by independence. So $\sum_{S_1,S_2 \in \mathcal{S} : |S_1 \cap S_2| = 1} Pr(X_{S_1} \text{ and } X_{S_2}) = {n \choose 4}{n-4 \choose 3}p^{12}$.
\subparagraph{Case $i=2$}
If $i=2$ then $Pr(X_{S_1} \text{ and } X_{S_2}) = p^{12} p^{-1} = p^{11}$ as there is a shared edge. So $\sum_{S_1,S_2 \in \mathcal{S} : |S_1 \cap S_2| = 2} Pr(X_{S_1} \text{ and } X_{S_2}) = {n \choose 4}{n-4 \choose 2}p^{11}$.
\subparagraph{Case $i=3$}
If $i=3$ then $Pr(X_{S_1} \text{ and } X_{S_2}) = p^{12} p^{-3} = p^{9}$ as there is a shared triangle. So $\sum_{S_1,S_2 \in \mathcal{S} : |S_1 \cap S_2| = 3} Pr(X_{S_1} \text{ and } X_{S_2}) = {n \choose 4}(n-4)p^{9}$.
\subparagraph{Case $i=4$}
If $i=4$ then $Pr(X_{S_1} \text{ and } X_{S_2}) = p^6$ as the subgraphs completely overlap. So $\sum_{S_1,S_2 \in \mathcal{S} : |S_1 \cap S_2| = 4} Pr(X_{S_1} \text{ and } X_{S_2}) = {n \choose 4}p^{6}$.
\paragraph{}
So by $(2)$ and the preceding case analysis,
\begin{align}
E[X^2] &= {n \choose 8}p^{12} + {n \choose 4}{n-4 \choose 3}p^{12} + {n \choose 4}{n-4 \choose 2}p^{11} + {n \choose 4}(n-4)p^{9} + {n \choose 4}p^{6} .
\end{align}
Thus, since $E[X]^2 = ({n \choose 4} p^6)^2  = {n \choose 4}^2 p^{12}$, we have by $(3)$ that:
\begin{align*}
\frac{E[X^2]}{E[X]^2} &= \frac{{n \choose 8}p^{12} + {n \choose 4}{n-4 \choose 3}p^{12} + {n \choose 4}{n-4 \choose 2}p^{11} + {n \choose 4}(n-4)p^{9} + {n \choose 4}p^{6}}{{n \choose 4}^2 p^{12}} \\
&= O(\frac{n^8p^{12} + n^7p^{12} + n^6p^{11} + n^5p^{9} + n^4p^{6}}{n^8 p^{12}}) \\
&= O(1 + n^{-1} + n^{-2}p^{-1} + n^{-3}p^{-3} + n^{-4}p^{-6}) \\
&<< O(1 + n^{-1} + n^{-2}n^{2/3} + n^{-3}n^2 + n^{-4}p^4).
\end{align*}
Therefore $lim_{n \to \infty} \frac{E[X^2]}{E[X]^2} = 1$. Since $\frac{Var(X)}{E[X]^2} = \frac{E[X^2]}{[E[X]^2} - 1$,
\begin{align*}
lim_{n \to \infty} \frac{Var(X)}{E[X]^2} &= \lim_{n \to \infty} ( \frac{E[X^2]}{[E[X]^2} - 1) \\
&= 1 - 1 \\
&= 0.
\end{align*}
Now by theorem, $Pr[X = 0] \leq \frac{Var(X)}{E[X]^2}$, and so $\lim_{n \to \infty} Pr[X = 0] \leq \lim_{n \to \infty} \frac{Var(X)}{E[X]^2} = 0$. Therefore $G_{n,p}$ has a $K_4$ subgraph asymptotically almost surely.$\blacksquare$
\end{document}
