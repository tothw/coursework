\documentclass[letterpaper,12pt,oneside,onecolumn]{report}
\usepackage{amsmath, amssymb, amstext}
\usepackage{fancyhdr}
\usepackage{algorithm}
\usepackage{algpseudocode}
\usepackage{mathtools}
\usepackage{centernot}

\DeclarePairedDelimiter{\ceil}{\lceil}{\rceil}

\newcommand{\adjacent}[1][.7em]{\mathrel{\rule[.5ex]{#1}{.4pt}}}
\newcommand{\notadjacent}[1][.7em]{\mathrel{\centernot{\adjacent[#1]}}}

\pagestyle{fancy}

\listfiles

\setlength{\hoffset}{0pt}			% 1 inch left margin
\setlength{\oddsidemargin}{0pt}		% 1 inch left margin
\setlength{\voffset}{0pt}			% 1 inch top margin
\setlength{\marginparwidth}{0pt}	% no margin notes
\setlength{\marginparsep}{0pt}		% no margin notes
\setlength{\textwidth}{6.375in}
\raggedbottom

\rhead{William Justin Toth 642 4} %CHANGE n to ASSIGNMENT NUMBER ijk TO COURSE CODE
\renewcommand{\headrulewidth}{0pt}
%\renewcommand{\baselinestretch}{1.2} % 1.2 line spacing for legibility (optional)

\begin{document}
\section*{1}
\paragraph{}
Let $G$ be a graph with $n$ vertices and minimum degree $\delta > 1$. Let $X \subseteq V(G)$ chosen at random such that for any $v \in V(G)$, $Pr(v \in X) = \frac{ln(\delta)}{\delta + 1}$. For convenience let $p = \frac{ln(\delta)}{\delta + 1}$. Then $Pr(|X| = k ) = {n \choose k} p^k (1-p)^k$, and $E[|X|] = np$. Let $u  \adjacent v$ denote that $u$ is adjacent to $v$ for any $u,v \in V(G)$. Since $G$ has minimum degree $\delta$, for any $u, v \in V(G)$, $Pr(u \adjacent v) \geq \frac{\delta}{n}$. Since $G$ has $n$ vertices, $Pr(u = v) = \frac{1}{n}$. Now $Pr(u \notadjacent v \text{ and } u \neq v) = 1 - (Pr(u \adjacent v) + Pr(u = v)) = \frac{n - 1- \delta}{n}$.
\paragraph{}
Now for any $y \in V(G)$, $Pr(y \not \in N(X) \text{ and } y \not\in X) = \Pi_{x \in X} Pr(y \notadjacent x \text{ and } y \neq x) \leq (\frac{n - 1- \delta}{n})^{|X|}$. Let $Y = \{y \in V(G) | y \not \in N(X) \text{ and } y \not\in X)\}$. Then
\begin{align*}
E[|Y|] &= E[y \mid y \not\in N(x) \text{ and } y \not\in X] \\
&\leq \sum_{y \in V(G)} E[(\frac{n - 1- \delta}{n})^{|X|}] \\
&= n \sum_{k = 1}^n Pr(|X| = k) (\frac{n - 1- \delta}{n})^{k} \\
&= n \sum_{k=1}^n {n \choose k} p^k (1-p)^{n-k} (\frac{n - 1- \delta}{n})^{k} \\
&= n(1-p)^n \sum_{k=1}^n {n \choose k} (\frac{p(n - 1- \delta)}{(1-p)n})^{k} \\
&=  n(1-p)^n [(\frac{p(n - 1- \delta)}{(1-p)n})^{n} - 1] &\text{ since $\sum_{k=1}^n {n \choose k} (x)^{k} = x^n - 1$ for any $x$}\\
&= n(1-p)^n (\frac{n - (\delta + 1)p}{(1-p)n})^n - n(1-p)^n \\
&= n(1 + \frac{-(d+1)p}{n})^n - n(1-p)^n \\
&\leq ne^{-(d+1)p} &\text{ since $n(1-p)^n \geq 0$ and $(1 + \frac{x}{n})^n \leq e^x$ for any $x$.}
\end{align*}
\paragraph{}
Now by our choice of $X$ and $Y$, $X \cup Y$ is a dominating set for $G$. We will proceed by computing the expectation of $|X \cup Y|$ and concluding that there is a set $X \cup Y$ that at most realizes this size. We have:
\begin{align*}
E[|X \cup Y|] &\leq E[|X|] + E[|Y|] \\
&\leq  np + ne^{-(d+1)p} &\text{ by our computations of $E[|X|]$ and $E[|Y|]$}\\
&= \frac{n ln(\delta)}{\delta + 1} + ne^{-\frac{(\delta + 1)ln(\delta)}{\delta+1}} \\
&= n\frac{1 + ln(\delta + 1)}{\delta + 1}.
\end{align*}
\paragraph{}
Therefore there exists a dominating set of size at most $n\frac{1 + ln(\delta + 1)}{\delta + 1}$ as desired. $\blacksquare$
\section*{2}
\paragraph{}
Let $G$ be a graph on $n$ vertices with $nd/2$ edges, where $d \geq 1$. Let $X \subseteq V(G)$ chosen at random such that for any $v \in V(G)$, $Pr(v \in X) = \frac{1}{\delta}$. For convenience let $p = \frac{1}{\delta}$. Then for any $uv \in E(G)$, $Pr(uv \subseteq X) Pr(u \in X \text{ and } v \in X) = Pr(u \in X)Pr(v \in X)$, with the last equality holding by independence. Therefore since $Pr(uv \subseteq X) = p^2$.
\end{document}
