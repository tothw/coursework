\documentclass[letterpaper,12pt,oneside,onecolumn]{report}
\usepackage{amsmath, amssymb, amstext}
\usepackage{fancyhdr}
\usepackage{algorithm}
\usepackage{algpseudocode}
\pagestyle{fancy}

\listfiles

\setlength{\hoffset}{0pt}			% 1 inch left margin
\setlength{\oddsidemargin}{0pt}		% 1 inch left margin
\setlength{\voffset}{0pt}			% 1 inch top margin
\setlength{\marginparwidth}{0pt}	% no margin notes
\setlength{\marginparsep}{0pt}		% no margin notes
\setlength{\textwidth}{6.375in}
\raggedbottom

\rhead{William Justin Toth 642 2} %CHANGE n to ASSIGNMENT NUMBER ijk TO COURSE CODE
\renewcommand{\headrulewidth}{0pt}
%\renewcommand{\baselinestretch}{1.2} % 1.2 line spacing for legibility (optional)

\begin{document}
\section*{1}
\paragraph{}
Let $k$ and $l$ be non-negative integers. Let $G$ be a graph. Suppose that $G$ does not contain $k$ vertex disjoint cycles, each of length at least $l$. Let $H$ be the graph formed by $k$ disjoint copies of the cycle graph of length $l$. Then $G \in Forb(H)$. Since $H$ is planar, by the Grid Theorem, there exists a function $f: \mathbb{Z}^+ \rightarrow \mathbb{Z}^+$ such that $tw(G) \leq f(k)$.
\paragraph{}
Thus there exists a tree-decomposition $(T, X(G))$ of width at most $f(k)$. For every cycle, $C_l$, of length at least $l$,  in $G$ let $S_{C_l} = \{ v \in V(T) : X(v) \cap V(C_l) \neq \emptyset \}$. Note that $S_{C_l} \subseteq V(T)$. Since every cycle of length $l$, $C_l$, is connected, by the definition of tree decompositions, $S_{C_l}$ is a subtree of $T$. Thus $\{S_{C_l}\}_{C_l \in G}$ is a collection of subtrees of $T$. By the Helly Property Lemma (proven in Assignment $1$) either there exists $k$ vertex disjoint trees $S_{C_l}$ or there exists $X \subseteq V(T)$, $|X| \leq k-1$, such that for all $C_l$, $S_{C_l} \cap X \neq \emptyset$. If there are $k$ vertex disjoint $S_{C_l}$ then there are $k$ vertex disjoint cycles in $G$ of length at least $l$, a contradiction. So there exists $X \subseteq V(T)$, $|X| \leq k-1$, such that for all $C_l$, $S_{C_l}\cap X \neq \emptyset$.
\paragraph{}
Let $X' = \bigcup_{v\in X} X(v) \subseteq V(G)$. Suppose there exists a cycle $C_l$ of length at least $l$ such that $X' \cap C_l = \emptyset$. Then $S_{C_l} \cap X = \emptyset$, a contradiction. Thus every cycle of length at least $l$, $C_l$, in $G$ is such that $X' \cap C_l \neq \emptyset.$ Then $G-X$ has no cycle of length $\geq l$. Now $|X'| = |\bigcup_{v\in X} X(v)| \leq |X|\cdot (tw(G) + 1) \leq (k-1)(f(k) + 1)$.  Choosing $h = (k-1)(f(k) + 1) \in \mathbb{Z}$ (independent of $G$) gives $|X| \leq h$ as desired. $\blacksquare$ 

\section*{4}
\paragraph{}
Let $G$ be a graph and let $\phi$ be a $k$-coloring of $G$. Let $G_{ij}$ denote the subgraph of $G$ induced by the vertices colored $i$ and $j$ in $\phi$.
\subsection*{a}
\paragraph{}
Let $e = \{u,w\} \in E(G)$. Then there exists a unique $i$ such that $\phi(u) = i$ and a unique $j$ such that $\phi(w) = j$. Thus $e \in E(G_{ij})$ and $e \not\in E(G_{kl})$ for $k \neq i$ or $l \neq j$. That is, $e$ is counted in exactly one $G_{ij}$. Therefore $\sum_{i \neq j} |E(G_{ij})| = \sum_{e \in E(G)} \sum_{G_{ij}: e \in E(G_ij)} 1 = \sum_{e \in E(G)} 1 = |E(G)|$.
\paragraph{}
Let $v \in V(G)$. Then there exists a unique $i$ such that $\phi(v) = i$. So for all $j\neq i$, $v \in V(G_{ij})$. There are $k-1$ such $G_{ij}$. Thus $\sum_{G_{ij}: v \in V(G_{ij})} 1 = k-1$. Therefore $\sum_{i\neq j} |V(G_{ij})| = \sum_{v \in V(G)} \sum_{G_{ij} : v \in V(G_{ij})} 1 = \sum_{v \in V(G)} k-1 = (k-1)|V(G)|$.
\paragraph{}
Hence combining the two previous summation results we obtain the result $\sum_{i\neq j}|E(G_{ij})| - |V(G_{ij})| = \sum_{i\neq j} |E(G_{ij})| - \sum_{i\neq j} |V(G_{ij})| = |E(G)| - (k-1)|V(G)|$. $\blacksquare$
\subsection*{b}
\paragraph{Lemma 4.1}
Let $G$ be a graph. Then $|V(G)| \leq |E(G)| + \kappa(G)$ where $\kappa(G)$ is the number of connected components of $G$.
\paragraph{Proof of Lemma 4.1}
Proceed by induction on $|E(G)|$. If $|E(G)| = 0$ then each vertex of $G$ is in its own connected component. Hence $|V(G)| = \kappa(G) = |E(G)| + \kappa(G)$. Now suppose for induction that for all graphs, $H$, such that $0 \leq |E(H)| < |E(G)|$ that $|V(H)| \leq |E(H)| + \kappa(H)$. Let $e \in E(G)$. Let $G' = G - e$. Then $|E(G')| = |E(G)| - 1 < |E(G)|$. So by the induction hypothesis $|V(G')| \leq |E(G')| + \kappa(G')$. Now $V(G') = V(G)$ so in fact we have $|V(G)| \leq |E(G)| - 1 + \kappa(G')$. If $e$ is a bridge then $\kappa(G') = \kappa(G) + 1$. Otherwise $\kappa(G') = \kappa(G) \leq \kappa(G) + 1$. In either case $\kappa(G') \leq \kappa(G) + 1$. So finally we have $|V(G)| \leq |E(G)| -1 + \kappa(G) + 1 \leq |E(G)| + \kappa(G)$ as desired. $\blacksquare$ 
\paragraph{}
Let $k \in \mathbb{Z}$ such that $k \geq 2$. Choose $m =$ $\frac{2}{k(k-1)} > 0$. Let $\epsilon > 0$. Suppose that $|E(G)| \leq (k - 1 - \epsilon) |V(G)|$. Then we have $|E(G)| - (k-1)|V(G)| \leq -\epsilon |V(G)|$. So by $4(a)$, $\sum_{i \neq j} |E(G_{ij})| - |V(G_{ij})| \leq -\epsilon |V(G)|$. 
\paragraph{}
If for all $i \neq j$, $|E(G_{ij})| - |V(G_{ij})| < - \epsilon m |V(G)|$ then $\sum_{i \neq j} |E(G_{ij})| - |V(G_{ij})| < \sum_{i \neq j } -  \epsilon m |V(G)| < \frac{-\epsilon m |V(G)|}{m}$. The last inequality follows since there are $k \choose 2$ $= \frac{k(k-1)}{2}$ ways to choose $i,j$ such that $i \neq j$. But then $\sum_{i \neq j} |E(G_{ij})| - |V(G_{ij})| < -\epsilon |V(G)|$, a contradiction. Hence there exists $i, j$, such that $i \neq j$, and $|E(G_{ij})| - |V(G_{ij})| \leq - \epsilon m |V(G)|$.
\paragraph{}
Thus we have $\epsilon m |V(G)| \leq |V(G_{ij})| - |E(G_{ij})|$. So by lemma $4.1$, $\epsilon m |V(G)| \leq \kappa(G_{ij})$. Let $Comp$ denote the set of connected components of $G_{ij}$. Then $|Comp| = \kappa(G_{ij})$. Let $\mathcal{K} = \{ K \subseteq Comp \}$. Then $|\mathcal{K}| = 2^{|Comp|} = 2^{\kappa(G_{ij})} \geq 2^{\epsilon m |V(G)|}$. Let $K \in \mathcal{K}$. Let $\phi_{K}$ be the coloring given by $\phi$, except that every connected component in $K$ has its vertices that are colored $i$ in $\phi$ colored $j$, and those colored $j$ in $\phi$ colored $i$. Since $G_{ij}$ is bipartite this is still a valid coloring. Each possible $\phi_{K}$ is a distinct coloring. There are $|\mathcal{K}|$ ways to choose $K$ and thus $|\mathcal{K}|$ such distinct colorings. Therefore there are at least $2^{\epsilon m |V(G)|}$ distinct colorings of $G$. $\blacksquare$
\end{document}
