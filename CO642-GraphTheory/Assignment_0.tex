\documentclass[letterpaper,12pt,oneside,onecolumn]{report}
\usepackage{amsmath, amssymb, amstext}
\usepackage{fancyhdr}
\pagestyle{fancy}

\listfiles

\setlength{\hoffset}{0pt}			% 1 inch left margin
\setlength{\oddsidemargin}{0pt}		% 1 inch left margin
\setlength{\voffset}{0pt}			% 1 inch top margin
\setlength{\marginparwidth}{0pt}	% no margin notes
\setlength{\marginparsep}{0pt}		% no margin notes
\setlength{\textwidth}{6.375in}
\raggedbottom

\rhead{William Justin Toth 642  0} %CHANGE n to ASSIGNMENT NUMBER
\renewcommand{\headrulewidth}{0pt}
%\renewcommand{\baselinestretch}{1.2} % 1.2 line spacing for legibility (optional)

\begin{document}
\section*{1}
\paragraph{}
Let $n\ \geq\ 1$ be an integer. Let $d_1,d_2,\ldots,d_n$ be a positive sequence of integers.
\paragraph{If direction}
Suppose that $d_1,d_2,\ldots,d_n$ is the degree sequence of a tree, $T = (V,E)$.\\
Since $T$ is a graph $d_1 + d_2 + \ldots + d_n\ =\ 2|E|$.\\
Since $T$ is a tree $|E| = n - 1$.\\
Thus $d_1 + d_2 + \ldots + d_n = 2(n-1) = 2n-2$.
\paragraph{Only if direction}
Now suppose that $d_1 + d_2 + \ldots + d_n = 2n-2$. Since addition is commutative, say without loss of generality that $d_1 \leq d_2 \leq \ldots \leq d_n$.
We'll proceed by induction on $n$.
\paragraph{Base cases}
Suppose that $n=1$. Then $d_1 = 2(1) - 2 = 0$. Therefore $d_1$ is the degree sequence of the tree given by a single vertex and no edges.\\
Suppose that $n=2$ (this base case will come up later). Then $d_1 + d_2 = 2(2) -2 = 2$. Since $d_1 \geq 1$ and $d_2 \geq 1$ we conclude that $d_1 = 1$ and $d_2 = 1$. Therefore $d_1, d_2$ is the degree sequence of the tree given by two vertices and a single edge connecting them.\\
\paragraph{Claim 1}
If $n \geq 2$ then $d_1 = 1$.
\paragraph{Proof of claim 1}
Suppose $n \geq 2$. Suppose for contradiction that $d_1 > 1$.\\
Then $d_i \geq 2$ for all $i$ such that $1 \leq i \leq n$.\\
Thus $d_1 + \ldots + d_n \geq 2n > 2n-2$, a contradiction.\\
Thus $d_1 \leq 1$ and $d_1 \geq 1$ (as $d_1$ is a positive integer), so $d_1 = 1$ as desired.$\blacksquare$
\paragraph{Claim 2}
If $n \geq 3$ then there exists an $i$ such that $d_i > 1$.
\paragraph{Proof of claim 2}
Suppose that $n \geq 3$.\\
Suppose for contradiction that for all $i$ such that $1 \leq i \leq n$ we have $d_i \leq 1$.\\
That is each $d_i = 1$ as $d_i$ is a positive integer.\\
Then $d_1 + \ldots + d_n = n < 2n-2$ since $n \geq 3$, a contradiction.$\blacksquare$
\paragraph{Inductive Step}
Suppose that for all $k$ such that $2 \leq k < n$  if $d_1 + d_2 + \ldots + d_n = 2k - 2$ then $d_1,d_2,\ldots,d_k$ is the degree sequence of a tree.\\
Suppose that $n \geq 3$.\\
Consider that $d_1 + d_2 + \ldots + d_n = 2n-2$.\\
Since $d_1 = 1$ by claim 1, we have $d_2 + \ldots + d_n = 2n-3$. Then\
\begin{equation}
d_2 + \ldots + d_n - 1 = 2n-4
\end{equation} 
By claim 2 choose $i$ such that $d_i > 1$. By claim 1 this $i$ is not 1.\\
Set $d'_i = d_i -1$. Notice that $d'_i \geq 1$ by our choice of $i$.\\
By equation $(1)$, $d_2 + \ldots + d_{i-1} + d'_i + d_{i+1} + \ldots + d_n = 2(n-1) - 2$.\\
By the inductive hypothesis, $d_2,\ldots,d_{i-1},d'_i,d_{i+1},\ldots,d_n$ is the degree sequence of a tree $T=(V,E)$.\\
Construct a new tree $T'=(V',E')$ by adding a vertex, $v$, to $T$ and connecting $v$ to the vertex corresponding to $d'_{i}$, $w$. $T'$ is connected, and no cycles are created by this construction so $T'$ is indeed a tree as claimed.\\
Now the degree of $v$ is $1=d_1$, and the degree of $w$ in $T'$ is $d'_{i} = d_i$. Since no other modifications were made to $T$ to obtain $T'$ all other vertices in $T$ maintain the same degree in $T'$.\\
That is to say $d_1,\ldots,d_n$ is the degree sequence of $T'$.
Therefore by the principle of mathematical induction $d_1, d_2, \ldots, d_n$ is the degree sequence of a tree.$\blacksquare$

\section*{2}
\paragraph{}
Let $S = \{x_1,x_2,\ldots,x_n\}$ be a set of points in the plane such that the distance between any two distinct points in $S$ is at least one. Let $d$ be the distance function on the plane. Let $G=(S,E)$ be a graph where $\{x_i,x_j\}$ is in $E$ if and only if $d(x_i,x_j) = 1$.\\
\paragraph{Claim 3}
Let $x$ be in $S$. The degree of $x$ in $G$ is at most $6$.
\paragraph{Proof of claim 3}
Let $x$ be in $S$. Consider distinct $x_i, x_j$ in $S$ such that $d(x,x_i) = 1$ and $d(x,x_j)=1$.\\
Notice that points adjacent to $x$ in $G$ lie on the unit circle centered at $x$.
By the construction of $S$, $d(x_i,x_j) \geq 1$. Consider the points $x_i$ and $x_j$ to be as close as possible, that is $d(x_i,x_j) = 1$.\\
Then $x,x_i,x_j$ are the points of an equilateral triangle. Thus the angle $x_ixx_j$ is of size $\frac{\pi}{3}$ radians.\\
We can fit $2\pi / \frac{\pi}{3} = 6$ such triangles around $x$ at most, and thus at most $6$ points in $S$ lie on the unit circle around $x$.\\
That is to say, the degree of $x$ in $G$ is at most $6$. $\blacksquare$
\paragraph{}
Since $G$ is a graph, $\sum_{x \in S} deg(x) = 2|E|$.\\
By claim 3, $\sum_{x \in S} deg(x) \leq \sum_{i=1}^n 6 = 6n$.\\
Thus $2|E| \leq 6n$, so $|E| \leq 3n$.\\
That is to say there are at most 3n pairs of points at distance exactly one. $\blacksquare$
\end{document}